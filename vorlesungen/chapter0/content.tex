\setcounter{chapter}{-1}
\chapter{Wiederholung}

\begin{defi}
Sei $D \subseteq \CC$ offen. $f\colon D \ra \CC$ heißt \myemph{holomorph}, falls $f$ in jedem $z_0 \in D$ komplex differenzierbar ist, d.\,h.
\[
	f'(z_0) = \lim_{h \to \infty} \frac{f(z_o + h)-f(z_0)}{h}
\]
existiert.
\end{defi}

\begin{satz}[Cauchyscher Integralsatz für Sterngebiete]
Sei $D\subseteq\CC$ ein Sterngebiet, $f\colon D \ra \CC$ holomorph.
Dann gilt
\begin{enumerate}
\item $f$ hat auf $D$ eine Stammfunktion.
\item $\int _ C f(z) \opd z = 0$ für jede stückweise glatte, geschlossene Kurve $C \subseteq D$.
\end{enumerate}
\end{satz}

\begin{satz}[Cauchysche Integralformel]
Sei $D\subseteq\CC$ offen und $\closure{U_r(z_0)} \subseteq D$. Dann gilt für alle $z\in U_r(z_0)$
\[
	f(z)
	= \frac{1}{2\pi i} \int _ C \frac{f(w)}{w-z}\opd w
	\,,
\]
wobei $C$ gegeben ist durch $z_0 + re^{2\pi it}$ für $t\in[0,1]$.
\end{satz}

\noindent Daraus folgen einige Aussagen:
\begin{itemize}
\item $f:D \ra \CC$ holomorph, dann ist $f \in \mathcal{C}^\infty$.
\item \emph{Satz von Taylor}: $f:U_r(z_0) \ra \CC$ holomorph, dann gilt
\[
	f(z) = \sum _{n=0}^\infty a_n(z-z_0)^n \qquad \text{mit } a_n = \frac{f^{(n)}(z_0)}{n!}\,.
\]
\item $f$ ist genau dann auf $D$ holomorph, wenn $f$ auf $D$ analytisch ist ist.
\item Lokale Abbildungseigenschaften holomorpher Funktionen
\begin{itemize}
\item Identitätssatz
\item Satz von der Gebietstreue
\item Maximumsprinzip
\end{itemize}
\end{itemize}

\begin{defi}[Singularitäten]
Sei $D\subseteq \CC$ offen, $f\colon D \ra \CC$ holomorph, $a \not\in D$, $\dot{U}_r(a) \subseteq{D}$.
Dann heißt $a$ eine \myemph{Singularität} von $f$.
Die Klassifikationen einer Singularität sind
\begin{itemize}
\item $a$ ist \emph{hebbar} (Riemannscher Hebbarkeitssatz)
\item $a$ ist ein \emph{Pol} (\,$\lim\limits_{z\to a} \abs{f(z)} = \infty$ wobei $z\not=a$ gelten muss)
\item $a$ ist wesentlich (Casorati-Weierstraß)
\end{itemize}
\end{defi}

\begin{satz}[Laurentzerlegung]
Sei $\mathcal{R} = \Set{z\in\CC \mid r < \abs{z-a} < R}$ ein Ringgebiet mit $0 \leq r < R \leq \infty$, $f\colon \mathcal{R} \ra \CC$ holomorph.
Dann existiert eine eindeutige Zerlegung
\[
	f(z)
	= g(z-a) + h\left(\frac{1}{z-a}\right)
	\qquad z\in\mathcal{R}
	\,,
\]
wobei $g\colon U_R(0) \ra \CC$ der \myemph{Nebenteil} und $h\colon U_{\inv r}(0)\ra\CC$ der \myemph{Hauptteil} holomorph mit $h(0) = 0$.
\end{satz}

\noindent\emph{Anwendung auf Singularitäten:}
$f\colon D \ra \CC$ holomorph und $a$ eine Singularität von $f$.
Dann gibt es $\delta > 0$ mit $\dot U_\delta(a) \subseteq D$. Dann gilt
\[
	f(z) = \sum_{n=-\infty}^\infty a_n(z-a)^n \qquad z\in\dot U_\delta(a)
\]
\begin{itemize}[$\ra$]
\item $a$ ist genau dann hebbar, wenn $a_n = 0$ für alle $n \leq -1$.
\item $a$ ist genau dann ein Pol der Ordnung $m\geq 1$, wenn $a_{-m} \not= 0$ und $a_n = 0$ für alle $n < -m$.
\item $a$ ist genau dann wesentlich, wenn es unendlich viele $n < 0$ gibt mit $a_n \not= 0$.
\end{itemize}

\begin{satz}[Residuensatz]
Sei $D\subseteq\CC$ ein Elementargebiet, $z_1, \ldots, z_k \in D$
\[
	f\colon D\setminus\Set{z_1, \ldots, z_k} \ra \CC
\]
holomorph und $C$ eine glatte geschlossene Kurve in $D\setminus\Set{z_1, \ldots, z_k}$.
Dann gilt
\[
	\int_C f(z) \opd z
	= 2\pi i \sum_{j=1}^k \res\limits_{z=z_j}f \cdot \mathcal{X}(C, z_j)
	\,,
\]
wobei $\res\limits_{z=z_j}f$ das Residuum von $f$ ist und $\mathcal{X}(C, z_j)$ die Umlaufzahl von $C$ um $z_j$ ist.
\end{satz}
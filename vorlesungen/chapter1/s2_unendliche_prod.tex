\section{Unendliche Produkte}

\newdate{26. Oktober 2017}

Gegeben sei eine Folge $(p_n)_{n\in\NN}$ komplexer Zahlen.
Wir wollen nun in sinnvoller Weise das unendliche Produkt
\[
	\prod _ {n=1} ^\infty p_n
\]
definieren.
Ein naheliegender Vorschlag dafür ist:
$\prod _ {n=1} ^\infty p_n$ heißt konvergent, falls die Folge $(P_n)_{n\in\NN}$ der Partialprodukte $P_N = \prod _{n=1}^N p_n$ konvergent ist.
In diesem Fall setzen wir
\[
	\prod_{n=1} ^\infty p_n := \lim_{N\to\infty} P_n =: P\,.
\]

Das Problem was sich mit dieser Definition stellt ist, dass falls einer der Faktoren Null ist, so ist der Wert des Produktes gleich $0$.
Dieses hängt also gar nicht von der Gesamtheit der Faktoren ab.
Ferner möchte man oft $\prod _{n=1}^N p_n$ bzw. $P$ mit der Summe $\sum _{n=1}^N \log p_n$ bzw. mit $\log P$ vergleichen.
Und das geht nur falls $p_n\not=0$ für alle $n\in\NN$ und entsprechend $P\not=0$.
Später wollen wir allerdings holomorphe Funktionen als Produkte darstellen, dies sollte auch möglich sein, wenn diese Nullstellen haben.

\begin{defi}[Unendliches Produkt]
Sei $(p_n)_{n\in\NN}$ eine Folge in $\CC$ derart, dass nur endlich viele der $p_n$ Null sind.
Sei $m\in \NN$ der größte Index mit $p_m = 0$ (und $m:=0$, falls $p_n \not= 0$ für alle $n\in\NN$).
Dann heißt das \myemph[unendliches Produkt]{unendliche Produkt}
\[
	\prod_{n=1}^\infty p_n
\]
konvergent, falls der Limes
\[
	\lim_{\substack{N\to\infty\\ \scriptscriptstyle N \geq m+1}} P_n \text{ mit } P_N = \prod _ {n=m+1} ^N p_n
\]
existiert und ungleich Null ist.
Man setzt dann
\[
	\prod _ {n=1} ^\infty p_n :=
	\begin{cases}
		\lim\limits_{N\to\infty} P_n &\text{falls } m=0 \\
		0 &\text{sonst}
	\end{cases}
	\,.
\]
Dabei ist zu beachten, dass nach Definition ein konvergentes unendliches Produkt den Wert 0 genau dann hat, wenn ein Faktor gleich Null ist.
\end{defi}

\begin{bsp}
\begin{enumerate}
\item Das unendliche Produkt $\prod_{n\geq2}(1-\frac{1}{n^2})$ ist konvergent und hat den Wert $\frac{1}{2}$.
\begin{bewe} Zunächst sind alle Faktoren ungleich Null und
\begin{align*}
	P_n
	&= \prod_{n=2}^N \left(1-\frac{1}{n^2}\right)
	= \prod_{n=2}^N \frac{(n-1)(n+1)}{n^2} \\
	&= \frac{(2\cdot3\cdot\,\ldots\,\cdot(N-1))\,\cdot\,(3\cdot4\cdot\,\ldots\,\cdot(N+1)}{(2\cdot3\cdot\,\ldots\,\cdot N)\,\cdot\,(2\cdot3\cdot\,\ldots\,\cdot N)} \\
	&= \frac{1}{N}\frac{N+1}{2} \\
	&= \frac{1}{2}\left(1+\frac{1}{N}\right) \\
	&\xto{N \to \infty} \frac{1}{2}
\end{align*}
\end{bewe}

\item $\prod_{n\geq1}(1-\frac{1}{n^2}) = 0\cdot \prod_{n\geq2}(1-\frac{1}{n^2})$ ist konvergent und hat Wert 0

\item $\prod_{n=1}^\infty \frac{1}{n}$ ist nicht konvergent in unserem Sinn.
Denn
\[
	P_N
	= \prod_{n=1}^N \frac{1}{n} = \frac{1}{N!}\xto{N\to\infty}0
	\,.
\]
\end{enumerate}
\end{bsp}

\begin{satz}\label{satz:konvergenz-unendlicher-produkte}
Für eine unendliche Reihe $\prod_{n=1} ^\infty p_n$ gilt:
\begin{enumerate}
\item Ist $\prod_{n=1} ^\infty p_n$ konvergent, so gilt notwendigerweise $\lim\limits_{n\to\infty}p_n=1$.
\item Sei $p_n \not= 0$ für alle $n\in \NN$.
Dann ist $\prod_{n=1} ^\infty p_n$ konvergent genau dann, wenn
\[
	\sum _{n=1}^\infty \Log p_n
\]
konvergiert. (Erinnerung $\Log z = \log \abs z + i\Arg z$ der Hauptwert des Logarithmus und $-\pi < \Arg z \leq \pi$ das Argument von $z$.)
Insbesondere ist $\prod_{n=1} ^\infty p_n = P$, so existiert $h\in\ZZ$ so dass
\[
	\sum _{n=1}^\infty \Log p_n = \Log P + 2\pi ih
\]
gilt.
Ist umgekehrt $S = \sum _{n=1}^\infty \Log p_n$, so gilt
\[
	e^S = \prod_{n=1} ^\infty p_n
\]
\end{enumerate}
\end{satz}

\begin{bewe}
\begin{enumerate}
\item Es ist
\[
	p_{N+1}
	= \frac{P_{N+1}}{P_N}
	\xto{N\to\infty} \frac{P}{P}
	= 1
	\,.
\]
für $N\geq m+1$, hierbei benutzt man $p_n \not= 0$ für $n\geq m+1$ und $P \not= 0$.

\item Es gelte
\[
	S
	= \sum _{n=1}^\infty \Log p_n
	\,.
\]
Also $S = \lim \limits _{N\to\infty} S_N$ mit $S_n = \sum_{n=1}^N \Log p_n$.
Da $\exp$ stetig ist, folgt
\begin{align*}
	0
	\not= e^S
	&= \lim _ {N\to \infty} e^{S_n} = \lim_{N\to\infty} e^{\log p_1 + \ldots + \log p_n} \\
	&= \lim _ {N\to \infty} e^{\log p_1} \cdot\,\ldots\,\cdot e^{\log p_n}
	= \lim _ {N\to \infty} (p_1 \cdot\,\ldots\,\cdot p_N) \\
	&= \lim _ {N\to \infty} \prod _{n=1}^N p_n
	= P
	\,.
\end{align*}

Gelte nun andererseits $\prod_{n=1}^\infty p_n = P$.
Wir wollen zeigen, dass $\sum _{n=1}^\infty \Log p_n = \Log P + 2\pi ih$.

Aus $\prod_{n=1}^\infty p_n = P$ folgt
\[
	\frac{\prod_{n=1}^N p_n}{P} \xto{N\to\infty} 1
\]

Sei
\[
	\epsilon_N := \Log \left(\frac{\prod_{n=1}^N p_n}{P}\right)
\]
Wegen der Stetigkeit von $\Log z$ in $z=1$ und $\Log 1 = 0$ folgt
\[
	\lim_{N\to\infty}\epsilon_N = \Log 1 = 0
\]

Wir wollen nun zeigen, dass es für jedes $N\in\NN$ ein $h_N\in\ZZ$ gibt mit
\begin{equation}\label{eq:satz1.6.bewe}
	\epsilon_N = \sum_{n=1}^N \Log p_n - \Log P + 2\pi i h_N
\end{equation}
Zunächst gilt  offensichtlich $\exp \epsilon_N = \frac{\prod_{n=1}^\infty p_n}{P}$.
Nach den Additionstheoremen und wegen $\exp\Log z = z$ gilt außerdem
\[
	\exp \left(\sum_{n=1}^N \Log p_n - \Log P\right)
	= \frac{\prod_{n=1}^\infty p_n}{P}
\]
Für $z, z' \in \CC$ folgt aus $\exp z = \exp z'$ stets, dass $z-z'\in 2\pi i\ZZ$.

Damit folgt dann \eqref{eq:satz1.6.bewe}.

Es gilt
\[
	2\pi i(h_{N+1}-h_N) = \Log p_{N+1} + \epsilon_{N+1}-\epsilon_N \xto{N\to\infty} 0
\]
da alle Einzelterme der rechten Seite gegen 0 gehen.
Da $h_{N+1}, h_N\in\ZZ$ folgt $(h_{N+1} - h_N)_{N\geq 1}$ ist konstant für große $N$, also $h_{N+1} = h_N$ für alle großen $N$, d.\,h. $h_n = h$ für $N$ groß.

Nun gilt wegen \eqref{eq:satz1.6.bewe} und $\lim \epsilon_N = 0$, dass
\[
	\sum_{n=1}^N \Log p_n
	\xto{N\to\infty}
	\Log P - 2\pi i h
	\,.
\]
\end{enumerate}
\end{bewe}

\begin{nota}
Man schreibt oft $p_n = 1 + a_n$. Dann lautet die notwendige Konvergenzbedingung aus dem Satz, dass $a_n \xto{n\to\infty} 0$.
\end{nota}




\newdate{30. Oktober 2017}

\begin{satz}\label{satz:zusammenhang-konvergenz-log-reihe}
Es gilt folgender Zusammenhang
\begin{enumerate}
\item Sei $1+a_n \not= 0$ für $n \geq 1$. Dann ist
\[
	\sum _{n=1}^\infty \Log (1+a_n)
\]
genau dann absolut konvergent, wenn $\sum_{n=1}^\infty a_n$ absolut konvergiert.

\item Sei $\sum_{n=1}^\infty a_n$ absolut konvergiert. Dann ist $\prod _{n=1}^\infty (1+a_n)$ konvergent.
Außerdem ist das Produkt unbedingt konvergent, d.\,h. jede Umordnung konvergiert und hat den gleichen Limes.
\end{enumerate}
\end{satz}

\begin{bewe}
\begin{enumerate}
\item Es gilt
\begin{align*}
	\lim_{h\to0} \frac{\Log (1+h)}{h}
	&= \lim_{h\to0} \frac{\Log (1+h) - \Log 1}{h} \\
	&= \at{\derive \Log z}{z=1}
	= \at{\frac{1}{z}}{z=1}
	= 1
	\,.
\end{align*}
Daher auch
\[
	\lim_{h\to0}\,\abs{\frac{\Log (1+h)}{h}} = 1
	\,.
\]
Falls (Fall 1) $\sum_{n=1}^\infty \abs{a_n}$ oder (Fall 2) $\sum_{n=1}^\infty \abs{\Log (1+a_n)}$ konvergent ist, so folgt $a_n\xto{n\to\infty} 0$.
\emph{Denn} für Fall 1 ist dies gerade die notwendige Konvergenz-Bedingung.
Und für Fall 2 lautet die entsprechende Bedingung $\Log (1+a_n) \xto{n\to\infty} 0$, da $\exp$ stetig ist, folgt
\[
	1 + a_n
	= e ^{\Log (1+a_n)}
	\xto{n\to\infty} e^0
	= 1
	\,.
\]
Also $a_n \xto{n\to\infty} 0$.

Damit folgt, für $\epsilon > 0$, gilt für alle $a_n$ mit $n$ groß genug, dass
\[
	(1-\epsilon)\abs{a_n}
	\leq \abs{\Log(1+a_n)}
	\leq (1+\epsilon)\abs{a_n}
\]

Die Aussage des Satzes folgt jetzt aus dem Majoranten-Kriterium.

\item Ist $\sum_{n=1}^\infty \abs{a_n}$ konvergent, dann gilt $a_n \xto{n\to\infty} 0$, also $\abs{a_n} < \frac{1}{2}$ für alle $n > N$.
Dann $1+a_n \not= 0$ für alle $n > N$, also folgt die Konvergenz von $\prod_{k=N+1}^\infty (1+a_n)$ aus (i) und \autoref{satz:konvergenz-unendlicher-produkte} (ii).
Also ist insbesondere $\prod_{k=1}^\infty (1+a_n)$ konvergent.

Die unbedingte Konvergenz von $\prod_{n\geq N+1}(1+a_n)$  (also auch von $\prod_{n\geq1} (1+a_n)$) folgt wegen (\autoref{satz:konvergenz-unendlicher-produkte} (ii)):
\[
	\sum \abs{\Log (1+a_n)} < \infty \Rla \sum \Log (1+a_n) \text{ ist absolut konvergent}
\]
\end{enumerate}
\end{bewe}

\begin{satz}\label{satz:prod_holomorpher_fkt}
Sei $D \subseteq \CC$ offen und $(f_n)_{n\in\NN}$ eine Folge holomorpher Funktionen $f_n\colon D\ra\CC$ derart, dass die Reihe $\sum_{n=1}^\infty f_n(z)$ auf jedem Kompaktum $K\subseteq D$ gleichmäßig, absolut konvergiert.
Dann ist
\[
	F(z) := \prod_{n=1}^\infty (1+f_n(z)) \qquad z\in D
\]
ein unbedingt konvergentes Produkt und $F$ ist eine auf $D$ holomorphe Funktion.
Insbesondere gilt $F(z) = 0$ genau dann wenn $1+f_n(z) = 0$ für ein $n\in\NN$.
\end{satz}

\begin{bewe}
Unbedingte Konvergenz des Produktes folgt aus \autoref{satz:zusammenhang-konvergenz-log-reihe} (ii) mit $a_n = f_n(z)$ für $z\in D$.

Es verbleibt zu zeigen, dass $F$ holomorph auf $D$ ist.

Sei $U \subseteq D$ offen mit $\closure U \subseteq D$ kompakt.
Es genügt Holomorphie von $F$ für beliebiges solches $U$ zu zeigen.
Da $U$ kompakt ist, ist $\sum_{n=1}^\infty f_n(z)$ auf $\closure U$ (also auch auf $U$) absolut gleichmäßig konvergent.
Nach dem notwendigen Konvergenzkriterium für gleichmäßige Konvergenz konvergiert daher $(f_n)_{n\in\NN}$ auf $U$ gleichmäßig gegen Null.

Es gibt also ein $m\in\NN$, so dass für $n > m$ für $z\in U$ gilt
\begin{equation}\label{satz1.10.bewe}
	\abs{f_n(z)} < 1
\end{equation}
Also (siehe Beweis von \autoref{satz:zusammenhang-konvergenz-log-reihe} (i) mit $\epsilon = \frac{1}{2}$)
\[
	\abs{\Log(1+f_n(z))} \leq \frac{3}{2}\abs{f_n(z)}
\]

Die Reihe $\sum_{n=m+1}^\infty f_n(z)$ ist auf $U$ gleichmäßig absolut konvergent, nach dem Cauchy-Kriterium für gleichmäßige Konvergenz und wegen \eqref{satz1.10.bewe} ist daher
\[
	S_m(z) := \sum_{n=m+1}^\infty \Log(1+f_n(z))
\]
auf $U$ gleichmäßig konvergent.

Nach dem Satz von Weierstraß (FT 1) folgt, dass $S_m(z)$ auf $U$ holomorph ist.
Also ist $e^{S_m(z)} = \prod_{n=m+1}^\infty (1+f_n(z))$ auf $U$ holomorph (siehe Beweis von \autoref{satz:konvergenz-unendlicher-produkte} (ii)).
Damit ist
\[
	F(z) = (1+f_1(z)) \cdot \ldots \cdot (1+f_m(z)) \prod_{n=m+1}^\infty (1+f_n(z))
\]
auf $U$ holomorph.
\end{bewe}

\begin{erin*}
Ist $h\colon U_r(z_0) \ra \CC$ holomorph, $h$ nicht identisch Null, $h(z) = (z-z_0)^m g(z)$ mit $m>0$ und $g(z_0) \not= 0$, so nennt man $\ord_ {z=z_0} h = m$ die Ordnung von $z_0$ bezüglich h.
\end{erin*}

\emph{Problem:} Gegeben $S\subseteq \CC$ diskret. Zu jedem $s\in S$ sei ein $m_s \in \NN$ gegeben.

\emph{Frage:} Gibt es eine ganze Funktion $h\colon \CC \ra \CC$ derart, dass (i) $h(z) = 0$ genau dann, wenn $z\in S$ und (ii) $\ord_{z=s} h = m_s$ für alle $s \in S$.

Man nennt $\Set{(s, m_s) \mid s\in S}$ eine \myemph[Verteilungen!Nullstellenverteilung]{Nullstellenverteilung}.
Und eine Funktion $h$ wie oben heißt Lösung der Nullstellenverteilung.

\emph{Antwort:} Ja! Solche $h$ kann man mit Hilfe von Weierstraß-Produkten konstruieren!

\newdate{2. November 2017}

\begin{satz}[Weierstraß'scher Produktsatz]\label{satz:weierstrasprodukt}
\begin{enumerate}
\item Sei $S\subseteq \CC$ diskret und für jedes $s\in S$ sei ein $m_s \in \NN$ gegeben. Dann hat die Nullstellenverteilung $\Set{(s, m_s) \mid s\in S}$ eine Lösung $h$.
Alle Lösungen erhält man als $H(z) = h(z)\cdot e^{g(z)}$ wobei $h$ eine gegebene Lösung und $g$ ganz ist.

\item Sei $f$ ganz und nicht identisch Null, $S=\Set{z\in\CC \mid f(z) = 0}\subseteq\CC$ (Beachte $S$ ist diskret).
Für $s\in S$ sei $m_s := \ord_{z=s} f(z)$.
Dann gibt es zu jedem $s\in S$ ein Polynom $P_s$ und eine ganze Funktion $g$, so dass gilt
\[
	f(z) =
	\begin{cases}
		\displaystyle \prod_{s\in S} \left(1-\frac{z}{s}\right)^{m_s}\cdot e^{P_s(z)}\cdot e^{g(z)} & 0 \not\in S \\
		\displaystyle z^{m_0} \cdot \prod_{\substack{s\in S\\ \scriptscriptstyle s\not=0}} \left(1-\frac{z}{s}\right)^{m_s}\cdot e^{P_s(z)}\cdot e^{g(z)} & 0 \in S
	\end{cases}
\]
wobei die Produkte rechts \myemph[Weierstraß-Produkt]{Weierstraß-Produkte} genannt werden, diese sind auf Kompakta $K\subseteq \CC$ unbedingt konvergent.
\end{enumerate}
\end{satz}

\begin{bewe}
\begin{enumerate}
\item Falls $S$ endlich ist, so ist das Produkt $\prod _{s \in S} (z-s)^{m_s}$ eine Lösung.
Sei $S$ nun unendlich.
Wir können außerdem annehmen, dass $0 \not\in S$, denn eine Nullstelle in $z=0$ der Ordnung $m_0$ kann man immer durch Multiplikation mit $z^{m_0}$ erzeugen.

Sei $s_1,\ s_2,\ \ldots, s_n,\ \ldots$\ nun eine Abzählung von $S$ mit 
\[
	0 < \abs{s_1} \leq \abs{s_2} \leq ... \to \infty
	\,,
\]
und sei $m_n := m_{s_n}$.

Da die holomorphe Funktion $(1-\frac{z}{s_n})^{m_n}$ auf dem Elementargebiet $U_{\abs{s_n}}(0)$ keine Nullstellen hat, kann man eine holomorphe Funktion $A_k\colon U_{\abs{s_n}}(0) \ra \CC$ finden (nach Funktionentheorie 1), so dass
\[
	\left(1-\frac{z}{s_n}\right)^{m_n}
	= e^{-A_n(z)}
	\qquad \text{für } \abs z < \abs{s_n}
\]
Es ist $e^{-A_n(0)} = 1$, also $A_n(0) \in 2\pi i\ZZ$.
Also kann durch Addition eines ganzzahligen Vielfachen von $2\pi i$ erreicht werden, dass $A_n(0) = 0$.
Die Potenzreihenentwicklung von $A_n$ um $z=0$ ist auf das Kompaktum $\abs z \leq \frac{\abs{s_n}}{2}$ absolut gleichmäßig konvergent.
Also können wir durch Abbruch dieser Reihe ein Polynom $P_n$ finden, so dass für $\abs{z} \leq \frac{\abs{s_n}}{2}$ gilt
\[
	\abs{\left(1-\frac{z}{s_n}\right)^{m_n}\cdot e^{P_n(z)} - 1}
	= \abs{e^{P_n(z) - A_n(z)} - 1}
	\leq \frac{1}{n^2}
	\,.
\]

Da $\sum_{n=1}^\infty \frac{1}{n^2}$ konvergent ist, ist die Reihe
\[
	\sum_{n=1}^\infty \abs{\left(1-\frac{z}{s_n}\right)^{m_n}\cdot e^{P_n(z)} - 1}
\]
auf Kompakta gleichmäßig, absolut konvergent. Durch anwenden von \autoref{satz:prod_holomorpher_fkt} erhalten wir, dass das Produkt
\[
	h(z) := \prod _{n=1}^\infty \left(1-\frac{z}{s_n}\right)^{m_n}\cdot e^{P_n(z)}
\]
auf $\CC$ unbedingt konvergiert und eine holomorphe Funktion ist.
Offenbar ist $h$ eine Lösung der Nullstellenverteilung (Beachte $(1-\frac{z}{s_n})^{m_n} = s_n^{-m_n}(z-s_n)^{m_n}(-1)^{m_n}$).

Sei $H$ nun eine weitere Lösung.
Betrachte $\frac{H(z)}{h(z)}$ für $z\not\in S$. Diese Funktion hat in allen Punkten $s\in S$ hebbare Singularitäten und ist nullstellenfrei.
Da $\CC$ ein Elementargebiet ist, existiert eine ganze Funktion $g$ mit $\frac{H(z)}{h(z)} = e^{g(z)}$, also $H(z) = h(z)\cdot e^{g(z)}$ für $z\in\CC$.



\item Betrachte die Nullstellenverteilung $\Set{(s,m_s) \mid s \in S}$.
Diese hat $f$ nach Definition als Lösung und auch die in (i) konstruierte Lösung.
Nach (i) unterscheiden sich beide Lösungen nur um einen Faktor $e^{g(z)}$ mit $g$ ganz.
Daraus folgt die Behauptung.
\end{enumerate}
\end{bewe}

\begin{beme}
Die holomorphen Funktionen $A_n(z)$ für $\abs z < \abs{s_n}$ sind durch die Bedingung
\[
	\left(1-\frac{z}{s_n}\right)^{m_n} = e^{-A_n(z)} \text{ und } A_n(0) = 0
\]
eindeutig bestimmt.

\emph{Denn} wäre $\widetilde{A}_n$ eine weitere solche Funktion, so würde gelten $e^{-\widetilde{A}_n(z)} = e^{-A_n(z)}$. Damit folgt $e^{\widetilde{A}_n(z) - A_n(z)} = 1$. Also $\widetilde{A}_n(z) - A_n(z) = 2\pi it_z$ mit $t_z \in \ZZ$.
Da $U_{\abs{s_n}}(0)$ zusammenhängend und $\widetilde{A}_n - A_n$ stetig ist, ist $z \mapsto 2\pi it_z$ stetig, also folgt $B := t_z$ konstant.
Also gilt $\tilde{A_n}(z) - A_n(z) = B$.
Und mit $z=0$ folgt $B=0$.

Daher muss gelten $A_n(z) = -m_n\Log(1-\frac{z}{s_n})$ für $\abs z \leq \abs{s_n}$.
\emph{Denn} zunächst ist $-m_n\Log(1-\frac{0}{s_n}) = 0$ klar, außerdem folgt mit den Additionstheoremen 
\[
	e^{m_n\Log(1-\frac{z}{s_n})}
	= e^{\Log(1-\frac{z}{s_n})} \cdot\,\ldots\,\cdot e^{\Log(1-\frac{z}{s_n})}
	= (1-\frac{z}{s_n})^{m_n}
	\,.
\]

Es ist $-\Log(1-z) = \sum_{\nu = 1}^\infty \frac{z^\nu}{\nu}$ für $\abs z < 1$.
Also $A_n(z) = m_n \sum_{\nu = 1}^\infty \frac{1}{\nu}(\frac{z}{s_n})^\nu$ für $\abs z < \abs{s_n}$, die Polynome $P_n$ erhält man durch Abbruch dieser Reihe.
\end{beme}

\begin{koro}
Jede auf $\CC$ meromorphe Funktion $f$ ist als Quotient zweier ganzer Funktionen darstellbar.
\end{koro}

\begin{bewe}
Sei $f$ auf $\CC$ meromorph, $S$ die Polstellenmenge von $f$.
Gelte $S\not= \emptyset$, sonst ist die Aussage klar.
Für $s \in S$ sei $m_s := \ord_{z=s} f(z)$ (dabei ist die Polstellenordnung gemeint, eine natürliche Zahl).
Nach \autoref{satz:weierstrasprodukt} existiert eine ganze Funktion $h$ mit Nullstellen genau in den Punkten aus $S$ und so dass die Nullstellenordnung von $h$ in $z=s$ genau $m_s$ ist.

Sei $g:=f \cdot h$.
Dann ist $g$ ganz, denn die Nullstellen kürzen sich gegen Polstellen.
Es folgt $f = \frac{g}{h}$.
\end{bewe}

\begin{bsp}\label{bsp:nullstellenverteilungen}
Nullstellenverteilungen kann man beispielsweise so berechnen:
\begin{enumerate}
\item Sei $S = \Set{n^2 \mid n \in \NN_0}$, $m_s = 1$ für alle $s\in S$.
Dann ist
\[
	h(z) = z\prod_{n=1}^\infty \left(1-\frac{z}{n^2}\right)
\]
eine Lösung der Nullstellenverteilung, denn $\sum_{n=1}^\infty \frac{z}{n^2}$ ist auf Kompakta gleichmäßig, absolut konvergent.

\item $S = \ZZ$, $m_s = 1$ für alle $s\in S$. Die Reihe $\sum_{n\not=0} \frac{z}{n}$ hat schlechte Konvergenzeigenschaften, man muss also konvergenzerzeugende Faktoren einbauen.
Der lineare Term von $A_n$ ist gleich $\frac{z}{n}$ mit $n\not=0$.
Betrachte also
\begin{align*}
	\left(1-\frac{z}{n}\right)e^{\frac{z}{n}}
	&= \left(1-\frac{z}{n}\right)\left(1+\frac{z}{n}+\frac{1}{2}\left(\frac{z}{n}\right)^2 + \ldots\right) \\
	&= 1 - \frac{1}{2}\left(\frac{z}{n}\right)^2 + \text{ höhere Terme}
\end{align*}
(\emph{formal} $= 1 + (\frac{z}{n})^2B(\frac{z}{n})$, wobei $B(z) = \frac{(1-z)e^z-1}{z^2}$, $B(0) = -\frac{1}{2}$).

Die Reihe $\sum_{n\not=0} (\frac{z}{n})^2B(\frac{z}{n})$ ist auf Kompakta gleichmäßig absolut konvergent, \emph{denn} gelte $\abs z \leq c$, dann gilt $\abs{\frac{z}{n}} \leq c$ und $B$ ist als stetige Funktion auf Kompakta beschränkt, damit gilt
\[
	\sum_{n\not=0} \abs{\left(\frac{z}{n}\right)^2 B\left(\frac{z}{n}\right)}
	\leq C \sum _{n\not=0} \frac{1}{n^2}
	< \infty
	\,.
\]

Also ist
\[
	h(z) = z \prod_{n\not=0} \left(1-\frac{z}{n}\right)\cdot e^{\frac{z}{n}}
\]
eine Lösung der gegebenen Verteilung.

\emph{Beachte} $h(z) = z \prod_{n\geq 1} (1-\frac{z}{n})\cdot e^{\frac{z}{n}} \cdot \prod_{n\leq 1} (1-\frac{z}{n})\cdot e^{\frac{z}{n}} = z \prod_{n\geq 1} (1-\frac{z^2}{n^2})$ da die Produkte unbedingt konvergent sind.

\item Es gilt
\[
	\sin\pi z
	= \pi z \prod_{n\geq 1} \left(1-\frac{z^2}{n^2}\right)
	\qquad \text{für } z \in \CC
\]
\emph{Denn:} Beide Seiten sind Lösungen der Nullstellenverteilung $\Set{(n, 1) \mid n\in\ZZ}$.
Nach \autoref{satz:weierstrasprodukt} existiert eine ganze Funktion $g$, so dass
\[
	e^{g(z)} \sin\pi z
	= \pi z \prod_{n\geq 1} \left(1-\frac{z^2}{n^2}\right)
	\qquad z\in\CC
\]

Beachte $\frac{\sin\pi z}{\pi z}$ hat in $z=0$ eine hebbare Singularität (Taylorentwicklung) und dort den Wert 1.
Schreibe 
\[
	e^{g(z)} \cdot \frac{\sin\pi z}{\pi z}
	= \prod_{n\geq 1} \left(1-\frac{z^2}{n^2}\right)
	\qquad z\in\CC
	\,.
\]
Zeige nun $g$ konstant und $g\equiv 0$.
Zweites folgt aus erstem durch auswerten in $z=0$ aus, dann gilt $e^{g(0)} = 1$, also $g(0) \in 2\pi i \ZZ$.
Addiere ganzzahliges Vielfaches von $2\pi i$ und erreiche $g(0) = 0$.
Im folgenden gelte deshalb $g(0) = 0$.

Für $\abs z < 1$ sind alle Faktoren rechts ungleich Null, damit folgt aus \autoref{satz:konvergenz-unendlicher-produkte} (ii):
\[
	\sum _ {n=1}^\infty \Log\left(1-\frac{z^2}{n^2}\right)
	= \Log\left(e^{g(z)}\cdot \frac{\sin\pi z}{\pi z}\right) + 2\pi it_z
\]
mit $t_z \in \ZZ$.
Damit erhalten wir
\[
	\sum_{n=1}^\infty \Log\left(1-\frac{z^2}{n^2}\right)
	= \Log e^{g(z)} + \Log\left(\frac{\sin\pi z}{\pi z}\right) + 2\pi it_z
\]
für $\abs z < \delta$ und $0 < \delta$ klein.
\emph{Denn} $\Log(w_1w_2) = \Log w_1 + \Log w_2$ für $w_1$, $w_2$ nahe bei 1,
beachte $g(0) = 0$ und $g$ stetig.
Also gilt
\[
	\sum_{n=1}^\infty \Log\left(1-\frac{z^2}{n^2}\right)
	= g(z) + \Log\left(\frac{\sin\pi z}{\pi z}\right) + 2\pi it_z
	\,,
\]
da $\Log e^{g(z)} = g(z)$ für $z$ bei $0$, denn $\Log$ und $\exp$ sind zueinander invers.

Da die Reihe links absolut lokal gleichmäßig konvergiert, ist die Funktion stetig, also ist auch $z \mapsto 2\pi it_z$ stetig.
Aber $t_z\in\ZZ$ und $U_\delta(0)$ ist zusammenhängend, also ist $t_z = t\in\ZZ$ konstant.

Nach Weierstraß darf gliedweise differenziert werden:
\[
	\sum_{n=1}^\infty \frac{-2\frac{z}{n^2}}{1-\frac{z^2}{n^2}}
	= g'(z) + \frac{1}{\frac{\sin\pi z}{\pi z}} \derive \left(\frac{\sin\pi z}{\pi z}\right)
\]
Also
\begin{align*}
	\sum _{n=1}^\infty \frac{2z}{z^2-n^2}
	&= g'(z) + \frac{\pi z}{\sin\pi z} \cdot \frac{\pi\cos(\pi z)\cdot \pi z - \pi \sin(\pi z)}{\pi^2 z^2} \\
	&= g'(z) + \pi \cot \pi z - \frac{1}{z}
	\qquad z \in \dot{U}_\delta(0)
	\,.
\end{align*}

Wir kennen bereits die Partialbruchzerlegung des Kotangens (\autoref{bsp:partialbruch_cot}):
\begin{align*}
	\pi \cot \pi z
	&= \frac{1}{z} + \sum _{n\not=0} \left(\frac{1}{z-n} + \frac{1}{n}\right) \\
	&= \frac{1}{z} + \sum _{n\geq 1} \left(\frac{1}{z-n} + \frac{1}{n} + \frac{1}{z+n} - \frac{1}{n}\right)
	= \frac{1}{z} + \sum_{n\geq 1} \frac{2z}{z^2-n^2}
	\,.
\end{align*}

Damit folgt $g'(z) = 0$ für $z\in \dot{U}_\delta(0)$.
Also ist $g$ konstant auf $\dot{U}_\delta(0)$.
Damit ist $g$ bereits auf ganz $\CC$ konstant, wegen des Identitätssatz.
Insgesamt haben wir also $g\equiv 0$ (wegen $g(0) = 0$).
\end{enumerate}
\end{bsp}
\section{Gamma-Funktion}

\emph{Ausgangsproblem:} Man gebe eine \emph{vernünftige} Interpolationsfunktion für die Fakultät $n\mapsto n!$ für $n\in\NN_0$ an.
D.\,h. man sucht eine (zumindest stetige Funktion) $\Gamma$ auf $\Set{x\in\RR \mid x > 0}$ mit
\begin{enumerate}[a)]
\item $\Gamma(n) = (n-1)!$ für $n\in\NN$,
\item $\Gamma(x+1) = x\Gamma(x)$ für alle $x>0$ (Grundeigenschaft der Fakultät).
\end{enumerate}

Welche Eigenschaften hat $\Gamma$?
Kann $\Gamma$ auf $\Set{z\in\CC \mid \Re(z) > 0}$ holomorph oder meromorph fortgesetzt werden?
Vielleicht sogar auf ganz $\CC$?
Durch welche Eigenschaften ist $\Gamma$ charakterisiert?

\emph{Heuristische Vorgehensweise:}
Für $n$ klein ist $(n-1)! = \Gamma(n)$ \myquote{klein}, nicht klar wie man interpolieren sollte.
Für $n$ groß ist $\Gamma(n) = (n-1)!$ \myquote{riesengroß}, also wächst die Funktion \myquote{gleichmäßiger}und sollte daher einfacher zu interpolieren sein!

Sei $x\in\NN$ fest, $N\in\NN$.
Dann soll gelten
\begin{align*}
	\Gamma(x+N)
	&= (N+x-1)\cdot\,\ldots\,\cdot(N+1)\cdot N \cdot (N-1)! \\
	&= N^x\underbrace{\left(1+\frac{x-1}{N}\right)}_{\to 1}\underbrace{\left(1+\frac{x-2}{N}\right)}_{\to 1} \cdot\,\ldots\,\cdot \underbrace{\left(1+\frac{1}{N}\right)}_{\to 1}\cdot 1 \cdot (N-1)!
\end{align*}
Damit erhalten wir
\[
	\lim_{N\to\infty} \frac{N^x(N-1)!}{\Gamma(x+N)}
	= 1
	\,.
\]
Außerdem soll gelten
\begin{align*}
	\Gamma(x+N)
	&= (x + N -1)! \\
	&= (x + N -1)\cdot(x+N-2)\cdot\,\ldots\,\cdot (x+1)\cdot x\cdot \Gamma(x)
	\,.
\end{align*}
Durch einsetzen erhalten wir
\begin{equation}\label{gamma:herleitung_limes}
	\Gamma(x)
	= \lim _ {N\to\infty} \frac{N^x(N-1)!}{x(x+1)\ldots(x+N-1)}
	\,.
\end{equation}
Also probiere man \eqref{gamma:herleitung_limes} als Definition für $\Gamma(x)$ oder sogar für $\Gamma(z)$ mit $z\in\CC, z \not= 0, -1, -2, \ldots$ aus, vorausgesetzt der Limes existiert.

\begin{satz-noind}\label{satz:gamma-fkt}
Sei
\[
	D_{-\NN_0}
	:= \Set{z\in\CC \mid z\not= 0, -1, -2, \ldots}
	\,.
\]
\begin{enumerate}
\item Durch
\[
	\Gamma(z)
	= \lim _ {N\to\infty} \frac{N^z(N-1)!}{z(z+1)\ldots(z+N-1)}
\]
für $z\in D_{-\NN_0}$ wird eine holomorphe Funktion erklärt, diese heißt \myemph{Gamma-Funktion}. Obige Darstellung heißt \myemph[Gamma-Funktion!Gauß'sche Produktdarstellung]{Gauß'sche Produktdarstellung} von $\Gamma(z)$.

\item Es gilt für $z\in D_{-\NN_0}$
\[
	\Gamma(z) = \frac{1}{z} \cdot \prod_{n=1}^\infty \frac{\left(1+\frac{1}{n}\right)^z}{1+\frac{z}{n}}
\]
die \myemph[Gamma-Funktion!Eulersche Produktdarstellung]{Eulersche Produktdarstellung}.
(Beachte $N^z = e^{z\Log N}$ für $z\in\CC$, $N \in \NN$.)
\item Es gilt $\Gamma(z+1) = z\Gamma(z)$ für alle $z\in D_{-\NN_0}$ und $\Gamma(n) = (n-1)!$ für alle $n\in\NN$.
\end{enumerate}
\end{satz-noind}

\begin{bewe-noind}
Zeige zunächst (i) und (ii):
\begin{enumerate}
\item[(i, ii)] Für $N\in\NN$ setze
\[
	\Gamma_N(z)
	:= \frac{N^z(N-1)!}{z(z+1)\ldots(z+N-1)}
	\,.
\]
Dann gilt
\begin{align*}
	\frac{\Gamma_{N+1}(z)}{\Gamma_N(z)}
	&= \frac{(N+1)^zN!}{z(z+1)\ldots(z+N)} \cdot \frac{z(z+1)\ldots(z+N-1)}{N^z(N-1)!} \\
	&= \frac{(N+1)^z}{N^z}\cdot \frac{N}{z+N}
	= \left(1+\frac{1}{N}\right)^z\cdot \frac{1}{1+\frac{z}{N}}
	\,.
\end{align*}

Für $z\in\CC$ fest sei $(1+w)^z = e^{z\Log(1+w)}$ für $\abs w < 1$.
Dies ist eine holomorphe Funktion und hat um $w=0$ die Taylorentwicklung für $\abs w < 1$
\[
	(1+w)^z
	= 1 + \sum_{n=1}^\infty \binom{z}{n} w^n
	\,.
\]
wobei $\binom{z}{n} = \frac{z(z+1)\cdot\,\ldots\,\cdot (z-n+1)}{n!}$ der \myemph{verallgemeinerte Binominalkoeffizient} ist (Beweis diese Aussage durch Induktion).
Schreibe nun
\[
	(1+w)^z
	= 1 + zw + w^2A(z, w)
\]
wobei
\[
	A(z, w)
	= \sum_{n=2}^\infty \binom{z}{n} w^{n-2}
	\qquad \abs w < 1
\]

\emph{Behauptung} $A(z, w)$ ist auf Mengen der Form
\[
	\Set{z\in \CC \mid \abs z \leq c} \times \Set{w\in\CC \mid \abs w \leq \alpha}
	\qquad 0< c, 0 < \alpha < 1
\]
beschränkt, denn
\begin{align*}
	\abs{A(z, w)}
	&\leq \sum_{n=2}^\infty \abs{\binom{z}{n}} \cdot \abs{w}^{n-2} \\
	&\leq \sum_{n=2}^\infty \underbrace{\frac{c(c+1)\ldots(c+n-1)}{n!}}_{=(-1)^n\cdot\binom{-c}{n}}\alpha^{n-2} \\
	&= \sum_{n=2}^\infty \binom{-c}{n}(-\alpha)^{n-2} \\
	&= A(-c, -\alpha)
	< \infty
	\,.
\end{align*}

Sei $K\subseteq D_{-\NN_0}$ kompakt.
Dann gilt für alle $z\in K$ und $N \in \NN$ (mit $N$ groß genug) unter Benutzung der geometrischen Reihe und mit $w=\frac{1}{N}$, dass
\begin{align*}
	&\quad\left(1+\frac{1}{N}\right)^z \cdot \frac{1}{1+\frac{z}{N}} \\
	&= \left(1 + \frac{z}{N} + \frac{1}{N^2}A\left(z, \frac{1}{N}\right)\right) \left(1-\frac{z}{N}+\left(\frac{z}{N}\right)^2B\left(\frac{z}{n}\right)\right)\\[1.3em] %extra space for single long formular
	\begin{split}
		&= 1 - \frac{z^2}{N^2} + \left(1+\frac{z}{N}\right)\left(\frac{z}{N}\right)^2 B\left(\frac{z}{n}\right)+\frac{1}{N^2}A\left(z, \frac{1}{N}\right)\left(1-\frac{z}{N}\right) \\
		&\qquad+ \frac{1}{N^2}A\left(z, \frac{1}{N}\right) \cdot \left(\frac{z}{N}\right)^2\cdot B\left(\frac{z}{n}\right)
	\end{split}\\
\end{align*}
Also $\abs{\left(1+\frac{1}{N}\right)^z \cdot \frac{1}{1+\frac{z}{N}} - 1} \leq \frac{C}{N^2}$ für ein $C > 0$.
Wegen $\sum_{N=1}^\infty \frac{1}{N^2} < \infty$ und \autoref{satz:prod_holomorpher_fkt} folgt, dass
\[
	\prod_{N=1}^\infty \frac{\Gamma_{N+1}(z)}{\Gamma_N(z)}
\]
unbedingt konvergent und eine holomorphe Funktion definiert auf $D_{-\NN_0}$.
Alle Faktoren sind ungleich Null, also nach Definition
\begin{align*}
	\prod _{N=1}^\infty \frac{\Gamma_{N+1}(z)}{\Gamma_N(z)}
	&= \lim _{M \to \infty} \prod_{N=1}^M \frac{\Gamma_{N+1}(z)}{\Gamma_N(z)} \\
	&= \lim _{M \to \infty} \frac{\Gamma_{2}(z)}{\Gamma_1(z)}\cdot \frac{\Gamma_{3}(z)}{\Gamma_2(z)} \cdot\,\ldots\,\cdot \frac{\Gamma_{M+1}(z)}{\Gamma_M(z)} \\
	&= \lim _{M\to\infty} \frac{\Gamma_{M+1}(z)}{\Gamma_1(z)}
	= z \lim _{M\to\infty} \Gamma_M(z)
	\,,
\end{align*}
dies zeigt die Behauptungen (i) und (ii).

\item[(iii)] Es gilt für alle $z+1\in D_{-\NN_0}$
\begin{align*}
	\Gamma(z+1)
	&= \lim_{N\to\infty} \frac{N^{z+1}(N-1)!}{(z+1)(z+2)\ldots(z+N)} \\
	&= \lim_{N\to\infty} \frac{N}{z+N}\cdot\frac{zN^z(N-1)!}{z(z+1)(z+2)\ldots(z+N-1)}
	= z\Gamma(z)
	\,.
\end{align*}
Es gilt weiter
\[
	\Gamma(1)
	= \lim_{N\to\infty} \frac{N(N-1)!}{N!}
	= 1
	\,.
\]
Induktiv folgt damit $\Gamma(n+1) = n!$ für alle $n\in\NN_0$.
\end{enumerate}
\end{bewe-noind}

\begin{satz-noind}\label{satz:gamma-eigenschaften}
Für die $\Gamma$-Funktion gelten folgende Eigenschaften:
\begin{enumerate}
\item Die Gammafunktion lässt sich in ganz $\CC$ meromorph fortsetzen mit einfachen Polstellen in $-\NN_0$ und holomorph in $D_{-\NN_0}$. Und es gilt
\[
	\res_{z=-n} \Gamma(z)
	= \frac{(-1)^n}{n!}
	\qquad \text{für alle } n\in\NN_0
\]

\item die Funktion $z\mapsto \frac{1}{\Gamma(z)}$ ist ganz mit hebbaren Singularitäten in $z\in-\NN_0$. Es gilt die \myemph[Gamma-Funktion!Weierstraß-Produktentwicklung]{Weierstraß-Produktentwicklung}
\[
	\frac{1}{\Gamma(z)} = ze^{\gamma z} \prod_{n=1}^\infty \left(1+\frac{z}{n}\right)e^{-\frac{z}{n}}
\]
mit $\gamma = \lim_{n\to\infty} 1 + \frac{1}{2} + \frac{1}{3} + \ldots + \frac{1}{n} - \log(n) = 0,57721\dots$ die \myemph{Euler-Mascheroni-Konstante}.

\item Es gilt für alle $z\in\CC\setminus\ZZ$ der \myemph{Ergänzungssatz}:
\[
	\Gamma(z)\Gamma(1-z)
	= \frac{\pi}{\sin(\pi z)}
\]
\end{enumerate}
\end{satz-noind}

\begin{bewe-list}
\item Es gilt mit \autoref{satz:gamma-fkt} (iii) für $N\in\NN_0$:
\[
	\Gamma(z+N+1)
	= (z+N)\ldots(z+1)z\Gamma(z)
	\,.
\]
Damit erhält man
\[
	\Gamma(z)
	= \frac{\Gamma(z+N+1)}{z(z+1) \ldots (z+N)}
	\,.
\]
Also
\[
	\Gamma(z)
	= \frac{g(z)}{z-(-N)}
	\,,
\]
wobei $g$ lokal holomorph und $g(-N) \not=0$.

Also besitzt $\Gamma$ in $z=-N$ einen Pol der Ordnung 1 mit Residuen
\begin{align*}
	\res_{z=-N} \Gamma(z)
	&= \lim _{z\to-N} (z+N)\Gamma(z)
	= g(-N) \\
	&= \frac{\Gamma(1)}{(-N)(-N+1)(-N+2)\ldots(-1)} \\
	&= \frac{(-1)^N}{N!}
	\,.
\end{align*}

\item Sei $a_n = 1 + \frac{1}{2} + \ldots + \frac{1}{n} - \log(n)$.

Zeige zunächst $a_n$ ist eine konvergente Folge.
Zunächst gilt
\[
	a_n - a_{n+1}
	= -\frac{1}{n+1} + \log\left(1+\frac{1}{n}\right)
	\geq 0
	\,.
\]
denn
\begin{align*}
	\log\left(1+\frac{1}{n}\right)
	&= \int_1^{1+\frac{1}{n}} \frac{1}{t} \opd t
	\geq \int_1^{1+\frac{1}{n}} \frac{n}{n+1} \opd t \\
	&= \frac{1}{n}\frac{n}{n+1}
	= \frac{1}{n+1}
	\,.
\end{align*}
Also ist die Folge $a_n$ monoton fallend.
Weiter gilt
\begin{align*}
	\sum_{\nu=1}^n \frac{1}{\nu}
	&= \sum_{\nu=1}^n \int_\nu^{\nu+1} \frac{1}{\nu}\opd t
	\geq \sum_{\nu=1}^n \int_\nu^{\nu+1} \frac{1}{t}\opd t \\
	&= \int_1^{n+1}\frac{1}{t}\opd t
	= \log(n+1)
	\geq \log(n)
\end{align*}

Also ist $a_n \geq 0$ für alle $n\in\NN$.
Aus dem Satz über die monotone Folge, folgt, dass die $a_n$ konvergieren.
Den Grenzwert bezeichnen wir mit $\gamma$.

Nun folgt
\[
	\frac{N^z}{e^{z(1+\frac{1}{2} + \frac{1}{3} + \ldots + \frac{1}{N})}}
	= e^{z(Log(N)- 1 - \frac{1}{2} - \frac{1}{3} - \ldots - \frac{1}{N})}
	\xto{N\to\infty} e^{-\gamma z}
	\,,
\]
also
\begin{align*}
	z\Gamma(z)
	&= \Gamma(z+1)
	= \lim_{N\to\infty} \frac{N^{z+1} (N-1)!}{(z+1)(z+2)\ldots(z+N)} \\
	&= \lim_{N\to\infty} \frac{N^z}{e^{z(1+\frac{1}{2} + \frac{1}{3} + \ldots + \frac{1}{N})}} \frac{e^{z(1+\frac{1}{2} + \frac{1}{3} + \ldots + \frac{1}{N})}N!}{(z+1)\ldots(z+N)} \\
	&= e^{-\gamma z} \lim_{N\to\infty} \frac{e^{z(1+\frac{1}{2} + \frac{1}{3} + \ldots + \frac{1}{N})}N!}{(z+1)\ldots(z+N)} \\
	&= e^{-\gamma z} \prod_{n=1}^\infty \left(1+\frac{z}{n}\right)^{-1}e^{\frac{z}{n}}
	\,.
\end{align*}
Damit erhalten wir
\[
	\frac{1}{\Gamma(z)}
	= ze^{\gamma z} \prod_{n=1}^\infty \left(1+\frac{z}{n}\right)e^{-\frac{z}{n}}
	\,.
\]
Insbesondere gilt diese Schreibweise auch für $z\in-\NN_0$.

\item Aus $z\Gamma(z) = \Gamma(z+1)$ für $z\in D_{-\NN_0}$ folgt
\[
	\Gamma(1-z)\Gamma(z)
	= -z \Gamma(-z)\Gamma(z)
	\,,
\]
also für $z\not\in \ZZ$ (\autoref{bsp:nullstellenverteilungen})
\begin{align*}
	\frac{1}{\Gamma(1-z)\Gamma(z)}
	&= -\frac{1}{z\Gamma(-z)\Gamma(z)} \\
	&= -\frac{1}{z}(-z)ze^{-\gamma z}e^{\gamma z}\prod_{n=1}^\infty \left(1-\frac{z^2}{n^2}\right) \\
	&= \frac{1}{\pi} \sin(\pi z)
	\,.
\end{align*}
\end{bewe-list}

\begin{satz-list}[Charakterisierung nach Wielandt]\label{satz:wielandt}
\item Die Gammafunktion ist auf jedem Vertikalstreifen
\[
	\Set{z=x+iy \in\CC \mid a \leq x \leq b}
\]
mit $0 < a \leq b$ beschränkt.

\item Es sei $D\subseteq D_{-\NN_0}$ ein Gebiet, welches den Vertikalstreifen $1 \leq x \leq 2$ enthält.
Sei $f\colon D\ra \CC$ eine holomorphe Funktion mit den Eigenschaften
\begin{enumerate}[(a)]
\item $zf(z) = f(z+1)$ für alle $z, z+1\in D$,
\item $f$ ist auf dem Streifen $1\leq x \leq 2$ beschränkt.
\end{enumerate}
Dann gilt:
$f(z) = f(1) \Gamma(z)$ für $z\in D$.
\end{satz-list}

\begin{beme}
Die Voraussetzung $D$ Gebiet darf nicht weggelassen werden.
Denn sei $D := \Set{z = x+iy\in\CC \mid \frac{1}{2} < x < 3} \dot\cup\ U_1(10)$ und
\[
	f(z) :=
	\begin{cases}
		\Gamma(z) &\text{wenn } \frac{1}{2} < x < 3 \text{ für } z=x+iy \\
		0 &\text{sonst}
	\end{cases}
	\,.
\]
Dann erfüllt $f$ offensichtlich die Bedingungen (a) und (b), aber es gilt nicht $f(z) = \Gamma(z)$ für alle $z\in D$.
\end{beme}

\begin{bewe-list}
\item Für beliebiges $z=x+iy\in\CC$ mit $0 < a \leq \Re(z) \leq b$ gilt wegen
\[
	\abs{N^z}
	= \abs{e^{\log(N)(x+iy)}}
	= N^x
\]
und $\abs{w} \geq \Re(w)$ für alle $w\in\CC$:
\begin{align*}
	\abs{\Gamma(z)}
	&= \lim_{N\to\infty} \abs{\frac{N^z(N-1)!}{z(z+1)(z+2)\ldots(z+N-1)}} \\
	&\leq \lim_{N\to\infty} \frac{N^x(N-1)!}{x(x+1)\ldots(x+N-1)}
	= \Gamma(x)
\end{align*}
und, da $\Gamma$ stetig ist und somit auf dem Kompaktum [a,b] beschränkt ist, ist $\Gamma(x)$ beschränkt.

\item Durch die Vorschrift
\[
	f(z+1)
	= zf(z)
\]
wird $f$ holomorph auf $1+D$ fortgesetzt, man erhält also eine holomorphe Funktion auf $D\cup (D+1)$
Letzteres ist ein Gebiet, da $D$ und $1+D$ Gebiete und $D \cap (1+D)$ enthält die Gerade $\Set{z=x+iy\in\CC \mid x=2}$.

Nach dem Identitätssatz ist die holomorphe Fortsetzung $\widetilde{f}$ von $f$ auf $D\cup (1+D)$ eindeutig bestimmt.
Verfährt man nach dem selben Prinzip weiter, so erhält man eine eindeutig bestimmte holomorphe Fortsetzung $\widetilde{f}$ von $f$ auf dem Gebiet
\[
	D \cup (1+D) \cup (2+D) \cup (3+D) \cup ...
\]

Ebenso wird $f$ durch die Vorschrift $f(z+1) = zf(z)$ sukzessive auf
\[
	D \cup (-1+D) \cup (-2+D) \cup (-3+D) \cup ...
\]
meromorph fortgesetzt, mit Polstellen höchstens in den Punkten $-\NN_0$.
Insgesamt erhält man also eine meromorphe Fortsetzung von $f$ auf ganz $\CC$ mit Polen in $-\NN_0$.
Wegen des Identitätssatzes gilt global die Funktionalgleichung
\[
	zf(z) = f(z+1)
\]
und analog zu \autoref{satz:gamma-eigenschaften} (i) erhalten wir eine Laurent-Entwicklung von $f$ in jedem Punkt $z = -N \in -\NN_0$ der Form
\[
	f(z)
	= \frac{(-1)^N}{N!}f(1)\frac{1}{N+z} + \ldots
	\,.
\]
Sei
\[
	h(z)
	= f(z) - f(1)\Gamma(z)
	\qquad \text{für } z \in \CCminusNN
	\,.
\]

Nach \autoref{satz:gamma-eigenschaften} (i) definiert $h$ dann eine ganze Funktion auf $\CC$ und es gilt
\[
	h(z+1)
	= zh(z)
\]
aus Stetigkeitsgründen in ganz $\CC$.

Wegen (i) und (b) ist $h(z)$ beschränkt auf dem Streifen $1 \leq x \leq 2$.
Es ist $h(z) = \frac{h(z+2)}{z(z+1)}$ für $z\in\CCminusNN$.
Sei $-1 \leq x \leq 0$.
Dann ist $1 \leq x+2 \leq 2$, also folgt aus der Beschränktheit von $h(z)$ in dem Streifen $1\leq x \leq 2$ und wegen $\abs z \geq \abs y$ die Beschränktheit von $h(z)$ in $\Set{z=x+iy\in\CC \mid -1 \leq x \leq 0, 1\leq\abs y}$.

Da $h$ stetig ist und $-1 \leq x \leq 0, \abs y \leq 1$ kompakt ist, ist also $h(z)$ auf $-1\leq x \leq 0$ beschränkt.
Sei
\[
	H(z) = h(z)h(1-z)
	\qquad \text{für } z\in\CC
	\,.
\]
Dann ist $H$ in $1 \leq x \leq 2$ beschränkt.
Außerdem ist für alle $z\in\CC^\times$
\[
	H(z+1)
	= h(z+1)h(-z)
	= \frac{zh(z)h(1-z)}{-z}
	= -H(z)
	\,.
\]
Also gilt $\abs{H(z+1)} = \abs{H(z)}$ aus Stetigkeitsgründen für alle $z\in\CC$.
Daher ist $H$ auf $\CC$ beschränkt, nach dem Satz von Liouville folgt damit
\[
	H(z)
	= H(1)
	= h(1)h(0)
	= 0
	\qquad z\in\CC
	\,,
\]
da $h(1) = f(1) - f(1)\Gamma(1) = 0$.

Hieraus folgt $h \equiv 0$, also die Behauptung.
\emph{Denn} wäre $h(z_0) \not= 0$, so gäbe es aus Gründen der Stetigkeit eine Umgebung $U_\delta(z_0)$ mit $h(w) \not=0$ für alle $w\in U_\delta(z_0)$.
Damit folgt $h|_{U_\delta(1-z_0)} \equiv 0$, also da $\CC$ ein Gebiet ist mit dem Identitätssatz $h\equiv 0$.
\end{bewe-list}

\begin{satz}[Legendresche Duplikationsformel]\label{satz:legendre_dupli}
Es gilt
\[
	\Gamma\left(\frac{z}{2}\right) \Gamma\left(\frac{z+1}{2}\right)
	= \sqrt{\pi} \cdot 2^{1-z} \cdot \Gamma(z)
	\qquad \text{für } z\in \CCminusNN
\]
\end{satz}

\begin{bewe}
Wir wollen den Satz von Wielandt (\autoref{satz:wielandt}) benutzen mit
\[
	f(z)
	:= \frac{2^{z-1}\Gamma\left(\frac{z}{2}\right)\Gamma\left(\frac{z+1}{2}\right)}{\sqrt{\pi}}
	\,.
\]
Dann gilt:
$\CCminusNN$ enthält den Streifen $1\leq x \leq 2$ und $f$ ist dort holomorph.
Es gilt
\[
	f(z+1)
	= \frac{2^{z}\Gamma\left(\frac{z+1}{2}\right)\Gamma\left(1+\frac{z}{2}\right)}{\sqrt{\pi}}
	= \frac{z2^{z-1}\Gamma(\frac{z}{2})\Gamma(\frac{z+1}{2})}{\sqrt{\pi}}
	= zf(z)
	\,.
\]

Nach \autoref{satz:wielandt} (i) ist $f$ beschränkt auf $1 \leq x \leq 2$.
Somit gilt nach dem Satz von Wielandt (\autoref{satz:wielandt})
\[
	f(z)
	= f(1)\Gamma(z)
	\,.
\]
Es gilt weiterhin
\[
	f(1)
	= \frac{\Gamma(\frac{1}{2})\Gamma(1)}{\sqrt{\pi}}
\]
und $\Gamma(\frac{1}{2})^2 = \frac{\pi}{\sin(\frac{\pi}{2})} = \pi$ nach dem Ergänzungssatz (\autoref{satz:gamma-eigenschaften}).

Wegen $\Gamma(\frac{1}{2}) > 0$ (folgt aus der Produktformel von Euler (\autoref{satz:gamma-fkt}) gilt somit $\Gamma(\frac{1}{2}) = \sqrt{\pi}$ und damit $f(1) = 1$.
\end{bewe}

\begin{lemm}\label{lemma:majoranten-int}
Sei $f\colon (0,1] \ra \RR$ stetig. $f(t) \geq 0$ für alle $t \in\RR$.
Ist dann
\[
	\int_A^1 f(t) \opd t
	\leq C
	\qquad \text{für } 0 < A < 1
	\,,
\]
so existiert
\[
	\lim_{A\to 0} \int_A^1 f(t) \opd t
	\,.
\]
\end{lemm}
\begin{bewe}
Wir müssen zeigen, es gibt $a \in \RR$, so dass für jede Folge $(a_n)_{n\in\NN}$ mit $\lim_{n\to\infty} a_n = 0$ und $a_n > 0$ gilt
\[
	\lim_{n\to\infty} \int_{a_n}^1 f(t) \opd t
	= a
	\,.
\]
Sei zunächst $a_n = \frac{1}{n}$. Die Folge
\[
	\left(\int_{\frac{1}{n}}^1 f(t) \opd t\right)_{n\in\NN}
\]
ist monoton steigend und nach oben beschränkt, also existiert
\[
	a = \lim_{n\to\infty} \int_{\frac{1}{n}}^1 f(t) \opd t
\]
als die kleinste obere Schranke.

Sei jetzt $(a_n)_{n\in\NN}$ beliebig und $a_n \xto{n\to\infty} 0$ mit $a_n > 0$ für alle $n\in\NN$.
Sei $\epsilon > 0$.
Dann existiert $N_0 \in \NN$ mit
\[
	a - \epsilon
	< \int_{\frac{1}{N_0}}^1 f(t) \opd t
	\,.
\]
Wegen $\lim_{n\to\infty} a_n = 0$ gibt es $N\in\NN$ mit $a_n \leq \frac{1}{N_0}$ für alle $n\in\NN$ mit $N < n$.
Es folgt
\[
	a - \epsilon
	< \int_{\frac{1}{N_0}}^1 f(t)\opd t
	\leq \int_{a_n}^1 f(t)\opd t
	\leq \int^1_{\frac{1}{N_n}} f(t)\opd t
	\leq a
	< a+\epsilon
	\,.
\]

Damit folgt
\[
	\lim_{n\to\infty} \int_{a_n}^1 f(t) \opd t
	= a
	\,.
\]
\end{bewe}

\begin{satz-list}[Eulersches Integral]\index{Gamma-Funktion!Eulersches Integral}
\item Das uneigentliche Integral
\[
	\int_0^\infty t^{z-1}e^{-t} \opd t
\]
mit $t^{z-1} = e^{(z-1)\log(t)}$ ist für $\Re(z) > 0$ absolut konvergent, d.\,h.
\begin{align*}
	\lim_{A\to 0} &\int_A^1 t^{\Re(z)-1}e^{-t} \opd t
	\qquad \text{und} \\
	\lim_{B\to\infty} &\int_1^B t^{\Re(z)-1}e^{-t} \opd t
\end{align*}
existieren.

\item Für $\Re(z) > 0$ ist
\[
	\Gamma(z) = \int_0^\infty t^{z-1}e^{-t} \opd t
\]
\end{satz-list}

\begin{bewe-list}
\item Sei $z = x+iy \in\CC$ fest mit $x = \Re(z) > 0$.
Es ist
\[
	\abs{t^{z-1}e^{-t}}
	= t^{x-1}e^{-t}
	\leq t^{x-1}
	\qquad \text{für } t>0
	\,.
\]

Daraus folgt, dass
\[
	\int_A^1 \abs{t^{z-1}e^{-t}}\opd t
	\leq \int_A^1 t^{x-1} \opd t
	= \at{\frac{1}{x}t^x}_A^1
	= \frac{1}{x} (1-A^x)
	\mystackrel{$\scriptstyle x > 0$}{\leq} \frac{1}{x}
\]
Nach dem Majorantenkriterium für Integrale folgt also die Existenz von
\[
	\lim_{A\to 0} \int_A^1 t^{\Re(z)-1}e^{-t} \opd t
	\qquad \text{für } \Re(z) > 0
	\,.
\]

Es gilt weiterhin
\[
	t ^{x-1} \leq C\cdot e^{\frac{t}{2}}
	\qquad \text{für alle } t \geq 1
\]
mit geeignetem $C = C(x) > 0$.
Daher folgt
\[
	\int_1^B t^{x-1} \opd t
	\leq C(x) \int_1^B e^{-\frac{t}{2}} \opd t
	= 2C(x) (e^{-\frac{1}{2}} -e^{-\frac{B}{2}})
	\leq 2C(x)e^{\frac{1}{2}}
	\,.
\]
Es folgt, dass
\[
	\lim_{B\to\infty} \int_1^B t^{\Re(z)-1}e^{-t} \opd t
\]
existiert.



\item Wir wollen wieder den Satz von Wielandt (\autoref{satz:wielandt}) benutzen mit
\[
	f(z) = \int_0^\infty t^{z-1}e^{-t} \opd t
\]
Für $\Re(z) > 0$ ist (Partielle Integration!)
\[
	f(z+1)
	= \int_0^\infty t^ze^{-t} \opd t
	= -e^{-t}t^z\bigg|_0^\infty - \int_0^\infty zt^{z-1}(-e^{-t}) \opd t
	= zf(z)
	\,.
\]
Das geht, wegen \autoref{lemma:majoranten-int}. Nach dem Leibniz-Kriterium ist das betrachte Integral holomorph, daher ist partielle Integration gerechtfertigt.

Schließlich ist für $1 \leq x \leq 2$
\begin{align*}
	\int_0^\infty \abs{t^{z-1}e^{-t}}
	&\leq \int_0^\infty t^{x-1}e^{-t} \opd t \\
	&= \int_0^1 t^{x-1}e^{-t} \opd t + \int_1^\infty t^{x-1}e^{-t} \opd t \\
	&\leq \int_0^1 e^{-t} \opd t + \int_1^\infty te^{-t} \opd t
	= 1 + C
	\qquad \text{für ein } C > 0
	\,.
\end{align*}
Also ist $f(z)$ auf $1 \leq x \leq 2$ beschränkt.

Schließlich muss man noch zeigen, dass $f(z)$ für $\Re(z) > 0$ holomorph ist.
Es ist leicht zu sehen, dass
\[
	f_n(z) = \int_{\frac{1}{n}}^n t^{z-1}e^{-t} \opd t
\]
holomorph ist für alle $n\in\NN$, da $\left[\frac{1}{n}, n\right]$ ein kompaktes Intervall ist (Leibniz-Regel).
Unter Benutzung ähnlicher Argumente wie in (i) zeigt man, dass $(f_n)_{n\in\NN}$ auf Kompakta  $K\subseteq \Set{Re(z) \geq 0}$ gleichmäßig konvergiert.
Damit folgt die Behauptung mit dem Satz von Weierstraß und die Aussage des Satzes folgt mit Wielandt (\autoref{satz:wielandt}), da
\[
	f(1)
	= \int_0^\infty e^{-t} \opd t = 1
\]
klar ist.
\end{bewe-list}

\emph{Problem:} Wie verhält sich $\Gamma(z)$ für sehr große Werte $\abs{z}$? Kann man $\Gamma(z)$ dort durch eine \myquote{einfache Funktion} gut approximieren?

\begin{satz-list}[Stirlingsche Formel]\label{satz:stirling}
\item Sei $\CC_- = \Set{z\in\CC \mid z \not= x \text{ für } x\in\RR\colon x \leq 0}$ die geschlitzte Ebene ohne die negative reelle Achse und $0$.
Für $z\in\CC_-$ gilt dann die Darstellung\footnote{Insbesondere gilt für natürliche Zahlen $n\in\NN$ \[(n-1)! \approx \sqrt{2\pi} \cdot n^{n-\frac{1}{2}} \cdot e^{-n}\]}
\[
	\Gamma(z)
	= \sqrt{2\pi} \cdot z^{z-\frac{1}{2}} \cdot e^{-z} \cdot e^{H(z)}
\]
mit einer Funktion $H(z)$, welche in jedem Winkelbereich
\[
	W_\delta
	= \Set{z\in\CC_- \mid -\pi + \delta < \Arg(z) < \pi - \delta}
	\qquad \text{mit } \delta > 0
\]
für $\abs z \to \infty$ gegen $0$ konvergiert.
\item Für $x>$ ist
\[
	\Gamma(x)
	= \sqrt{2\pi} \cdot x^{x-\frac{1}{2}} \cdot e^{x-\frac{1}{2}} \cdot e^{\frac{\theta(x)}{12x}}
\]
mit $0 < \theta(x) < 1$.
\end{satz-list}

Für den Beweis zeigen wir zunächst einige Lemma.

\begin{lemm}\label{lemma:stirling:hl1}
Für $z\in\CC_-$ sei
\[
	H_0(z) := \left(z+\frac{1}{2}\right)\left(\Log(z+1)-\Log(z)\right) - 1
\]
Sei $A=\CC_- \cap \Set{z\in\CC \mid \abs{z+\frac{1}{2}} > \frac{1}{2}}$.
Für $z\in A$ gilt dann
\begin{equation}\label{eqn:h0_reihe}
	H_0(z)
	= \sum_{\substack{\nu \geq 2 \\ \scriptscriptstyle \nu \text{ gerade}}} \frac{1}{\nu+1} \left(\frac{1}{2z+1}\right)^\nu
\end{equation}
\end{lemm}

\begin{bewe}
Sei $K \subseteq A$ kompakt.
Dann existiert $c>1$, so dass $\abs{z+\frac{1}{2}} \geq \frac{c}{2}$ für alle $z\in K$, also konvergiert die Reihe \eqref{eqn:h0_reihe} auf Kompakta absolut und gleichmäßig und stellt somit eine in $A$ holomorphe Funktion da.
Für $x\in\RR$, $x>0$ gilt
\begin{align*}
	\log(x+1) - \log(x)
	&= \log\left(\frac{x+1}{x}\right)
	= \log\left(\frac{1+\frac{1}{2x+1}}{1-\frac{1}{2x+1}}\right) \\
	&= \log\left(1+\frac{1}{2x+1}\right) - \log\left(1-\frac{1}{2x+1}\right)\\
	&= \sum_{\nu \geq 1} \frac{(-1)^{\nu+1}}{\nu} \left(\frac{1}{2x+1}\right)^\nu + \sum_{\nu \geq 1} \frac{1}{\nu}\left(\frac{1}{2x+1}\right)^\nu
	\,,
\end{align*}
wobei sich die letzte Gleichheit aus $0 < \frac{1}{2x+1} < 1$ und
\[
	\log(1+\delta) = \sum_{n=1}^\infty \frac{(-1)^{n-1}}{n}\delta^n
	\qquad \abs \delta < 1
\]
ergibt.
Damit erhalten wir
\[
	\log(x+1) - \log(x)
	= 2\cdot\mkern-15mu\sum_{\substack{\nu \geq 1\\ \scriptscriptstyle \nu \text{ ungerade}}} \mkern-10mu \frac{1}{\nu} \left(\frac{1}{2x+1}\right)^\nu.
\]
Für $x > 0$ folgt also
\begin{align*}
	H_0(x)
	&= \left(x+\frac{1}{2}\right) (\log(x+1)-\log(x)) - 1 \\
	&= \left(\sum_{\substack{\nu \geq 1\\ \scriptscriptstyle \nu \text{ ungerade}}} \frac{1}{\nu} \left(\frac{1}{2x+1}\right)^{\nu-1}\right) -1 \\
	&= \sum_{\substack{\nu \geq 2\\ \scriptscriptstyle \nu \text{ ungerade}}} \frac{1}{\nu} \left(\frac{1}{2x+1}\right)^{\nu - 1}
\end{align*}
also gilt \eqref{eqn:h0_reihe} für $z\in A\cap \RR_{\geq 0}$, beide Seiten sind in $A$ holomorphe Funktionen, also gilt \eqref{eqn:h0_reihe} nach dem Identitätssatz.
\end{bewe}

\begin{lemm}\label{lemma:stirling:hl2}
Für $z\in\CC_-$ mit $\abs{z+\frac{1}{2}} > 1$ ist
\[
	\abs{H_0(z)}
	\leq \frac{1}{2} \frac{1}{\abs{2z+1}^2}
	\,.
\]
\end{lemm}

\begin{bewe}
Für $\abs{z+\frac{1}{2}} > 1$ ist $\frac{1}{\abs{2z+1}} < \frac{1}{2}$.
Mit $w=\frac{1}{2z+1}$ ist für $\abs w < \frac{1}{2}$ dann nach \autoref{lemma:stirling:hl1}
\begin{align*}
	\abs{H_0(z)}
	&\leq \frac{\abs w ^2}{3} + \frac{\abs w ^4}{5} + \frac{\abs w ^6}{7} + \ldots \\
	&\leq \frac{\abs w ^2}{3} (1 + \abs{w}^2 + \abs{w}^4 + \ldots) \\
	&\leq \frac{\abs w ^2}{3} \left(1 +\left(\frac{1}{2}\right)^2+\left(\frac{1}{2}\right)^4+ \ldots \right) \\
	&= \frac{\abs{w}^2}{3} \cdot \frac{1}{1-(\frac{1}{2})^2}
	= \frac{4}{3} \cdot \frac{\abs{w}^2}{3}
	\leq \frac{1}{2} \abs{w}^2
	\,.
\end{align*}
\end{bewe}

\begin{lemm-noind}\label{lemma:stirling:hl3}
Sei
\[
	H(z)
	= \sum_{k=0}^\infty H_0(z+k)
	\qquad z\in\CC_-
	\,.
\]
\begin{enumerate}
\item Die Reihe $H(z)$ konvergiert auf jeder Teilmenge $\CC_-$ absolut und gleichmäßig und ist damit eine holomorphe Funktion in $\CC_-$.

\item Es gilt
\begin{equation}\label{eqn:h_lim}
	\lim_{\substack{\abs{z}\to\infty\\ \scriptscriptstyle z\in W_\delta}} H(z) = 0
	\,.
\end{equation}
\end{enumerate}
\end{lemm-noind}

\begin{bewe-list}
\item Sei $K\subseteq \CC_-$ kompakt.
Für $n\in\NN_0$, $n$ groß, ist dann
\[
	\abs{(z+n) + \frac{1}{2}}
	= \abs{\left(n+\frac{1}{2}\right) - (-z)}
	\geq n+\frac{1}{2} - \abs{z}
	> 1
	\qquad z\in K
	\,.
\]
Also folgt für solche $n$ für alle $z\in K$ nach \autoref{lemma:stirling:hl2}
\[
	\abs{H_0(z+n)}
	\leq \frac{1}{2} \frac{1}{\abs{2(z+n)+1}^2}
	\leq \frac{1}{2}\cdot \frac{1}{n^2}
	\,,
\]
falls $n$ zusätzlich noch so groß ist, dass $2(x+n) + 1 \geq n$, d.\,h. $x \geq \frac{-(n+1)}{2}$.

Weil $\sum_{n=0}^\infty \frac{1}{n^2}$ absolut konvergiert, also $H(z)$ auf $K$ gleichmäßig absolut konvergiert, ist $H(z)$ auf $\CC_-$ holomorph.



\item Es genügt \eqref{eqn:h_lim} für kleines $\delta$, z.\,B. $\delta < \frac{\pi}{2}$, zu zeigen.
Dann ist
\[
	W_\delta
	= \Set{z = x+iy\in\CC \mid x > 0 \text{ oder } \abs{y} > C\abs{x}}
\]
mit geeignetem $C = C(\delta) > 0$.

Offenbar gibt es $N(\delta) \in \NN$ so dass für alle $n \geq N(\delta)$ und $z\in W_\delta$ gilt
\[
	z+n \in \CC_-
	\text{ und }
	\abs{(z+n)+\frac{1}{2}} > 1
	\,.
\]
Also folgt nach \autoref{lemma:stirling:hl2}
\begin{align*}
	\abs{H_0(z+n)}
	&\leq \frac{1}{2} \frac{1}{\abs{2(z+n)+1}^2} \\
	&= \frac{1}{2} \frac{1}{(2x+2n+1)^2+4y^2}
	\qquad \forall n \geq N(\delta), z\in W_\delta
\end{align*}

Für $x > 0$ ist
\[
	(2x+2n+1)^2+4y^2\geq (2n+1)^2
	\,.
\]

Für $\abs{y} > c\abs x$ ist
\begin{align*}
	(2x+2n+1)^2+4y^2
	&\geq (2x+2n+1)^2 + 4C^2x^2 \\
	&\geq (2x_{0,n} + 2n + 1)^2 + 4C^2x_{0,n}^2
	\geq 4C^2x_{0,n}^2
\end{align*}
mit $x_{0,n} = -\frac{2n+1}{2+2C^2}$, denn die Funktion $x\mapsto (2x+2n+1)^2+4C^2x^2$ für $x\in\RR$ nimmt ihr Minimum in $x_{0,n}$ an (Analysis 1).
Es folgt daher
\[
	\abs{H_0(z+n)} \leq C_1\cdot \frac{1}{(2n+1)^2}
	\qquad \forall n\geq N(\delta), z \in W_\delta
\]
mit geeignetem $C_1 = C_1(\delta) > 0$.

Sei $\epsilon > 0$. Da $\sum_{n=0}^\infty \frac{1}{(2n+1)^2} < \infty$, existiert $N(\epsilon) \in \NN$ mit $\sum_{n\geq N(\epsilon)} \frac{1}{(2n+1)^2} \leq \epsilon$.
Sei $N := \max (N(\delta),\,N(\epsilon))$.
Für $z\in W_\delta$ folgt dann
\[
	\sum_{n\geq N} \abs{H_0(z+n)} \leq C'\sum_{n\geq N} \frac{1}{(2n+1)^2}
	\leq C'\epsilon
	\,.
\]
Sei $n < N$. Für $z\in W_\delta$, $\abs z$ groß, ist dann
\[
	\abs{(z+n) + \frac{1}{2}}
	= \abs{z-\left(-n-\frac{1}{2}\right)}
	\geq \abs z - \abs{n+\frac{1}{2}}
	> 1
	\,.
\]
Also gilt nach \autoref{lemma:stirling:hl2} für solche $n$ und $z$
\[
	\abs{H_0(z+n)}
	\leq \frac{1}{2} \frac{1}{\abs{2(z+n)+1}^2}
	\longrightarrow 0
	\qquad \text{für } \abs z \to \infty,\, z \in W_\delta
\]
Insgesamt folgt also
\[
	\lim_{\substack{\abs{z}\to\infty\\ \scriptscriptstyle z\in W_\delta}} H(z) = 0
	\,.
\]
\end{bewe-list}

\begin{lemm-noind}\label{lemma:stirling:hl4}
Sei
\[
	h(z) = z^{z-\frac{1}{2}}e^{-z}e^{H(z)}
	\,.
\]
\begin{enumerate}
\item Dann gilt
\[
	h(z+1)
	= zh(z)
\]
\item $h(z)$ ist beschränkt in $1 \leq x \leq 2$.
\end{enumerate}
\end{lemm-noind}

\begin{bewe-list}
\item Es ist $h(z) = \exp((z-\frac{1}{2})\Log(z) - z + H(z))$, also folgt
\begin{align*}
	\frac{h(z+1)}{h(z)}
	&= \exp\left( \left(z+\frac{1}{2}\right)\Log(z+1) - (z+1)+H(z+1)\right. \\
	&\qquad - \left.\left(\left(z-\frac{1}{2}\right)\Log(z) - z + H(z)\right) \right) \\
	&= \exp((H(z+1) - H(z) + H_0(z) + \Log(z))
	= z
	\,,
\end{align*}
wegen der Teleskopsumme
\begin{align*}
	H(z+1) - H(z)
	&= \sum_{k=0}^\infty H_0(z+k+1) - \sum_{k=0}^\infty H_0(z+k) \\
	&= -H_0(z)
	\,.
\end{align*}




\item Die Funktion $e^{-z}$ ist auf $1 \leq x \leq 2$ beschränkt. Im Winkelbereich $W_{\frac{\pi}{2}}$ gilt $H(z) \xto{\abs z \to \infty} 0$ nach \autoref{lemma:stirling:hl3}, insbesondere ist $e^{H(z)}$ in $1 \leq x \leq 2$ beschränkt.
Es ist
\[
	\abs{z^{z-\frac{1}{2}}}
	= \abs{e^{(z-\frac{1}{2})\Log(z)}}
	= e^{\Re\left(\left(z-\frac{1}{2}\right)\Log(z)\right)}
\]
und
\begin{align*}
	\Re\left( \left(z-\frac{1}{2}\right)\Log(z)\right)
	&= \Re\left( \left(x-\frac{1}{2} + iy\right) \left(\log\abs z+ i \Arg(z)\right)\right) \\
	&= \left(x-\frac{1}{2}\right)\log\abs{z} - y \Arg(z) \\
	&\mystackrel{$y\not=0$}{=} y \left(\left(x-\frac{1}{2}\right)\frac{\log\abs z}{y} - \Arg(z)\right)
	\,.
\end{align*}
\end{bewe-list}

\begin{bewe-list}[Stirlingsche Formel \autoref{satz:stirling}]
\item Nach dem Satz von Wielandt (\autoref{satz:wielandt}) folgt mit \autoref{lemma:stirling:hl4}
\[
	\Gamma(z) = Ah(z)
\]
für ein $A\in\CC^\times$.
Nach der Legendreschen Duplikationsformel (\autoref{satz:legendre_dupli}) ist für ein $n\in\NN$:
\begin{align*}
	\sqrt{\pi}
	&= 2^{n-1} \frac{\Gamma(\frac{n}{2}) \Gamma(\frac{n+1}{2})}{\Gamma(n)}
	= 2^{n-1} \frac{h(\frac{n}{2}) h(\frac{n+1}{2})}{h(n)} \cdot A \\
	&= A \cdot 2^{n-1}\frac{\left(\frac{n}{2}\right)^{\frac{n}{2}-1} e^{-\frac{n}{2}} e^{H(\frac{n}{2})} \cdot \left(\frac{n+1}{2}\right)^{\frac{n+1}{2}-1} e^{-\frac{n+1}{2}} e^{H(\frac{n+1}{2})}}{n^{n-\frac{1}{2}}e^{-n}e^{H(n)}} \\
	&= A \cdot 2^{-\frac{1}{2}}n^{-\frac{n}{2}}(1+n)^{\frac{n}{2}} e^{-\frac{1}{2}} e^{H(\frac{n}{2})+H(\frac{n+1}{2}) - H(n)} \\
	&= A\cdot (2e)^{-\frac{1}{2}} \left(1+\frac{1}{n}\right)^\frac{n}{2} \cdot e^{H(\frac{n}{2})+H(\frac{n+1}{2}) - H(n)} \\
	&\xto{n\to\infty} \frac{A}{\sqrt{2}}
\end{align*}
Also $A = \sqrt{2\pi}$. Das beweist (i).



\item Nach (i) ist für $x > 0$
\[
	\Gamma(x) = \sqrt{2\pi} \cdot x^{x-\frac{1}{2}} e^{-x} e^{H(x)}
\]
für $x>0$ ist mit $w = \frac{1}{2x+1}$ nach \autoref{lemma:stirling:hl1}
\[
	H_0(x) = \sum_{\substack{\nu\geq 2 \\ \scriptscriptstyle \nu \text{ gerade}}} \frac{1}{\nu+1}w^\nu
	= \frac{1}{3}w^2 + \frac{1}{5}w^4 + \ldots
\]
also folgt
\begin{align*}
	0
	&< H_0(x)
	< \frac{1}{3}w^2 (1+w^2+w^4+\ldots) \\
	&= \frac{1}{3}w^2 \frac{1}{1-w^2}
	= \frac{1}{3} \cdot \frac{1}{(2x+1)^2-1} \\
	&= \frac{1}{12(x+1)x}
	= \frac{1}{12} \cdot \left( \frac{1}{x} - \frac{1}{x+1} \right)
\end{align*}

Daber gilt für $x > 0$
\begin{align*}
	0
	&< H(x)
	= \sum_{n=0}^\infty H_0(x+n) \\
	&< \frac{1}{12} \sum_{n=0}^\infty \left(\frac{1}{x+n} - \frac{1}{x+1+n}\right) \\
	&= \frac{1}{12} \lim_{N\to\infty} \sum_{k=0}^N \left(\frac{1}{x+n} - \frac{1}{x+1+n}\right) \\
	&= \frac{1}{12}\lim_{N\to\infty} \left(\frac{1}{x}-\frac{1}{x+N+1}\right)
	= \frac{1}{12x}
	\,.
\end{align*}
Mit $\theta(x) = 12x\cdot H(x)$ folgt die Behauptung.
\end{bewe-list}
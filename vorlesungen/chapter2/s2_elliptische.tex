\section{Elliptische Funktionen}

\subsection{Einführung}

\emph{Problem} Gibt es nicht-konstante holomorphe bzw. meromorphe Funktionen $f$ auf $\CC$ mit zwei über $\RR$ linear unabhängigen Perioden, d.\,h. existieren $\omega_1, \omega_2 \in \CC\setminus\{0\}$, die über $\RR$ linear unabhängig sind und so dass $f(z+\omega_1) = f(z) = f(z+\omega_2)$ für alle $z\in\CC$.

\begin{erin*}
Eine auf $\CC$ meromophe Funktion ist eine Abbildung $f\colon \CC \ra \closure \CC = \CC \cup \{\infty\}$ derart, dass $S(f) = f^{-1}(\{\infty\})$ diskret in $\CC$ ist, die Einschränkung $f_0 = f|_{\CC\setminus S(f)}$ holomorph ist und alle Punkte aus $S(f)$ Pole von $f_0$ sind.

Sind $f, g$ auf $\CC$ meromorph, so ist $f_0 + g_0$ auf $\CC\setminus(S(f) \cup S(g))$ holomorph und hat in $S(f) \cup S(g)$ nur unwesentliche Singularitäten, also lässt sich $f_0 + g_0$ eindeutig zu einer auf $\CC$ meromorphen Funktion $f+g$ fortsetzen.
Genauso kann man $f\cdot g$, $f'$ und $\frac{f}{g}$ (für $g\not\equiv 0$) als meromorphe Funktionen definieren. Damit bilden die auf $\CC$ meromorphen Funktionen einen Körper.
\end{erin*}

\begin{defi}
Eine Teilmenge $L \subseteq \CC$ heißt \myemph{Gitter}\footnote{Im Englischen \emph{lattice}, deshalb werden Gitter mit $L$ bezeichnet.}, falls es über $\RR$ linear unabhängige Zahlen $\omega_1, \omega_2\in\CC$ gibt, so dass
\[
	L = \Set{ m_1\omega_1 + m_2 \omega_2 \mid m_1, m_2 \in \ZZ}
	=: \ZZ\omega_1 \oplus \ZZ\omega_2
	\,.
\]
Man nennt $\Set{\omega_1, \omega_2}$ eine \myemph[Gitter!Basis]{Basis} des Gitters $L$.
\end{defi}

\begin{bsp}
Ein [Bild, siehe altes Skript]
\end{bsp}

\begin{beme-list}
\item Man zeigt leicht: $L\subseteq \CC$ ein Gitter mit Basis $\Set{\omega_1, \omega_2}$. Dann ist $\Set{\omega_1', \omega_2'}$ genau dann eine weitere Basis von $L$, wenn es ein $M \in GL_2(\ZZ) = \Set{A = \left(\begin{smallmatrix}a & b \\ c & d\end{smallmatrix}\right) \mid \det A = \pm 1}$ gibt, so dass 
\[
	\begin{pmatrix}
		\omega_1' \\
		\omega_2'
	\end{pmatrix}
	= M \cdot
		\begin{pmatrix}
		\omega_1 \\
		\omega_2
	\end{pmatrix}
	\,.
\]

\item \emph{Periodentorus:} Ist $L$ ein Gitter mit Basis $\Set{\omega_1, \omega_2}$, so kann man eine Äquivalenzrelation auf $\CC$ definieren durch
\[
	z \sim z'
	\Rla z-z' \in L
	\,.
\]
Die Äquivalenzklasse von $z$ ist gerade $[z] := z+L := \Set{z + \omega \mid \omega \in L}$, man bezeichnet die Mengen der Äquivalenzklassen mit $\CC/L$.
Man definiert $[z] + [z'] :=[z+z']$. Beachte: $\CC/L$ ist gerade die Faktorgruppe der abelschen Gruppe $(\CC, +)$ nach dem Normalteiler $(L, +)$.

\emph{Geometrisches Modell von $\CC/L$} Offenbar ist jeder Punkt $z\in\CC$ äquivalent zu einem Punkt in der Grundmasche
\[
	\F(\omega_1, \omega_1)
	= \Set{t_1\omega_1 + t_2\omega_2 \mid 0 \leq t_1, t_2 \leq 1}
	\,,
\]
denn für alle $z\in\CC$ existieren $x,y\in\RR$ mit
\[
	z
	= x\omega_1 + y\omega_2
	= (x-\floor{x})\omega_1 + (y-\floor{y})\omega_2 + \underbrace{\floor{x}\omega_1 + \floor{y}\omega_2}_{\in L}
	\,.
\]
Zwei Punkte in $\F(\omega_1, \omega_2)$ sind genau dann äquivalent, wenn sie entweder gleich oder auf gegenüberliegenden Rändern liegen.
Man erhält ein geometrisches Modell von $\CC/L$ indem man gegnüberliegende Ränder identifiziert, man erhält dann einen Torus\footnote{Auch \emph{Donut} genannt.}.
\end{beme-list}

\begin{defi}
Sei $L\subseteq \CC$ ein Gitter mit Basis $\Set{\omega_1, \omega_2}$. Dann heißt eine meromorphe Funktion $f\colon \CC \ra \closure \CC$ \myemph{elliptisch} bezüglich $L$, falls gilt $f(z+\omega_1) = f(z) = f(z+\omega_2)$ für alle $z\in\CC$.
\end{defi}

\begin{beme-list}
\item Es gilt $f(z+\omega_1) = f(z) = f(z+\omega_2)$ für alle $z\in\CC$ genau dann, wenn $f(z+\omega) = f(z)$ für alle $z\in\CC$ und $w\in L$.
Deswegen heißen elliptische Funktionen auch doppelt periodische Funktionen.

\item Ist $c\in\closure\CC$ und $f(z) = c$ und $f$ elliptisch, so gilt $f(z+\omega) = c$ für alle $\omega \in L$.
Insbesondere macht es Sinn von den Null- oder Polstellen von $f$ modulo L zu sprechen.

\item Die elliptischen Funktionen zu einem Gitter $L$ bilden einen Körper $K(L)$. Dieser enthält $\CC$. Und für $f\in K(L)$ gilt auch $f'\in K(L)$.

\item \emph{Historie:} Der Name \emph{elliptische Funktionen} kommt von der Theorie der \emph{elliptischen Integrale}, d.\,h. Integralen der Form 
\[
	\int_a^z \frac{1}{\sqrt{P(t)}} \opd t
	\,,
\]
wobei $P(t)$ ein Polynom dritten oder vierten Gerade ohne mehrfache Nullstellen ist.
(Der Wert hängt im Allgemeinen von der Wahl von $a$, des Integtrationsweges und der Wahl der Wurzel ab.)
Solche Integrale treten bei der Berechnung von Bogenlängen von Ellipsen auf.
Man kann zeigen, dass die Umkehrfunktion eines elliptischen Integrales gerade eine elliptische Funktion ist.

Geometrisch gesehen sind Ellipsen (gegeben durch Gleichung der Form $ax^2+by^2 = c$) verschieden von elliptischen Kurven (das sind Gleichungen der Form $y^2 = x^3+ax+b$ mit $4a^3 - 27b^2 \not= 0$).

Erstere werden parametrisiert durch rationale Funktionen (z.,B. wird $x^2+y^2 = 1$ parametrisiert durch $(x, y) = (\frac{2t}{t^2+1}, \frac{t^2-1}{t^2+1})$), letztere durch elliptische Funktionen.
Erstere haben \myquote{Geschlecht Null} (isomorph zu $\mathds{P}_1(\CC)$), letztere haben \myquote{Geschlecht Eins} (isomorph zu einem Torus).
\end{beme-list}

\subsection{Die Liouvillschen Sätze}

Es handelt sich um vier Sätze, die notwendige Bedingungen geben für die Existenz von elliptischen Funktionen.

\begin{satz}
Eine elliptische Funktion $f\colon \CC \ra \closure \CC$ ohne Polstellen ist notwendigerweise konstant.
\end{satz}

\begin{bewe}
Sei $f$ elliptisch zum Gitter $L = \ZZ\omega_1 \oplus \ZZ\omega_2$ mit Grundmasche
\[
	\F
	= \Set{t_1\omega_1 + t_2\omega_2 \mid 0 \leq t_1, t_2 \leq 1}
	\,.
\]
Beachte
\begin{enumerate*}
\item $\F$ ist kompakt
\item zu jedem $z\in\CC$ gibt es $\omega \in L$, so dass $z+\omega \in \F$
\item $f(z+w) = f(z)$ für alle $z \in \CC$ und $\omega \in L$.
\end{enumerate*}

Es folgt: $f$ nimmt schon jeden seiner Werte auf dem Kompaktum $\F$ an, und $f$ ist ganz, also insbesondere stetig und damit auf jedem Komapaktum beschränkt.
Also ist $f$ bereits auf ganz $\CC$ beschränkt.
Nach Lioville (Funktionentheorie 1) ist $f$ somit bereits konstant.
\end{bewe}

\begin{satz}
Sei $f \in K(L)$.
Dann gilt
\[
	\sum_{z\in\CC/L} \res_z f
	= 0
	\,,
\]
wobei die linke Seite über ein Repräsentantensystem aller Punkte $z\in \CC/L$ läuft und nur die Pole von $f$ ungleich Null sind.
\end{satz}

\begin{beme-noind}
Die Summe links ist wohldefiniert, denn
\begin{enumerate}
\item  $f$ hat nur endlich viele Polstellen modulo $L$.
\emph{Denn} andernfalls hätte $f$ unendlich viele Polstellen in $\F$, und da $\F$ kompakt ist, hätte $S(f)$ nach dem Satz von Weierstraß hätte einen Häufungspunkt \blitz.

\item Das Residuum ist invariant unter Verschiebung um $\omega \in L$.
\emph{Denn} das Residuum ist der Koeffizient mit Nummer -1 in der Laurententwicklung.
Um $z_0 + \omega$ ist dies eine Summe von Potenzen $\frac{1}{z-(z_0+\omega)} = \frac{1}{(z-\omega)-z_0}$, also erhält man die Laurententwicklung von $f(z-\omega) = f(z)$ um $z_0$.
\end{enumerate}
\end{beme-noind}

\begin{bewe}
Für $a\in\CC$ sei $\F_a = a + \F = \Set{a + z \mid z\in\F}$.
Dann kann $\CC/L$ mit $\F_a$ (modulo Randidentifikation) identifiziert werden, denn zu $z\in\CC$ existiert $\omega \in L$ mit $z-a+\omega \in \F$, d.\,h. $z+\omega \in a + \F = \F_a$.
Man wähle $a\in\CC$ so, dass auf dem Rand $\partial \F_a$ von $\F_a$ kein Pol von $f$ liegt.
Dies geht, da $f$ auf $\F$ nur endlich viele Pole hat.

Man wende den Residuensatz an, unter Beachtung, dass nach der Wahl von $a$ das Innere $\innere F_a$ genau ein Repräsentanten jeder Polstelle von $f$ enthält.
Es folgt also
\begin{align*}
	2\pi i \sum_{z\in\CC/L} \res_z f
	&= 2\pi i \sum_{z\in \innere \F_a} \res_z f
	= \int_{\partial \F_a} f(z)\opd z \\
	&= \int_{C_1} f(z) \opd z + \int_{C_2} f(z) \opd z + \int_{C_3} f(z) \opd z + \int_{C_4} f(z) \opd z
	\,.
\end{align*}
Hier heben sich das erste und das dritte und das zweite und das vierte Integral jeweils auf, die Summe ist also Null.
\emph{Denn}
\[
	\int_{C_4} f(z)\opd z
	= \int_{\omega_1 + C_4} f(z-\omega_1) \opd z
	= \int_{\omega_1 + C_4} f(z)\opd z
	= -\int_{C_2} f(z) \opd z
	\,.
\]
\end{bewe}

\begin{koro}
Es gibt keine elliptische Funktion $f$ mit genau einer einfachen Polstelle modulo $L$.
\end{koro}
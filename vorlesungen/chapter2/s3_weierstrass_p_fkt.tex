\section[Die Weierstraß'sche \texorpdfstring{$\wp$}{p}-Funktion]{Die Weierstraß'sche \texorpdfstring{\boldmath$\wp$}{p}-Funktion}

Unsere Ziele in diesem Abschnitt sind
\begin{enumerate}
\item Die Konstruktion einer elliptischen Funktion $\wp(z)$ zum Gitter $L$ mit genau doppelten Polstellen in allen Gitterpunkten $\omega \in L$.
Die Konstruktion erfolgt nach dem Rezept des Beweis des Partialbruchsatzes von Mittag-Leffler (\autoref{satz:mittag-leffler}):
Ein guter Kandidat wäre
\[
	\sum_{\omega \in L} \frac{1}{(z-\omega)^2}
	\,.
\]
Allerdings hat diese Reihe schlechte Konvergenzeigenschaften. Das Lösen wir durch konvergenzerzeugende Summanden, also ein besserer Kandidat
\[
	\wp(z)
	= \frac{1}{z^2} + \sum_{\omega \in L\setminus\{0\}} \frac{1}{(z-\omega)^2} - \frac{1}{\omega^2}
	\,.
\]

\item Man zeige $K(L) = \CC(\wp) + \wp'\CC(\wp)$,
wobei $\CC(\wp)$ der Körper der rationalen Funktionen in $\wp$ ist.\footnote{Das heißt jedes $f \in \CC(\wp)$ lässt sich mit $a_0$, \ldots, $a_n$ und $b_0$, \ldots, $b_m \in \CC$ schreiben als
\[
	f(z) = \frac{a_0+a_1\wp(z)+\ldots+a_n\wp(z)^n}{b_0+b_1\wp(z)+\ldots+b_m\wp(z)^m}
	\,.
\]}
\end{enumerate}

\begin{nota}
In diesem Abschnitt sei $L = \ZZ\omega_1 \oplus \ZZ\omega_2 \subset \CC$ ein Gitter.
Satt $\sum_{\omega \in L\setminus\{0\}} \ldots$ schreiben wir einfach $\sumprime_{\omega \in L} \ldots$
\end{nota}

\begin{lemm}\label{lemm:weierstrass-konv}
Sei $r > 2$.
Dann ist die Reihe
\[
	\sumprime_{w \in L} \frac{1}{\abs \omega ^2}
\]
konvergent.
\end{lemm}

\begin{bewe}
Sei $f\colon \RR^2\setminus\{(0,0)\} \ra \RR$ mit
\[
	f(x, y)
	= \frac{\abs{x\omega_1 + y\omega_2}^r}{\abs{xi+y}^r}
	\,.
\]
Da $\Set{\omega_1, \omega_2}$ linear unabhängig über $\RR$ ist, gilt $f(x,y) >0$ für alle $(x, y) \in \RR^2\setminus\{(0,0)\}$.

Außerdem gilt $f(\lambda x, \lambda y) = f(x, y)$ für alle $\lambda \in \RR^\times$ und $(x,y) \in \RR^2\setminus\{(0,0)\}$. Da $f$ stetig auf dem Kompaktum $S' = \Set{(x,y) \in \RR^2 \mid x^2 + y^2 = 1}$ ist, folgt, dass es $C > 0$ gibt mit $f(x, y) > C$ für alle $(x, y) \in \RR^2\setminus\{(0,0)\}$.
Daher
\[
	\frac{1}{\abs{x\omega_1 + y\omega_2}^r}
	< \frac{1}{C} \cdot \frac{1}{\abs{xi + y}^r}
	\qquad \text{für alle } (x,y) \in \RR^2\setminus\{(0,0)\}
	\,.
\]
Man wende dies an mit $(x, y) = (m, n) \in \ZZ^2$, $(m, n) \not= (0,0)$.
Damit genügt es die Konvergenz für $L = \ZZ \oplus \ZZ$ zu zeigen, d.\,h.
\[
	\sumprime_{m,n} \frac{1}{\abs{mi+n}^r} < \infty
	\,.
\]
Die euklidische Norm ist äquivalent zur Maximusmnorm auf $\RR^2$.
Also genügt es zu zeigen, dass
\[
	\sumprime_{m,n} \frac{1}{\norm{(m,n)}_\infty^r} < \infty
	\,.
\]

Es gilt
\begin{align*}
	\sumprime_{m,n} \frac{1}{\norm{(m,n)}_\infty^r}
	&= \sum_{N=1}^\infty \underbrace{\# \Set{(m, n) \in \ZZ \mid \norm{(m,n)}_\infty^r = N}}_{=8N} \frac{1}{N^r} \\
	&\leq 8 \sum_{N=1} \frac{1}{N^{r-1}}
	< \infty
	\qquad \text{für } r > 2
	\,.
\end{align*}
\end{bewe}

\begin{satz-list}
\item Die Reihe
\[
	\wp(z)
	:= \frac{1}{z^2} + \sumprime_{\omega \in L} \left(\frac{1}{(z-\omega)^2} - \frac{1}{\omega^2}\right)
	\qquad \text{für } z\in\CC\setminus L
\]
ist auf Kompakta $K\subset \CC\setminus L$ gleichmäßig, absolut konvergent und heißt \myemph{Weierstraß'sche $\wp$-Funktion}.
Sie ist holomorph auf $\CC\setminus L$ und hat doppelte Pole in allen $w\in L$.

\item Es gilt
\[
	\wp'(z) = -2\sum_{\omega \in L} \frac{1}{(z-\omega)^3}
	\qquad \text{für } z\in\CC\setminus L
\]
wobei die Reihe absolut konvergiert.
Insbesondere ist $\wp'$ ungerade und elliptisch.

\item $\wp$ ist elliptisch zum Gitter $L$ mit Ordnung 2.
\item Um $z=0$ hat $\wp(z)$ die Laurententwicklung
\[
	\wp(z) = \frac{1}{z^2} + \sum_{n\geq 1} (2n+1) G_{2n+2} z^{2n}
	\qquad \text{für } 0 < \abs z < \rho
	\,,
\]
wobei
\[
	G_{2k}
	= \sumprime_{w\in L} \frac{1}{\omega^{2k}}
	\qquad \text{für } k\in\NN, k\geq 2
\]
die sogeannten \myemph{homogenen Eisensteinreihen} vom Gewicht $2k$ zu $L$ sind. Und
\[
	\rho := \min \Set{ \abs\omega \mid \omega \in L, \omega \not= 0}
	\,.
\]
\end{satz-list}

\begin{bewe-list}
\item Sei $K\subset \CC\setminus L$ kompakt.
Es gelte $\abs z < R$ für alle $z\in K$.
Sei $\omega \in L$ mit $\abs \omega > 2R$.
Durch Abschätzen ergibt sich
\begin{align*}
	\abs{\frac{1}{(z-\omega)^2} - \frac{1}{\omega^2}}
	&= \abs{\frac{\omega^2-z^2+2\omega z -\omega^2}{(z-\omega)^2\omega^2}}
	= \frac{\abs{z(2\omega - z)}}{\abs \omega ^2 \abs{z-\omega}^2} \\
	&\leq \frac{R(2\abs \omega + R)}{\abs \omega ^2(\abs \omega - R)^2}
	\leq \frac{R(2\abs \omega + \frac{\abs \omega}{2})}{\abs \omega ^2(\abs \omega - \frac{\abs \omega}{2})^2}
	= \frac{10R}{\abs \omega ^3}
	\,.
\end{align*}
Denn es gilt $\abs z \leq R < 2R < \abs w$ also $R < \frac{\abs \omega}{2}$ und
\[
	\abs{z-\omega}^2
	= \abs{\omega - z}^2
	\geq (\abs \omega - \abs z)^2
	\geq (\abs \omega - R)^2
	\,.
\]
Damit folgt die Behauptung wegen \autoref{lemm:weierstrass-konv} mit $r=3$.

Nach dem Satz von Weierstraß (Funktionentheorie 1) ist somit $\wp(z)$ holomorph auf $\CC\setminus L$.
Nach Definition hat $\wp(z)$ Pole zweiter Ordnung in $\omega \in L$.
Der Hauptteil ist gerade $\frac{1}{(z-\omega)^2}$.

\item Wegen lokaler gleichmäßiger Konvergent darf man die Reihe $\wp(z)$ gliedweise differenzieren
\[
	\wp'(z)
	= -2\frac{1}{z^3} - 2 \sumprime_{\omega \in L} \frac{1}{(z-\omega)^3}
	= -2\sum_{\omega\in L} \frac{1}{(z-\omega)^3}
	\qquad \text{für } z\in\CC\setminus L
	\,.
\]
Die absolute Konvergent lässt sich wie in (i) zeigen (Beachte Exponent 3 und \autoref{lemm:weierstrass-konv}).

Wegen der absoluten Konvergenz der Reihe $\wp'$ ist $\wp'$ elliptisch, denn sei $\omega_0 \in L$:
\begin{align*}
	\wp'(z + \omega_0)
	&= -2 \sum_{\omega \in L} \frac{1}{(z+\omega_0 - \omega)^3} \\
	&= -2 \sum_{\omega \in L} \frac{1}{(z-(\omega - \omega_0))^3}
	= -2 \sum_{\omega \in L} \frac{1}{(z - \omega)^3}
	\,,
\end{align*}
denn die Reihe ist absolut also auch unbedingt konvergent und mit $\omega$ durchläuft auch $\omega-\omega_0$ ganz $L$.

\item Für $j=1,2$ betrachten wir
\[
	\wp(z+\omega_j) - \wp(z)
	\,.
\]
Differenziert man nun, erhält man, da $\wp'$ elliptisch ist
\[
	\wp'(z+\omega_j) - \wp'(z)
	= 0
	\,.
\]
Da $\CC\setminus L$ ein Gebiet ist, folgt, dass
\[
	\wp(z+\omega_j) - \wp(z)
	= c_j
\]
konstant ist.
Für $z = -\frac{\omega_j}{2} \not\in L$ gilt dann
% Ich weiß nicht was ich hier mitgeschrieben habe. Stattdessen das aus dem alten Skript
%\[
%	c_j
%	= \wp\left(\frac{\omega_j}{2}\right)
%	= \wp\left(-\frac{\omega_j}{2}+\omega_j\right) - \wp\left(-\frac{\omega_j}{2}\right)
%	= -\wp\left(\frac{\omega_j}{2}\right)
%	= -c_j
%	\,.
%\]
da $\wp$ gerade ist
\[
	0
	= \wp\left(\frac{\omega_j}{2}\right) - \wp\left(\frac{\omega_j}{2}\right)
	= \wp\left(-\frac{\omega_j}{2} + \omega_j\right) - \wp\left(- \frac{\omega_j}{2}\right)
	= c_j
	\,.
\]

Also ist $c_j = 0$ und damit $\wp$ elliptisch.
Und $\ord \wp = 2$ folgt, da $\wp$ doppelte Polstellen in $L$ hat und holomorph auf $\CC\setminus L$ ist.

\item Es gilt $\wp(z) = \frac{1}{z^2} + g(z)$ mit
\[
	g(z)
	= \sumprime_{\omega \in L} \left(\frac{1}{(z-\omega)^2} - \frac{1}{\omega^2}\right)
	\qquad \text{für } \abs z < \rho
	\,.
\]
Durch sukzessives ableiten erhält man
\[
	g^{(n)}(z)
	= (-1)^n (n+1)! \sumprime_{\omega \in L} \frac{1}{(z-\omega)^{n+2}}
	\qquad \text{für } n \geq 1
	\,.
\]

Für $z=0$ gilt dann
\begin{align*}
	g^{(n)}(0)
	&= (-1)^n (n+1)! \sumprime_{\omega \in L} \frac{1}{(-\omega)^{n+2}} \\
	&= (n+1)! \sumprime_{\omega \in L} \frac{1}{\omega^{n+2}}
	=
	\begin{cases}
		0 & n \text{ ungerade} \\
		(n+1)!\ G_{n+2} & n \text{ gerade}
	\end{cases}
	\,.
\end{align*}
Es folgt mit dem Satz von Taylor
\[
	g(z)
	= \sum_{n=0}^\infty \frac{g^{(n)}(0)}{n!}z^n
	= \sum_{\substack{n \geq 1 \\ \scriptscriptstyle n \text{ gerade}}} (n+1)G_{n+2}z^n\,.
\]
\end{bewe-list}

\begin{satz}\label{satz:diff-wp}
Es gilt die Differentialgleichung
\[
	\wp'(z)^2
	= 4\wp(z)^3 - g_2\wp(z) - g_3
	\,,
\]
wobei $g_2 = 60G_4$ und $g_3 = 140G_6$, die sogenannten \myemph{Weierstraß-Konstanten} des Gitters $L$ sind.
\end{satz}

\begin{bewe}
Wir wollen beide Seiten in eine Laurentreihe um $z=0$ entwickeln und die Differenz bilden.

Es ist
\[
	\wp(z) = \frac{1}{z^2} + 3G_4z^2 + 5G_6z^4 + \ldots
\]
Also
\[
	\wp'(z) = -\frac{2}{z^3} + 6G_4z + 20G_6z^3 + \ldots
\]
Damit folgt
\[
	\wp'(z)^2 = \frac{4}{z^6} - \frac{24G_4}{z^2} - 80G_6 + \ldots
\]
Ähnlich erhalten wir
\[
	4\wp(z)^3
	= \frac{4}{z^6} + \frac{36G_4}{z^2} + 60G_6 + \ldots
\]
Also
\[
	\wp'(z)^2 - 4\wp(z)^3 + g_2\wp(z)
	= -140G_6 + \ldots
\]

Wir erhalten, dass $H(z) := \wp'(z)^2 - 4\wp(z)^3 + g_2\wp(z) \in K(L)$ eine holomorphe Fortsetzung in $z=0$ hat, denn $H$ hat kein Hauptteil in $z=0$.
Und hat dort den Wert $-140G_6 = -g_3$.
Wegen der Periodizität ist daher $H(z) \in K(L)$ und holomorph auf ganz $\CC$.
Nach dem 1. Liouvillschen Satz (\autoref{satz:liouville-1}) ist $H$ konstant,
also $H(z) = H(0) = -g_3$.
\end{bewe}

\begin{beme}
Eine elliptische Kurve $y^2=4x^3+ax-b$ wird für $a=-g_2$ und $b=-g_3$ durch die $\wp$-Funktion parametrisiert.
\end{beme}

\begin{koro}
Sei $\omega_3 := \omega_1 + \omega_2$ und $e_j = \wp(\frac{\omega_j}{2})$ für $j=1,2,3$.
Dann gilt
\[
	\wp'(z)^2 = 4(\wp(z)-e_1)(\wp(z)-e_2)(\wp(z)-e3)
	\,.
\]
\end{koro}

\begin{beme}
Die Punkte $\frac{\omega_1}{2}$, $\frac{\omega_2}{2}$ und $\frac{\omega_3}{2}$ sind \emph{ausgezeichnet} dadurch, dass sie ein Vertretersystem modulo $L$ der sogennanten \myemph{Zweiteilungspunkten} von L sind, d.\,h. $z\in\CC$ mit
\[
	2z \equiv 0 \mod L
\]
\end{beme}

\begin{bewe}
Beachte da $\wp'$ elliptisch und ungerade ist, gilt
\[
	\wp'\left(\frac{\omega_j}{2}\right)
	= \wp'\left(\frac{\omega_j}{2}-\omega_j\right)
	= \wp'\left(-\frac{\omega_j}{2}\right)
	= - \wp'\left(\frac{\omega_j}{2}\right)
\]
Also $\wp'(\frac{\omega_j}{2}) = 0$ für $j=1,2,3$.

\emph{Behauptung}
$e_1, e_2, e_3$ sind paarweise verschieden.
Denn Angenommen es gilt $e_k = e_j$ mit $k \not= j$.
So folgt
\[
	{\textstyle \ord_{\frac{\omega_k}{2}}} (\wp-e_k) \geq 2
	\,,\qquad
	{\textstyle \ord_{\frac{\omega_j}{2}}} (\wp-e_j) \geq 2
	\,,
\]
denn $\wp(\frac{\omega_j}{2}) = e_j$ und $\wp'(\frac{\omega_j}{2}) = 0$ genauso für $k$.
Da $\frac{\omega_k}{2} \not\equiv \frac{\omega_j}{2} \mod L$, folgt dass $\wp$ den Wert $e_j = e_k$ mindestens vier mal modulo $L$ annimmt.
Aber die Ordnung von $\wp$ ist gleich 2. \blitz

Das Polynom $4X^3-g_2X-g_3$ hat also nach \autoref{satz:diff-wp} die paarweise verschiedenen Nullstellen $e_1$, $e_2$ und $e_3$.
Also
\[
	4X^3-g_2X-g_3 = 4(X-e_1)(X-e_2)(X-e_3)
	\,.
\]
Mit $X = \wp(z)$ folgt die Behauptung mit \autoref{satz:diff-wp}.
\end{bewe}

\begin{erin}
Sei $K$ ein Körper (mit Addition $+$ und Multiplikation $\cdot$\,).
Dann heißt $k\subset K$ \myemph{Teilkörper}, falls $(k,+)$ eine Untergruppe von $(K,+)$ und $(k^\times,\cdot)$ eine Untergruppe von $(K^\times, \cdot)$ ist.
Man sagt auch, dass $K\mid k$ eine \myemph{Körpererweiterung} ist.
Man beachte, dass $k$ zusammen mit eingeschränkter Addition $+$ und Multiplikation $\cdot$ selbst ein Körper ist.

\emph{Beobachtung:}
Man kann $K$ als $k$-Vektorraum auffassen, mit Skalarmultiplikation
\[
	k\times K\ra K,
	\quad (\lambda,a)\mapsto \lambda a
	\,.
\]
Man nennt $\dim_k(K)$ den \myemph{Körpergrad} der Erweiterung $K\mid k$ und schreibt dafür $[K:k]$.
(Beispiel: $\CC\mid\RR$ hat Grad 2 ($\RR$-Basis von $\CC$ ist 1 und $i$).)

Seien $K$ und $K'$ Körper.
Eine Abbildung
\[
	\phi\colon K\ra K'
\]
heißt \myemph{Körperhomomorphismus}, falls für alle $a, b \in K$ gilt
\begin{align*}
	\phi(a+b) &= \phi(a)+\phi(b)\,, \\
	\phi(a\cdot b) &= \phi(a)\cdot\phi(b)\,, \\
	\phi(1)&=1
\end{align*}
Dann ist $\phi$ automatisch \emph{injektiv}, denn für $a \neq 0$ gilt
\[
	\phi(a)\phi(a^{-1})
	= \phi(a a^{-1})
	= \phi(1)= 1
	\,,
\]
also folgt $\phi(a)\neq 0$.
Man nennt $\phi$ einen Isomorphismus, falls $\phi$ \emph{bijektiv} ist (dafür genügt es nach Obigem die \emph{Surjektivität} von $\phi$ zu zeigen).
\end{erin}

\begin{satz-list}[Struktursatz]
\item Sei $K_+(L) = \Set{f\in K(L) \mid f \text{ gerade}}$ der Teilkörper der geraden elliptischen Funktionen.
Dann ist die Abbildung
\[
	\phi\colon \CC(X) \ra K_+(L)\,,
	\quad R(X) \mapsto R(\wp)
\]
wohldefiniert und ein Körper-Isomorphismus.
Insbesondere ist die Abbildung also surjektiv, das heißt $K_+(L) = \CC(\wp)$.

\item Die Körpererweiterung $K(L) \mid K_+(L)$ hat Grad 2 und eine $K_+(L)$-Basis von $K(L)$ ist 1 und $\wp'$.
Insbesondere ist also
\[
	K(L)
	= \CC(\wp) \oplus \CC(\wp) \wp'
	\,.
\]
\end{satz-list}

\begin{beme}[Übung]
Man sieht direkt (mit Hilfe der Differentialgleichung \autoref{satz:diff-wp}), dass $\CC(\wp) \oplus \CC(\wp) \wp'$ tatsächlich ein Körper ist.
\end{beme}

\begin{bewe-list}
\item \emph{Wohldefiniertheit}: Dafür zeigen wir zunächst, ist $P(X)\in \CC[X]$ nicht das Nullpolynom, so ist $P(\wp)$ auch nicht die Nullfunktion.

Ist $P$ konstant, so folgt die Behauptung.
Sonst sei $\grad(P(X)) = n \geq 1$.
Dann hat $P(\wp)$ eine Polstelle der Ordnung $2n$ in $z = 0$, also kann $P(\wp)$ nicht identisch Null sein.

Weiterhin gilt, da $\wp$ gerade ist, ist auch $R(\wp)$ gerade, also $R(\wp)\in K_+(L)$.

Offensichtlich ist die Abbildung $\phi$ ein Homomorphismus.
Also müssen wir noch die Surjektivität von $\phi$ zeigen, dafür genügt es für $f\in K_+(L)$ zu zeigen, dass $f\in\CC(\wp)$ gilt.

\emph{Behauptung}: Sei $f\in K_+(L)$ und $S(f)\subset L$. Dann ist $f$ ein Polynom in $\wp$.

\begin{bewe-ind}
Ohne Einschränkung sei $f$ nicht konstant.
Nach dem 1. Satz von Liouville (\autoref{satz:liouville-1}) gilt dann $S(f) \neq \emptyset$.
Sei $\omega_0 \in S(f)$.
Wegen $f(z - \omega_0 + \omega) = f(z)$ für alle $\omega \in L$ folgt mit $z = \omega_0$, dass $f(\omega) = f(\omega_0) = \infty$ für alle $\omega\in L$.
Also folgt $L \subset S(f)$ und wegen $S(f) \subset L$ folgt $S(f) = L$.

Insbesondere ist $0\in S(f)$.
Da $f$ \emph{gerade}, hat die Laurententwicklung von $f$ um 0 die Gestalt
\[
	f(z) = a_{-2N} z^{-2N} + \text{höhere Terme}
	\qquad \text{für } 0< \abs{z} <\rho
	\,,
\]
wobei $N\geq 1$, $a_{-2N} \neq 0$ (d.\,h., $f$ hat in 0 einen Pol der Ordnung $2N$).

Außerdem gilt $\wp(z)^N = \frac{1}{z^{2N}} + \text{höhere Terme}$ für $0<\abs{z} < \rho$.
Betrachte
\[
	g(z)
	:= f(z) - a_{-2N} \cdot\wp(z)^N
	\,.
\]
Dann hat $g(z)$ nach Konstruktion einen Pol in $z=0$ einer Ordnung echt kleiner als $2N$.
Auch ist $g(z)$ elliptisch und gerade.
Weiterhin gilt $S(g) \subset L$.
Nun machen wir den gleichen Schluss mit $f(z)$ ersetzt durch $g(z)$.
Verfährt man induktiv weiter, so erhält man nach endlich vielen Schritten ein Polynom $P(X)\in \CC[X]$, so dass $f - P(\wp)$ holomorph fortsetzbar ist in $z=0$ (d.\,h. dort keinen Hauptteil mehr hat).
Wegen $f - P(\wp) \in K(L)$, ist somit $f - P(\wp)$ holomorph auf ganz $\CC$. Nach dem ersten Liouvill'schen Satz (\autoref{satz:liouville-1}) folgt $f-P(\wp)$ ist konstant.
\end{bewe-ind}

Sei $f\in K_+(L)$, $a\in S(f)$, $a \not\in L$.
Es gelte, dass $f$ einen Pol der Ordnung $N$ in $a$ hat.
Dann hat $(\wp(z) - \wp(a))^N f(z)$ eine hebbare Singularität in $z=a$, denn $(\wp(z) - \wp(a))^N$ hat eine Nullstelle von mindestens $N$-ter Ordnung in $z=a$.
Da $f$ nur endlich viele Polstellen modulo $L$ hat, existieren $a_1$, \ldots, $a_s\in \CC\setminus L$ und $N_1$, \ldots, $N_s\in\NN$, so dass
\[
	\prod_{j=1}^s (\wp(z) - \wp(a_j) )^{N_j} \cdot f(z)
\]
höchstens Polstellen in $L$ hat.
Da diese Funktion gerade ist, ist sie nach dem schon Bewiesenen ein Polynom in $\wp$.
Daher ist $f$ eine rationale Funktion in $\wp$.



\item Sei $f(z) \in K(L)$.
Dann ist $f(-z)\in K(L)$.
Also sind $f_1(z) = \frac{f(z) + f(-z)}{2}$ und $f_2(z) = \frac{f(z) - f(-z)}{2}$ in $K(L)$.
Außerdem ist $f_1$ gerade und $f_2$ ungerade.
Nun gilt mit (i)
\[
	f(z)
	= f_1(z) + f_2(z)
	= \underbrace{f_1(z)}_{\text{gerade}} + \underbrace{\frac{f_2(z)}{\wp'(z)}}_{\text{gerade}} \wp'(z)
	\in \CC(\wp) + \CC(\wp)\wp'
	\,.
\]
Es verbleibt zu zeigen, dass 1 und $\wp'$ über $K_+(L) = \CC(\wp)$ linear unabhängig sind.

Gelte $g+h\wp ' = 0$ mit $g$, $h\in K_+(L)$.
Ist nun $h\neq 0$, so folgt $\wp' = -\frac{g}{h}$ und dies ist ein Widerspruch, da $g$ und $h$ gerade sind.
Also ist $h=0$ und damit auch $g=0$.
\end{bewe-list}
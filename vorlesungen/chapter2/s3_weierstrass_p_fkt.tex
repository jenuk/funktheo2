\section[Die Weierstraß'sche \texorpdfstring{$\wp$}{p}-Funktion]{Die Weierstraß'sche \texorpdfstring{\boldmath$\wp$}{p}-Funktion}

Unsere Ziele in diesem Abschnitt sind
\begin{enumerate}
\item Die Konstruktion einer elliptischen Funktion $\wp(z)$ zum Gitter $L$ mit genau doppelten Polstellen in allen Gitterpunkten $\omega \in L$.
Die Konstruktion erfolgt nach dem Rezept des Beweis des Partialbruchsatzes von Mittag-Leffler (\autoref{satz:mittag-leffler}):
Ein guter Kandidat wäre
\[
	\sum_{\omega \in L} \frac{1}{(z-\omega)^2}
	\,.
\]
Allerdings hat diese Reihe schlechte Konvergenzeigenschaften. Das Lösen wir durch konvergenzerzeugende Summanden, also ein besserer Kandidat
\[
	\wp(z)	
	= \frac{1}{z^2} + \sum_{\omega \in L\setminus\{0\}} \frac{1}{(z-\omega)^2} - \frac{1}{\omega^2}
	\,.
\]

\item Man zeige $K(L) = \CC(\wp) + \wp'\CC(\wp)$,
wobei $\CC(\wp)$ der Körper der rationalen Funktionen in $\wp$ ist.\footnote{Das heißt jedes $f \in \CC(\wp)$ lässt sich mit $a_0$, \ldots, $a_n$ und $b_0$, \ldots, $b_m \in \CC$ schreiben als
\[
	f(z) = \frac{a_0+a_1\wp(z)+\ldots+a_n\wp(z)^n}{b_0+b_1\wp(z)+\ldots+b_m\wp(z)^m}
	\,.
\]}
\end{enumerate}

\begin{nota}
In diesem Abschnitt sei $L = \ZZ\omega_1 \oplus \ZZ\omega_2 \subseteq \CC$ ein Gitter.
Satt $\sum_{\omega \in L\setminus\{0\}} \ldots$ schreiben wir einfach $\sumprime_{\omega \in L} \ldots$
\end{nota}

\begin{lemm}
Sei $r > 2$.
Dann ist die Reihe
\[
	\sumprime_{w \in L} \frac{1}{\abs \omega ^2}
\]
konvergent.
\end{lemm}

\begin{bewe}
Sei $f\colon \RR^2\setminus\{(0,0)\} \ra \RR$ mit
\[
	f(x, y)
	= \frac{\abs{x\omega_1 + y\omega_2}^r}{\abs{xi+y}^r}
	\,.
\]
Da $\Set{\omega_1, \omega_2}$ linear unabhängig über $\RR$ ist, gilt $f(x,y) >0$ für alle $(x, y) \in \RR^2\setminus\{(0,0)\}$.

Außerdem gilt $f(\lambda x, \lambda y) = f(x, y)$ für alle $\lambda \in \RR^\times$ und $(x,y) \in \RR^2\setminus\{(0,0)\}$. Da $f$ stetig auf dem Kompaktum $S' = \Set{(x,y) \in \RR^2 \mid x^2 + y^2 = 1}$ ist, folgt, dass es $C > 0$ gibt mit $f(x, y) > C$ für alle $(x, y) \in \RR^2\setminus\{(0,0)\}$.
Daher
\[
	\frac{1}{\abs{x\omega_1 + y\omega_2}^r}
	< \frac{1}{C} \cdot \frac{1}{\abs{xi + y}^r}
	\qquad \text{für alle } (x,y) \in \RR^2\setminus\{(0,0)\}
	\,.
\]
Man wende dies an mit $(x, y) = (m, n) \in \ZZ^2$, $(m, n) \not= (0,0)$.
Damit genügt es die Konvergenz für $L = \ZZ \oplus \ZZ$ zu zeigen, d.\,h.
\[
	\sumprime_{m,n} \frac{1}{\abs{mi+n}^r} < \infty
	\,.
\]
Die euklidische Norm ist äquivalent zur Maximusmnorm auf $\RR^2$.
Also genügt es zu zeigen, dass
\[
	\sumprime_{m,n} \frac{1}{\norm{(m,n)}_\infty^r} < \infty
	\,.
\]

Es gilt
\begin{align*}
	\sumprime_{m,n} \frac{1}{\norm{(m,n)}_\infty^r}
	&= \sum_{N=1}^\infty \underbrace{\# \Set{(m, n) \in \ZZ \mid \norm{(m,n)}_\infty^r = N}}_{=8N} \frac{1}{N^r} \\
	&\leq 8 \sum_{N=1} \frac{1}{N^{r-1}}
	< \infty
	\qquad \text{für } r > 2
	\,.
\end{align*}
\end{bewe}
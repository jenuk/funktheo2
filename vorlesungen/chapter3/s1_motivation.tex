\section{Motivation}

\emph{Grob gesagt} sind Modulformen auf der oberen Halbebene $\HH = \Set{z \in \CC \mid \Im(z) > 0}$ holomorphe Funktionen, die gewisse einfache Transformationseigenschaften unter diskreten Untergrupen von gebrochen linearen Transformationen haben.

Dies wollen wir nun genauer formulieren.
Sei $\SL_2(\RR) = \Set{M \in M_{2,2}(\RR) \mid \det(M) = 1}$.
Dann ist $\SL_2(\RR)$ eine Gruppe unter der gewöhnlichen Matrizenmultiplikation.
Jedes $M = \left(\begin{smallmatrix}a & b \\ c & d\end{smallmatrix}\right) \in \SL_2(\RR)$ definiert eine sogenannte gebrochen lineare Transformation von $\HH$ in sich selbst\footnote{Diese wird auch \myemph{Möbiustransformation} genannt.}
\[
	z \mapsto \frac{az + b}{cz + d}
	\,,
\]
\emph{denn} sei $z\in\HH$, d.\,h. $z=x+iy$ mit $y > 0$, dann gilt $cz+d \not= 0$ und
\begin{align}\label{eq:moebiusnachh}
	\nonumber
	\Im\left(\frac{az+b}{cz+d}\right)
	&= \Im\left( \frac{(az+b)\conj{(cz+d)}}{\abs{cz+d}^2}\right)
	= \Im\left( \frac{ac\abs z^2 + bd + adz + bc\conj z}{\abs{cz+d}^2} \right) \\
	&= \frac{(ad - bc)y}{\abs{cz+d}^2}
	= \frac{y}{\abs{cz+d}^2}
	> 0
	\,.
\end{align}

Eine \emph{Modulform} ist dann eine holomorphe Funktion $f\colon \HH \ra \CC$, die der Transformationseigenschaft
\[
	f\left(\frac{az+b}{cz+d}\right) = (cz+d)^k f(z)
\]
für alle $M = \left(\begin{smallmatrix}a & b \\ c & d\end{smallmatrix}\right)$ einer diskreten Untergruppe der $\SL_2(\RR)$ genügt mit einem $k\in\ZZ$ und gewissen sonstigen Eigenschaften genügt.
Hierbei wird die Zahl $k$ als das \myemph[Modulform!Gewicht]{Gewicht} der Modulform $f$ bezeichnet.

Über die Theorie der Modulformen hat die Funktionentheorie Anwendungen auf zahlentheoretische Probleme.

\subsection{Beispiel aus der Theorie der quadratischen Formen}

Sei $m\in\NN$ und $A = ( a_{\mu\nu})_{1\leq\mu,\nu\leq m} \in M_{m,m}(\ZZ)$.

\emph{Voraussetzungen}: $A$ ist gerade, d.\,h. die $a_{\mu\mu}$ sind für alle $\mu = 1$, \ldots, $m$ gerade, und $A$ ist symmetrisch, d.\,h. $a_{\mu\nu} = a_{\nu\mu}$ für alle $\mu$, $\nu = 1$, \ldots, $m$.
Zum Beispiel
\[
	A =
	\begin{pmatrix}
	2 & & & 0 \\
	& 2 \\
	& & \ddots \\
	0 & & & 2
	\end{pmatrix}
	\,.
\]
Für $x = \left(\begin{smallmatrix}x_1 \\ \vdots \\ x_m\end{smallmatrix}\right) \in M_{m,1}(\RR)$ sei
\[
	Q(x)
	= \frac{1}{2} x^t A x
	\,.
\]

Es gilt somit
\begin{align*}
	Q(x)
	&= \frac{1}{2} (x_1e_1^t + \ldots + x_me_m^t) A (x_1e_1 + \ldots + x_me_m) \\
	&= \frac{1}{2} \sum_{1\leq\mu,\nu\leq m} x_\mu x_\nu e_\mu^t A e_\nu
	= \frac{1}{2} \sum_{1\leq\mu,\nu\leq m} a_{\mu\nu}x_\mu x_\nu \\
	&= \sum_{1\leq \mu < \nu \leq m} a_{\mu\nu}x_\mu x_\nu + \sum_{\mu=1}^m \frac{a_{\mu\mu}}{2} x_\mu^2
	\,.
\end{align*}
für $e_\nu$ der $\nu$-te Standard-Einheitsnvektor.
Die letzte Gleichheit gilt wegen $a_{\mu\nu} = a_{\nu\mu}$.
Also ist $Q(x)$ eine ganzzahlige (man beachte, dass $a_{\mu\mu}$ gerade ist) quadratische Form (d.\,h. ein homogenes Polynom vom Grad 2) in den Variablen $x_1$, \ldots, $x_m$.

Wir wollen nun zusätzlich Voraussetzen, dass $Q$ positiv definit ist, d.\,h. $Q(x) > 0$ für alle $x \in M_{m,1}(\RR)$ mit $x\not=0$.

Für $n\in\NN$ setzen wir nun
\[
	r_Q(n)
	:= \# \Set{x \in M_{m,1}(\ZZ) \mid Q(x) = n}
\]
die Anzahl der Darstellungen von $n$ durch $Q$.
\begin{bsp}
Sei
\[
	A =
	\begin{pmatrix}
	2 & & & 0 \\
	& 2 \\
	& & \ddots \\
	0 & & & 2
	\end{pmatrix}
	\,.
\]
Dann ist
\[
	Q(x) = x_1^2 + \ldots + x_m^2
	\,,
\]
also ist in diesem Fall $r_Q(n)$ die Anzahl der Darstellungen von $n$ als Summe von $m$ Quadraten.
\end{bsp}

\begin{lemm}
Für eine solche Quadratische Form $Q$ gilt dann $r_Q(n) < \infty$.
\end{lemm}

\begin{bewe}
Aus der Linearen Algebra wissen wir, dass sich $A$ durch eine orthogonale Matrix diagonalisieren lässt, d.\,h. es gibt ein orthogonales $U \in \GL_m(\RR)$, d.\,h. $U^tU = E$, so dass
\[
	U^tAU =
	\begin{pmatrix}
	\lambda_1 & & & 0 \\
	& \lambda_2 \\
	& & \ddots \\
	0 & & & \lambda_m
	\end{pmatrix}
	\,,
\]
mit $\lambda_\nu \in \RR$ für alle $\nu=1$, \ldots, $m$\footnote{Dies sind gerade die Eigenwerte von $A$}.
Da $Q$ positiv definit ist, gilt
\[
	\lambda_\nu =
	e_\nu^t
			\begin{pmatrix}
			\lambda_1 & & & 0 \\
			& \lambda_2 \\
			& & \ddots \\
			0 & & & \lambda_m
			\end{pmatrix}
		e_\nu
	= (Ue_\nu)^tA(Ue_\nu)
	> 0
\]
für alle $1 \leq \nu \leq m$. Sei
\[
	M
	:= \Set{x\in M_{m,1}(\RR) \mid Q(x) = n}
	\,.
\]
Es ist
\[
	Q(x)
	= \frac{1}{2}x^tAx
	= \frac{1}{2} x^t(U ^t)^{-1}
			\begin{pmatrix}
			\lambda_1 & & & 0 \\
			& \lambda_2 \\
			& & \ddots \\
			0 & & & \lambda_m
			\end{pmatrix}
		U^{-1}x
		\,.
\]
Gilt $y = U^{-1} x$, so ist
\begin{align*}
	M &= \Set{Uy \;\Big|\; y \in M_{m,1}(\RR)\colon \frac{1}{2} \sum_{\nu=1}^r \lambda_\nu y_\nu^2 = n} \\
	&= \text{Bild von U unter der kompakten Menge } \Set{y \in M_{m,1}(\RR) \;\Big|\; \sum_{\nu=1}^m \lambda_\nu y_\nu^2 = 2n}
	\,.
\end{align*}
Da $U$ stetig ist, ist damit $M$ kompakt.
Es folgt, dass $M \cap M_{m,1}(\ZZ)$ als diskrete Teilmenge einer kompakten Menge endlich ist.
\end{bewe}

\emph{Problem} Man gebe eine \emph{genaue} Darstellung für $r_Q(n)$ oder zumindest eine asymptotische Formel für $n \to \infty$ an.

\emph{Idee} Für $\tau\in\HH$ setzt man
\[
	\theta_Q(\tau)
	:= \sum_{x\in\ZZ^m} e^{2\pi i Q(x) \tau}
\]
eine sogenannte \myemph{Thetareihe}, d.\,h. $\theta_Q$ ist die erzeugte Fouriereihe von $n \mapsto r_Q(n)$.

Formal gilt dann
\[
	\theta_Q(\tau) = 1 + \sum_{n=1}^\infty r_Q(n)e^{2\pi i n\tau}
\]
d.\,h. $\theta_Q$ ist die von den $r_Q(n)$ erzeugte Fourierreihe.

\begin{lemm}\label{lemm:konvergenz_theta}
Die Reihe $\theta_Q(\tau)$ konvergiert gleichmäßig absolut auf kompakten Teilmengen $K \subset \HH$ und stellt dort eine holomorphe Funktion da.
\end{lemm}

\begin{bewe}
Sei $\tau = u + iv$ mit $v > 0$ und
\[
	S(\tau)
	:= \sum_{x\in\ZZ^m} \abs{e^{2\pi i Q(x)\tau}}
	= \sum_{x\in\ZZ^m} e^{-2\pi Q(x)v}
	\,.
\]
Sei $\norm\cdot$ die euklidische Norm auf $\RR^m$.
Die stetige Funktion $Q(x)$ mit $x\in M_{m,1}(\RR) \cong \RR^m$ nimmt auf dem Kompaktum $\Set{x \in \RR^m \mid \norm x = 1}$ ihr Maximum an.
Wegen $Q(x) > 0$ für alle $x \not= 0$ existiert also ein $c > 0$ mit
\[
	Q\left(\frac{x}{\norm x}\right)
	\geq c
	\qquad \text{für alle } x\in\RR^m,\ x\not=0
	\,,
\]
also
\[
	Q(x)
	\geq c \norm x ^2
	= c \cdot \sum_{\nu=1}^m x_\nu^2
	\qquad \text{für } x \in \RR^m
	\,.
\]
Daher folgt
\begin{align*}
	S(\tau)
	&\leq \sum_{x\in\ZZ^m} e^{-2\pi c(x_1^2 + \ldots + x_m^2)v}
	= \left(\sum_{x\in\ZZ} e^{-2\pi x^2v}\right)^m \\
	&\leq \left(1 + 2\sum_{\lambda=1}^\infty e^{-2\pi \lambda^2 v}\right)^m \\
	&\leq \left(1 + 2\sum_{\lambda=1}^\infty e^{-2\pi c \epsilon \lambda}\right)^m
	< \infty
	\,,
\end{align*}
für $v \geq \epsilon > 0$. Die letzte Reihe ist die geometriche Reihe mit $0 < q := e^{-2\pi c \epsilon} < 1$
Hieraus folgt die Behauptung.
\end{bewe}

\subsubsection*{Fundamentale Tatsachen}

Sei $N$ die Stufe von $A$ (bzw. $Q$), d.\,h. die kleinste natürliche Zahl, so dass $N \cdot A^{-1}$ ganzzahlig und gerade ist (z.\,B. hat $A = 2E_m$ die Stufe 4).
Sei $\Gamma_0(N) = \Set{(\begin{smallmatrix}a&b\\c&d\end{smallmatrix}) \in \SL_2(\ZZ) \mid N|c}$, dies ist offensichtlich eine Untergruppe von $\SL_2(\ZZ)$.
Es gelte aus Gründen der Einfachheit $4|m$.
Sei $M_{\frac{m}{2}}(\Gamma_0(N))$ der $\CC$-VR der Modulformen vom Gewicht $\frac{m}{2}$ bezüglich der Gruppe $\Gamma_0(N)$.
Dann gilt für $f\colon H \ra \CC$ genau dann  $f \in M_{\frac{m}{2}}(\Gamma_0(N))$ wenn $f$ holomorph ist mit
\[
	f\left(\frac{a\tau + b}{c\tau + d}\right) = \epsilon_m \begin{pmatrix}a&b\\c&d\end{pmatrix} (c\tau + d)^{\frac{m}{2}} f(\tau)
	\qquad \text{für alle } \begin{pmatrix}a&b\\c&d\end{pmatrix} \in \Gamma_0(N)
\]
und $f$ ist holomorph in den Spitzen von $\Gamma_0(N)''$.
Wobei $\epsilon_m (\begin{smallmatrix}a&b\\c&d\end{smallmatrix})$ eine gewisse vierte Einheitswurzel ist und $z^{\frac{m}{2}} = e^{-\frac{m}{2}\Log(z)}$ für $m$ ungerade ist.
Ist nun $m$ gerade, gilt $\epsilon_m (\begin{smallmatrix}a&b\\c&d\end{smallmatrix}) \in \{\pm1\}$.
Dann gilt $\theta_Q \in M_\frac{m}{2}(\Gamma_0(N))$.

\begin{beme-list}
\item Offenbar ist $M_{\frac{m}{2}}(\Gamma_0(N))$ ein $\CC$-VR.
\item Ist $m$ ungerade, hat man \emph{Modulformen halbganzen Gewichts}, die Theorie ist jedoch viel komplizierter.
\item Die Bedingung \emph{holomorph in den Spitzen} wird später erklärt.
Wendet man das Transformationsgesetzt auf $M = (\begin{smallmatrix}1&1\\0&1\end{smallmatrix}) \in \Gamma_0(N)$ an, so ergibt sich
\[
	f(\tau + 1) = f(\tau)
	\qquad \text{für } \tau\in\HH
\]
d.\,h. $f$ hat eine Fourierreihe
\[
	f(\tau) = \sum_{n\in\ZZ} a(n)e^{2\pi i \tau n} = \sum_{n\in\ZZ} a(n)q^n
	\qquad \text{für } q = e^{2\pi i \tau}
	\,.
\]
Die Bedingung holomorph in den Spitzen bedeutet unter Anderem, dass $a(n) = 0$ für alle $n < 0$, d.\,h.
\[
	f(\tau)
	= \sum_{n=0}^\infty a(n)q^n
	\,,
\]
hat in $q=0$ eine hebbare Singularitäten.
\end{beme-list}
\begin{bsp-list}
\item Sei $A = \mathrm{diag}(2,2,2,2)$, $m=4$.
Dann ist $\theta_Q \in M_2(\Gamma_0(4))$.
\item $M_{\frac{m}{2}}(\Gamma_0(N))$ ist endlich dimensionaler $\CC$-VR.
\item Für $m \geq 4$ hat man eine Zerlegung
\[
	M_{\frac{m}{2}}(\Gamma_0(N))
	= \E_{\frac{m}{2}}(\Gamma_0(N)) \oplus \Sp_{\frac{m}{2}}(\Gamma_0(N))
	\,,
\]
wobei $\E_{\frac{m}{2}}(\Gamma_0(N))$ der Raum der \emph{Eisensteinreihen} und $\Sp_{\frac{m}{2}}(\Gamma_0(N))$ der Raum der \emph{Spitzenformen} von Gewicht $\frac{m}{2}$ sind
(Bedingung für $\Sp_{\frac{m}{2}}$: $f$ verschwindet in den Spitzen, also insbesondere $a(0) = 0$).

Ist $m$ gerade, $m\geq 4$, besitzt $\E_{\frac{m}{2}}(\Gamma_0(N))$ eine \emph{ausgezeichnete Basis} von Funktionen, deren Fourierkoeffizienten sich durch elementare Teilerfunktionen ausdrücken lassen.
Beispiel: ist $k$ gerade, $k \geq 4$, so ist $\E_k(\SL_2(\ZZ)) = \CC\cdot E_k$ mit
\[
	E_k(\tau)
	= \frac{1}{2} \sum_{\ggt(c,d) = 1} \frac{1}{(c\tau + d)^k}
	\qquad \text{für } \tau\in\HH
	\,.
\]
Man zeigt
\[
	E_k(\tau) = 1 + c_k \sum_{n=1}^\infty \sigma_{k-1}(n)e^{2\pi \tau n}
	\,,
\]
wobei $c_k$ eine Konstante und $\sigma_{k-1}(n) := \sum_{d|n}d^{k-1}$.
Im Gegensatz hierzu kennt man die Fourierkoeffizieten der Spitzenformen nicht.
\end{bsp-list}

\emph{Anwendungen}
\begin{enumerate}
\item Ist nun $m$ und $N$ klein, so gilt manchmal
\[
	S_{\frac{m}{2}}(\Gamma_0(N)) = \{0\}
	\,.
\]
Dies führt zu genauen Formeln von $r_Q(n)$.
\begin{bsp}
Sei $m=4$, $Q(x) = x_1^2+x_2^2+x_3^2+x_4^2$.
Dann ist $\theta_Q \in M_2(\Gamma_0(4))$.
Man weiß: $\dim M_2(\Gamma_0(4)) = 2$ inbesondere $M_2(\Gamma_0(4)) = E_2(\Gamma_0(4))$,
\[
	E_2(\Gamma_0(4)) = \CC\cdot(P(\tau) - 4P(\psi\tau)) \oplus \CC\cdot(P(\tau) - 2P(2\tau))
\]
mit
\[
	P(\tau)
	= \frac{3}{\pi^2} \sum_{n\in\ZZ} \sum_{\substack{m\in\ZZ\\ \scriptscriptstyle m\not=0}} \frac{1}{(m+n\tau)^2}
	= 1 - 24 \sum_{n=1}^\infty \sigma_i(n) e^{2\pi i n\tau}
	\,.
\]
Es gilt daher
\[
	\theta_Q(\tau)
	= \alpha\cdot(P(\tau) - 4P(\psi\tau)) + \beta\cdot(P(\tau) - 2P(2\tau))
	\,.
\]
Ein Vergleich der ersten beiden Fourierkoeffizienten liefert
\[
	1
	= -3\alpha - \beta
	\qquad\text{und} \qquad
	8 = -24\alpha - 24\beta
	\,,
\]
also $\alpha = -\frac{1}{3}$ und $\beta = 0$.
Dies liefert dann
\[
	r_Q(n)
	= - \frac{1}{3} \left(-24\sigma_1(n) + 4\cdot24\cdot\sigma_1\left(\frac{n}{4}\right)\right)
	\,,
\]
d.\,h.
\[
	r_Q(n)
	= 8 \left(\sigma_1(n) - 4\sigma\left(\frac{n}{4}\right)\right)
	\,.
\]

Insbesondere gilt $r_Q(n) \geq 1$ für alle $n\in \NN$\footnote{Dies ist der Satz von Lagrange aus der Zahlentheorie.}
\end{bsp}

\item Für beliebiges gerades $m$ und $N$ erhält man nur asymptotische Formeln, denn man weiß, dass
\begin{equation}\label{eq:theta-asymp}
	a(n) = \mathcal O(n^{\frac{m}{4}})
\end{equation}
für alle
\[
	f
	= \sum_{n=1}^\infty a(n)q^n \in S_{\frac{m}{2}}(\Gamma_0(N))
	\,.
\]

Schreibe
\[
	\theta_Q
	= (\text{Eisensteinreihen}) + (\text{Spitzenformen})
	\,,
\]
Fourierkoeffizientenvergleich liefert
\[
	r_Q(n) = (\text{elementare Teilerfunktion der Größenordnung } n^{\frac{m}{2}-1}) + \mathcal O (n^{\frac{m}{4}})
\]
wenn $n \to \infty$. Die Abschätzung \eqref{eq:theta-asymp} kann noch beachtlich verbessert werden.
\end{enumerate}
\section{Motivation}

\emph{Grob gesagt} sind Modulformen auf der oberen Halbebene $\HH = \Set{z \in \CC \mid \Im(z) > 0}$ holomorphe Funktionen, die gewisse einfache Transformationseigenschaften unter diskreten Untergrupen von gebrochen linearen Transformationen haben.

Dies wollen wir nun genauer formulieren.
Sei $\SL_2(\RR) = \Set{M \in M_{2,2}(\RR) \mid \det(M) = 1}$.
Dann ist $\SL_2(\RR)$ eine Gruppe unter der gewöhnlichen Matrizenmultiplikation.
Jedes $M = \left(\begin{smallmatrix}a & b \\ c & d\end{smallmatrix}\right) \in \SL_2(\RR)$ definiert eine sogenannte gebrochen lineare Transformation von $\HH$ in sich selbst\footnote{Diese wird auch \myemph{Möbiustransformation} genannt.}
\[
	z \mapsto \frac{az + b}{cz + d}
	\,,
\]
\emph{denn} sei $z\in\HH$, d.\,h. $z=x+iy$ mit $y > 0$, dann gilt $cz+d \not= 0$ und
\begin{align*}
	\Im\left(\frac{az+b}{cz+d}\right)
	&= \Im\left( \frac{(az+b)\conj{(cz+d)}}{\abs{cz+d}^2}\right)
	= \Im\left( \frac{ac\abs z^2 + bd + adz + bc\conj z}{\abs{cz+d}^2} \right) \\
	&= \frac{(ad - bc)y}{\abs{cz+d}^2}
	= \frac{y}{\abs{cz+d}^2}
	> 0
	\,.
\end{align*}

Eine \emph{Modulform} ist dann eine holomorphe Funktion $f\colon \HH \ra \CC$, die der Transformationseigenschaft
\[
	f\left(\frac{az+b}{cz+d}\right) = (cz+d)^k f(z)
\]
für alle $M = \left(\begin{smallmatrix}a & b \\ c & d\end{smallmatrix}\right)$ einer diskreten Untergruppe der $\SL_2(\RR)$ genügt mit einem $k\in\ZZ$ und gewissen sonstigen Eigenschaften genügt.
Hierbei wird die Zahl $k$ als das \myemph[Modulform!Gewicht]{Gewicht} der Modulform $f$ bezeichnet.

Über die Theorie der Modulformen hat die Funktionentheorie Anwendungen auf zahlentheoretische Probleme.

\subsection{Beispiel aus der Theorie der quadratischen Formen}

Sei $m\in\NN$ und $A = ( a_{\mu\nu})_{1\leq\mu,\nu\leq m} \in M_{m,m}(\ZZ)$.

\emph{Voraussetzungen}: $A$ ist gerade, d.\,h. die $a_{\mu\mu}$ sind für alle $\mu = 1$, \ldots, $m$ gerade, und $A$ ist symmetrisch, d.\,h. $a_{\mu\nu} = a_{\nu\mu}$ für alle $\mu$, $\nu = 1$, \ldots, $m$.
Zum Beispiel
\[
	A =
	\begin{pmatrix}
	2 & & & 0 \\
	& 2 \\
	& & \ddots \\
	0 & & & 2
	\end{pmatrix}
	\,.
\]
Für $x = \left(\begin{smallmatrix}x_1 \\ \vdots \\ x_m\end{smallmatrix}\right) \in M_{m,1}(\RR)$ sei
\[
	Q(x)
	= \frac{1}{2} x^t A x
	\,.
\]

Es gilt somit
\begin{align*}
	Q(x)
	&= \frac{1}{2} (x_1e_1^t + \ldots + x_me_m^t) A (x_1e_1 + \ldots + x_me_m) \\
	&= \frac{1}{2} \sum_{1\leq\mu,\nu\leq m} x_\mu x_\nu e_\mu^t A e_\nu
	= \frac{1}{2} \sum_{1\leq\mu,\nu\leq m} a_{\mu\nu}x_\mu x_\nu \\
	&= \sum_{1\leq \mu < \nu \leq m} a_{\mu\nu}x_\mu x_\nu + \sum_{\mu=1}^m \frac{a_{\mu\mu}}{2} x_\mu^2
	\,.
\end{align*}
für $e_\nu$ der $\nu$-te Standard-Einheitsnvektor.
Die letzte Gleichheit gilt wegen $a_{\mu\nu} = a_{\nu\mu}$.
Also ist $Q(x)$ eine ganzzahlige (man beachte, dass $a_{\mu\mu}$ gerade ist) quadratische Form (d.\,h. ein homogenes Polynom vom Grad 2) in den Variablen $x_1$, \ldots, $x_m$.

Wir wollen nun zusätzlich Voraussetzen, dass $Q$ positiv definit ist, d.\,h. $Q(x) > 0$ für alle $x \in M_{m,1}(\RR)$ mit $x\not=0$.

Für $n\in\NN$ setzen wir nun
\[
	r_Q(n)
	:= \# \Set{x \in M_{m,1}(\ZZ) \mid Q(x) = n}
\]
die Anzahl der Darstellungen von $n$ durch $Q$.
\begin{bsp}
Sei
\[
	A =
	\begin{pmatrix}
	2 & & & 0 \\
	& 2 \\
	& & \ddots \\
	0 & & & 2
	\end{pmatrix}
	\,.
\]
Dann ist
\[
	Q(x) = x_1^2 + \ldots + x_m^2
	\,,
\]
also ist in diesem Fall $r_Q(n)$ die Anzahl der Darstellungen von $n$ als Summe von $m$ Quadraten.
\end{bsp}

\begin{lemm}
Für eine solche Quadratische Form $Q$ gilt dann $r_Q(n) < \infty$.
\end{lemm}

\begin{bewe}
Aus der Linearen Algebra wissen wir, dass sich $A$ durch eine orthogonale Matrix diagonalisieren lässt, d.\,h. es gibt ein orthogonales $U \in \GL_m(\RR)$, d.\,h. $U^tU = E$, so dass
\[
	U^tAU =
	\begin{pmatrix}
	\lambda_1 & & & 0 \\
	& \lambda_2 \\
	& & \ddots \\
	0 & & & \lambda_m
	\end{pmatrix}
	\,,
\]
mit $\lambda_\nu \in \RR$ für alle $\nu=1$, \ldots, $m$\footnote{Dies sind gerade die Eigenwerte von $A$}.
Da $Q$ positiv definit ist, gilt
\[
	\lambda_\nu = 
	e_\nu^t 	
			\begin{pmatrix}
			\lambda_1 & & & 0 \\
			& \lambda_2 \\
			& & \ddots \\
			0 & & & \lambda_m
			\end{pmatrix}
		e_\nu
	= (Ue_\nu)^tA(Ue_\nu)
	> 0
\]
für alle $1 \leq \nu \leq m$. Sei
\[
	M
	:= \Set{x\in M_{m,1}(\RR) \mid Q(x) = n}
	\,.
\]
Es ist
\[
	Q(x)
	= \frac{1}{2}x^tAx
	= \frac{1}{2} x^t(U ^t)^{-1}
			\begin{pmatrix}
			\lambda_1 & & & 0 \\
			& \lambda_2 \\
			& & \ddots \\
			0 & & & \lambda_m
			\end{pmatrix}
		U^{-1}x
		\,.
\]
Gilt $y = U^{-1} x$, so ist
\begin{align*}
	M &= \Set{Uy \;\Big|\; y \in M_{m,1}(\RR)\colon \frac{1}{2} \sum_{\nu=1}^r \lambda_\nu y_\nu^2 = n} \\
	&= \text{Bild von U unter der kompakten Menge } \Set{y \in M_{m,1}(\RR) \;\Big|\; \sum_{\nu=1}^m \lambda_\nu y_\nu^2 = 2n}
	\,.
\end{align*}
Da $U$ stetig ist, ist damit $M$ kompakt.
Es folgt, dass $M \cap M_{m,1}(\ZZ)$ als diskrete Teilmenge einer kompakten Menge endlich ist.
\end{bewe}

\emph{Problem} Man gebe eine \emph{genaue} Darstellung für $r_Q(n)$ oder zumindest eine asymptotische Formel für $n \to \infty$ an.

\emph{Idee} Für $\tau\in\HH$ setzt man
\[
	\theta_Q(\tau)
	:= \sum_{x\in\ZZ^m} e^{2\pi i Q(x) \tau}
	\,.
\]
Dies ist eine sogenannte \myemph{Thetareihe}, d.\,h. $\theta_Q$ ist die erzeugte Fouriereihe von $n \mapsto r_Q(n)$.
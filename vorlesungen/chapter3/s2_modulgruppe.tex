\section{Modulgruppe und Fundamentalbereich}

\begin{defi}
Sei $(G, \circ, e)$ eine Gruppe und $S$ eine nicht-leere Menge. Man sagt, dass $G$ auf $S$ (von links) \myemph[Gruppenoperation]{operiert}, wenn eine Abbildung $G \times S \ra S$, $(g, s) \mapsto g \circ s$ gegeben ist, so dass
\[
	(g_1 \circ (g_2 \circ s)) = (g_1 \circ g_2) \circ s
\]
und
\[
	e \circ s = s
\]
für alle $g_1$, $g_2 \in G$ und $s \in S$.

Die Gruppe $G_s = \Set{g \in G \mid g \circ s = s}$ heißt \myemph{Stabilisator} von $s$.
Die Menge
\[
	Gs = \Set{g \circ s \mid g\in G}
\]
heißt \myemph{Orbit} von $s$ in $G$.
\end{defi}

Operiert $G$ auf $S$, so definiert dies eine Äquivalenzrelation auf $S$ durch
\[
	s \sim s'
	\Rla \exists g \in G\colon s' = g \circ s
	\,.
\]
Die Äquivalenzklasse von $s \in S$ ist der Orbit von $s$ unter $G$.
Man hat daher eine disjunkte Vereinigung
\[
	S = \mathop{\stackrel{\bullet}{\bigcup}}_{j\in I} Gs_j
\]
wobei $Gs_j$ die verschiedenen Orbits durchläuft.

\begin{satz-list}
\item Die Gruppe $\SL_2(\RR)$ operiert auf $\HH$ via
\[
		\begin{pmatrix}
			a & b \\
			c & d
		\end{pmatrix}
		\circ z
	:= \frac{az+b}{cz+d}
	\,.
\]

\item Sei $\Aut(\HH) := \Set{f\colon\HH \ra \HH \mid f \text{ biholomorph}}$ die \myemph{Automorphismengruppe von $\HH$}.
Dann ist die Abbildung
\begin{align*}
	\modulo{\SL_2(\RR)}{\Set{\pm E}} &\ra \Aut(\HH)\\
		\overline{\begin{pmatrix}
			a & b \\
			c & d
		\end{pmatrix}}
		&\mapsto
		\left( z \mapsto \frac{az+b}{cz+d} \right)
\end{align*}
ein Isomorphismus, wobei $E$ die Einheitsmatrix ist.
\end{satz-list}

\begin{bewe-list}
\item Dass mit $z$ auch $\frac{az+b}{cz+d}$ wieder in $\HH$ liegt, wurde schon im vorherigen Abschnitt gezeigt (siehe \eqref{eq:moebiusnachh} auf Seite \pageref{eq:moebiusnachh}).
Offenbar ist $(\begin{smallmatrix} 1 & 0\\0&1\end{smallmatrix}) \circ z = z$ und
\begin{align*}
		\begin{pmatrix}
			a_1 & b_1 \\
			c_1 & d_1
		\end{pmatrix}
		\circ
		\left(\begin{pmatrix}
			a_2 & b_2 \\
			c_2 & d_2
		\end{pmatrix}
		\circ z\right)
	&=
		\begin{pmatrix}
			a_1 & b_1 \\
			c_1 & d_1
		\end{pmatrix}
		\circ
		\frac{a_2z+b_2}{c_2z+d_2} \\
	&= \frac{a_1(a_2z+b_2)+b_1(c_2z+d_2)}{c_1(a_2z+b_2)+d_1(c_2z+d)} \\
	&= \frac{(a_1a_2+b_1c_2)z+(a_1b_2+b_1d_2)}{(c_1a_2+d_1c_2)z + (c_1b_2+d_1d_2)} \\
	&=
		\left(\begin{pmatrix}
			a_1 & b_1 \\
			c_1 & d_1
		\end{pmatrix}
		\circ
		\begin{pmatrix}
			a_2 & b_2 \\
			c_2 & d_2
		\end{pmatrix}\right)
		\circ z
\end{align*}
also operiert $\SL_2(\RR)$ auf $\HH$ in angegebener Weise.

\item Sei $(\begin{smallmatrix}a&b\\c&d\end{smallmatrix}) \in \SL_2(\RR)$.
Dann ist offenbar $z \mapsto \frac{az+b}{cz+d}$ eine biholomorphe Abbildung von $\HH$ in sich selbst.\footnote{die Umkehrabbildung ist durch $w \mapsto (\begin{smallmatrix}a&b\\c&d\end{smallmatrix})^{-1} \circ w$ gegeben.}
Nach (i) ist
\[
	\phi\colon \SL_2(\RR) \ra \Aut(\HH),
		\begin{pmatrix}
			a & b \\
			c & d
		\end{pmatrix}
		\mapsto
		\left( z \mapsto \frac{az+b}{cz+d} \right)
\]
ein Gruppenhomorphismus.

Wir zeigen als nächstes, dass $\phi$ surjektiv ist, dafür betrachten wir die zunächst die folgenden beiden Aussagen
\begin{enumerate}
\item Die Operation von $\SL_2(\RR)$ auf $\HH$ ist transitiv, d.\,h. zu gegebenem $z_1$, $z_2 \in \HH$ gibt es stets ein $M\in\SL_2(\RR)$ mit $M \circ z_1 = z_2$.

\begin{bewe}
Es genügt zu zeigen: zu gegebenem $z\in\HH$ gibt es $M\in \SL_2(\RR)$ mit
\[
	M \circ i = z
	\,.
\]
Sei $z=x+iy$ und
\[
	M
	:= \begin{pmatrix}
			y^{\frac{1}{2}} & xy^{-\frac{1}{2}} \\
			0 & y^{\frac{1}{2}}
		\end{pmatrix}
	\in \SL_2(\RR)
	\,.
\]
Dann gilt
\[
	M \circ i
	= \frac{y^{\frac{1}{2}}i + xy^{-\frac{1}{2}}}{y^{-\frac{1}{2}}}
	= yi+x
	= z
	\,.
\]
Dies zeigt (a).
\end{bewe}
\item Ist $f \in \Aut(\HH)$ und $f(i) = i$, so existiert $(\begin{smallmatrix}a&b\\c&d\end{smallmatrix}) \in \SL_2(\RR)$ mit
\[
	f(z)
	= \frac{az+b}{cz+d}
	\,.
\]

\begin{bewe}
Für $z\in\HH$ sei $g(z) := \frac{z-i}{z+i}$.
Dann ist $g$ eine biholomorphe Abbildung von $\HH$ auf dem Einheitskreis $\EE := \Set{w \in \CC \mid \abs w < 1}$ mit $g(i) = 0$.
Sei $F := g \circ f \circ g^{-1}$.
Dann ist $F \in \Aut(\EE)$ und $F(0) = 0$, denn $f(i) = i$.
Nach dem Schwarzschen Lemma (Funktionentheorie 1), muss gelten
\[
	\abs{F(w)} \leq \abs w,
	\quad \abs{F^{-1}(w)} \leq \abs w
	\qquad \text{für } w\in\EE
	\,,
\]
also
\[
	\abs{F(w)} \leq \abs w \leq \abs{F(w)}
	\,,
\]
das heißt
\[
	\abs{F(w)} = \abs{w}
	\qquad \text{für alle } w \in \EE
	\,.
\]
Nach dem Schwarzschen Lemma folgt jetzt
\[
	F(w) = e^{i\theta}w
	\qquad \text{für } w \in \EE
	\,,
\]
wobei $\theta \in \RR$ fest ist.
Es folgt
\[
	f(z)
	= (g^{-1} \circ F \circ g)(z)
	= g^{-1}((F \circ g)(z))
	= g^{-1} \circ (e^{i\theta} \cdot \frac{z-i}{z+i})
	\,.
\]
Wie man leicht sieht, ist
\[
	g^{-1}(w)
	= \frac{-wi-i}{w-1}
	\,.
\]
Also folgt
\begin{align*}
	f(z)
	&= \frac{-ie^{i\theta}\frac{z-i}{z+i} - i}{e^{i\theta}\frac{z-i}{z+i}-1}
	= \frac{i(e^{i\theta}+1)z - e^{i\theta} + 1}{(e^{i\theta}-1)z-i(e^{i\theta}+1)} \\
	&= \frac{\cos(\frac{\theta}{2})z + \sin(\frac{\theta}{2})}{-\sin(\frac{\theta}{2})z + \cos(\frac{\theta}{2})}
	= \underbrace{\begin{pmatrix}
			\cos(\frac{\theta}{2}) & \sin(\frac{\theta}{2}) \\
			- \sin(\frac{\theta}{2}) & \cos(\frac{\theta}{2})
		\end{pmatrix}}_{\in\SL_2(\RR)}
		\circ z
		\,.
\end{align*}
Dies zeigt (b).
\end{bewe}
\end{enumerate}

Sei nun $f\in\Aut(\HH)$.
Nach (a) existiert $M\in\SL_2(\RR)$ mit $M\circ i = f(i)$.
Die biholomorphe Abbildung $z\mapsto f^{-1}(M\circ z)$ lässt dann $i$ fest.
Nach (b) existiert also $M_i \in \SL_2(\RR)$, so dass
\[
	f^{-1}(M \circ z) = M_i \circ z
	\,.
\]
Ersetzt man nun $z$ durch $M_i^{-1} \circ z$, so folgt
\[
	f^{-1}(M \circ (M_i^{-1} \circ z))
	= z
	\,,
\]
also
\[
	MM_i^{-1}z = f(z)
	\qquad \text{für alle } z \in \HH
	\,.
\]
Dies zeigt die Surjektivität von $\phi$.

Offenbar ist $\pm (\begin{smallmatrix} 1 & 0\\ 0 & 1\end{smallmatrix}) \in \ker \phi$.
Sei $M = (\begin{smallmatrix} a & b\\ c & d\end{smallmatrix}) \in \ker \phi$, also $\frac{az+b}{cz+d} = z$.
Daraus folgt $(\begin{smallmatrix} a & b\\ c & d\end{smallmatrix}) = \pm (\begin{smallmatrix} 1 & 0\\ 0 & 1\end{smallmatrix})$.
Die Behauptung folgt jetzt mit dem Homomorphiesatz.
\end{bewe-list}

Wir interessieren uns im folgenden für diskrete Untergruppen $\Gamma \subset \SL_2(\RR)$ und deren Operation auf $\HH$.
Besonders wichtig sind $\Gamma = \SL_2(\ZZ)$ oder $\Gamma$ eine Untergruppe von endlichem Index von $\SL_2(\ZZ)$.

\begin{defi}
Die Gruppe
\[
	\SL_2(\ZZ) = \Set{ \begin{pmatrix} a & b \\ c& d\end{pmatrix} \in M_{2,2}(\ZZ) \mid ad-bc = 1}
\]
heißt \myemph[volle Modulgruppe]{(volle) Modulgruppe}.

Spezielle Matrizen in $\SL_2(\ZZ)$ sind $T = (\begin{smallmatrix} 1 & 1 \\ 0 & 1\end{smallmatrix})$, operiert durch $z \mapsto z + 1$ (\emph{Translation}), und $S = (\begin{smallmatrix} 0 & -1 \\ 1 & 0\end{smallmatrix})$, operiert durch $z \mapsto -\frac{1}{z}$ (\emph{Stürzung}).
\end{defi}

\begin{beme}
Der Begriff \emph{Modulgruppe} (oder auch \emph{Modulform}) hat nichts mit Moduln im Sinne der Algebra zu tun, sondern stammt aus der algebraischen Geometrie:
ein \emph{Modul} ist (klassich) eine Größe, die man einer gegebenen algebraischen Funktion $f$ (etwa $f(x) = \sqrt{4x^3 - g_2x-g_3}$, elliptische Kurve) zuordnen kann und die sich nicht ändert, wenn man $f$ gewissen Transformationen unterwirft (im obigen Beispiel ist ein Modul von $f$ etwa $j := \frac{g_2^3}{g_2^3-27g_3^2}$, parametrisiert man $g_2$ und $g_3$ durch $\tau\in\HH$, so ist $j(\frac{a\tau+b}{c\tau+d}) = j(\tau)$ für alle $(\begin{smallmatrix}a&b\\c&d\end{smallmatrix}) \in \SL_2(\ZZ)$).
\end{beme}

\emph{Ziel}: Aus einem Orbit $\Set{M \circ z \mid M \in \Gamma}$ wähle man einen geeigneten Repräsentanten aus.
Die Menge dieser Repräsentanten soll einfach beschrieben werden können und schöne geometrische Eigenschaften haben (z.\,B. zusammenhängend als topologischer Raum, meßbar, \ldots).

\begin{defi}
$\F \subset \HH$ heißt \myemph{Fundamentalbereich} für die Operation von $\Gamma \subset \SL_2(\ZZ)$ auf $\HH$, falls gilt
\begin{enumerate}
\item $\F$ ist offen,
\item zu jedem $z\in \HH$ gibt es $M \in \Gamma$ mit $M \circ z \in \closure\F$,
\item sind $z_1$, $z_2 \in \F$ und $z_2 = M \circ z_1$ mit $M \in \Gamma$, dann ist $M = \pm E$, also $z_1 = z_2$,
\item $\#\Set{M \in \Gamma \mid M \circ \closure\F \cap \closure\F \not= \emptyset} < \infty$.\footnote{Diese Eigenschaft ergibt sich bereits aus (i)--(iii)}
\end{enumerate}
\end{defi}

\begin{satz}
Die Menge $\F := \Set{z = x+iy \in \HH \mid \abs z >1, \abs x < \frac{1}{2}}$ ist ein Fundamentalbereich für die Operationen von Gamma $\Gamma(1) = \SL_2(\ZZ)$ auf $\HH$.
\end{satz}

\begin{figure}
\begin{center}
	\includestandalone{vorlesungen/chapter3/images/fundamentalbereich}
	\caption{Der Fundamentalbereich der vollen Modulgruppe.}
	\label{fig:fundamentalbereich}
\end{center}
\end{figure}

\begin{bewe-list}
\item Klar!
\item Jedem $z\in\HH$ ordne man seine \emph{Höhe} $h(z) := \Im(z)$ zu.
Dann gilt
\[
    h(M \circ z) = \frac{h(z)}{\abs{cz+d}^2}
\]
für $M = (\begin{smallmatrix}a & b\\c & d\end{smallmatrix})\in \Gamma(1)$ (siehe \eqref{eq:moebiusnachh} auf Seite \pageref{eq:moebiusnachh}).

\begin{lemm-ind}
Jeder Orbit $\Gamma(1) \circ z$ enthält Punkte $w$ maximaler Höhe.
Diese sind charakterisiert durch die Eigenschaft $\abs{cw + d} \geq 1$ für alle $(c,d) \in \ZZ^2$ mit $\ggt(c,d) = 1$.
\end{lemm-ind}
\begin{bewe-ind}
Sei $z = x+iy \in \HH$ fest. Dann hat die Ungleichung $\abs{cz + d} \leq 1$ nur endlich viele Lösungen $(c, d) \in \ZZ^2$ (Schnitt einer kompakten Menge $\abs z \leq 1$ und einer diskreten Menge $\ZZ z \oplus \ZZ$ (Gitter). Man hat
\[
    h(M \circ z) \geq h(z)
    \Rla \abs{cz+d} \leq 1
    \qquad \text{für } M = (\begin{smallmatrix}a & b\\c & d\end{smallmatrix})
    \,.
\]
Da die Höhe $h(M \circ z)$ nur von der zweiten Zeile von $M$ abhängt, folgt, dass es nur endlich viele verschiedene Werte $h(M \circ z)$ mit $h(M \circ z) \geq h(z)$ gibt, wobei $M \in \Gamma(1)$.
Also gibt es einen Punkt maximaler Höhe.

Sei $w\in \Gamma(1) \circ z$.
Dann ist $h(w)$ maximal, genau dann wenn $h(w) \geq h(M \circ z)$ für alle $M \in \Gamma(1)$.
Schreibe $z = M_0 \circ w$.
Da mit $M$ auch $MM_0$ alle Elemente von $\Gamma(1)$ durchläuft, gilt

\begin{align*}
    && &h(w) \geq h(M \circ z) &&\text{für alle } M = \abcd \in \Gamma(1) \\
    &\Rla& &h(w) \geq h(M \circ w) &&\text{für alle } M = \abcd \in \Gamma(1) \\
    &\Rla& &\abs{cw+d} \geq 1 &&\text{für alle } M = \abcd \in \Gamma(1) \\
    &\Rla& &\abs{cw+d} \geq 1 &&\text{für alle } (c, d) \in \ZZ^2 \text{ mit } \ggt(c,d) = 1\,,
\end{align*}
denn ist umgekehrt $\ggt(c,d) = 1$, so gibt es $a, b \in \ZZ$ mit $ad - bc = 1$, denn $\ZZ$ ist ein Hauptidealring, also ist $\abcd \in \Gamma(1)$.
\end{bewe-ind}

Die Höhe ist invariant unter Translation $(\begin{smallmatrix} 1 & b \\ 0 & 1\end{smallmatrix})$ mit $b\in\ZZ$.
Man kann daher $b$ so wählen, dass $\abs{\Re(z+b)} \leq \frac{1}{2}$.
Daher gilt:

\begin{lemm-ind}
Sei
\[
    \F' := \Set{w = x+iy \in \HH \mid \abs x \leq {\textstyle\frac{1}{2}},\ \abs{cw+d} \geq 1\ \forall (c,d) \in \ZZ^2 \text{ mit } \ggt(c,d) = 1}
    \,.
\]
Dann existiert zu jedem $z\in\HH$ ein $M \in \Gamma(1)$, so dass $M \circ z \in \F'$.
\end{lemm-ind}

\begin{lemm-ind}\label{lemm:fundamental_gamma1_lemm3}
Es gilt $\F' = \closure \F = \Set{z = x+iy \in \HH \mid \abs z \geq 1, \abs x \leq \frac{1}{2}}$.
\end{lemm-ind}

\begin{bewe-ind}
$\F' \subset \closure \F$ klar, denn $\ggt(1,0) = 1$.

Umgekehrt: Sei $z\in \closure\F$ und $(c,d) \in \ZZ^2$ mit $\ggt(c,d) = 1$.
Dann
\begin{align*}
    \abs{cz+d}^2
    &= \abs{cx + d + icy}^2 = (cx+d)^2 + c^2y^2 \\
    &= c^2(x^2+y^2) + 2cdx + d^2
    \geq c^2 + 2cdx + d^2 \\
    &\geq c^2 - \abs{cd} + d^2
    \geq 1
    \,,
\end{align*}
wobei die letzte Ungleichung gilt, weil die quadratischen Formen $x^2 \pm xy + y^2$ positiv definit sind.
Also $z \in \F'$. Also $\F' = \closure\F$.
\end{bewe-ind}

\item[(iii)+(iv)] Die Aussagen folgen aus folgendem Lemma

\begin{lemm-list}\label{lemm:fundamental_gamma1_lemm4}
\item Seien $z, z' \in \closure \F$ mit $z' = M \circ z$ mit $M \in \Gamma(1)$.
Sei $z' \not= z$.
Dann ist entweder $x = \pm \frac{1}{2}$ (und dann $x' = \mp \frac{1}{2}$, $z' = z - 1$ und $M = \pm T^{\pm 1}$) oder $\abs z = 1$, $z' = -\frac{1}{z}$, $M = \pm S$.
\item Sei $z \in \closure\F$. Dann gilt für den Stabilisator von $z$
\begin{equation}\label{eq:fundamental_gamma1}
    \Gamma(1)_z
    =
    \begin{cases}
        \erzeug{ST} & \text{falls } z = \rho \\
        \erzeug{TS} & \text{falls } z = -\conj\rho \\
        \erzeug{S} & \text{falls } z = i \\
        \Set{\pm E} & \text{sonst}
    \end{cases}
    \,.
\end{equation}
Mit $\rho = e^{\frac{2}{3}\pi i} = \frac{-1+i\sqrt{3}}{2}$. Die Gruppen rechts sind endlich.
\end{lemm-list}

\begin{bewe-ind}
Zunächst ist
\[
    ST
    = \begin{pmatrix} 1 & 1 \\ 0 & 1\end{pmatrix} \begin{pmatrix} 0 & -1 \\ 1 & 0\end{pmatrix}
    = \begin{pmatrix} 1 & -1 \\ 1 & 0\end{pmatrix}
    \,.
\]
Damit ergibt sich
\[
    (ST)^3 = (TS)^3 = S^2 = -E
    \,.
\]
Deswegen sind die rechts in \eqref{eq:fundamental_gamma1} stehenden Untergruppen endlich und zyklisch, der Ordnung 6 beziehungsweise 4.
Auch in diesen enthalten ist $-E$.

Die rechts in \eqref{eq:fundamental_gamma1} stehende UG fixieren die entsprechenden $z$ tatsächlich:
\begin{enumerate}
\item $z=i$: $S \circ i = - \frac{1}{i} = i$ für $z=i$.
\item $z = \rho$: $\rho^3 = 1$, $(\rho - 1)(1 + \rho + \rho^2) = 0$.
Da $\rho \not= 1$ folgt $1 + \rho + \rho^2 = 0$.
Daher $ST \circ \rho = S \circ (\rho + 1) = - \frac{1}{\rho + 1} = - \frac{1}{-\rho^2} = \frac{1}{\rho^2} = \rho$.
Also $ST \circ \rho = \rho$.
\item $z = - \conj\rho$: Genauso wie mit $z = \rho$.
\end{enumerate}

Seien $z$, $z' \in \closure \F$, $z' = M \circ z$ für $M = \abcd \in \Gamma(1)$.
Da $\closure\F = \F'$ (\autoref{lemm:fundamental_gamma1_lemm3}), folgt $z'$ und $z$ sind in selben Orbit und Punkte maximaler Höhe, also $h(z') = h(M \circ z) = h(z)$, also $\abs{cz+d} = 1$.
Also
\[
    1 = \abs{cz+d}^2 \geq \abs{c}^2 - \abs{cd} + \abs{d}^2 \geq \left(\abs{d} - \frac{\abs{c}}{2}\right)^2 + \frac{3}{4}\abs{c}^2 \geq 1.
\]
Also muss überall Gleichheit gelten, insbesondere
\[
    \left(\abs{d} - \frac{\abs c}{2}\right)^2 + \frac{3}{4}\abs c ^2 = 1
    \,.
\]

\emph{1. Fall}: $c=0$, $d=\pm 1$

Dann
\[
    M
    = \begin{pmatrix}\pm1 & b \\ 0 & \pm 1\end{pmatrix}
    \qquad \text{mit } b \in \ZZ
    \,.
\]
Also gilt
\[
    M
    = \pm\begin{pmatrix}1 & b \\ 0 & 1\end{pmatrix}
    \qquad \text{mit } b \in \ZZ
    \,.
\]
Damit gilt $z' = M \circ z = z + b$.
Da $-\frac{1}{2} \leq x, x' \leq \frac{1}{2}$, folgt entweder $b=0$ (also $M=\pm E$, $z'=z$) oder $b=1$ (dann $x=-\frac{1}{2}$, $x'=\frac{1}{2}$, $M = \pm(\begin{smallmatrix}1 & 1 \\ 0 & 1\end{smallmatrix}) =  \pm T$) oder $b = -1$ (dann $x=\frac{1}{2}$, $x'=-\frac{1}{2}$, $M = \pm T^{-1}$)

\emph{2. Fall}: $c=\pm 1$, $d=0$

Dann
\[
    M
    = \begin{pmatrix}a & \mp 1 \\ \pm 1 & 0\end{pmatrix}
    \qquad \text{mit } b \in \ZZ
    \,.
\]
Also
\[
    M
    = \begin{pmatrix}a & - 1 \\ 1 & 0\end{pmatrix}
    \qquad \text{mit } a \in \ZZ
    \,.
\]
Dann
\[
    z' = M \circ z = a - \frac{1}{z}
    \,.
\]

\emph{Beobachtung}: $\abs z = 1$. Dann gilt auch $\abs{- \frac{1}{z}} = 1$ und $-\frac{1}{z} \in \closure\F$.
Es folgt wie oben einer der Fälle
\begin{enumerate}
\item $a=0$

$M = (\begin{smallmatrix}0 & -1 \\ 1 & 0\end{smallmatrix}) = S$, $z' = -\frac{1}{z} = \pm S \circ z$.
Ist $z' = z$ so folgt $z=i$.


\item $a=1$

$M = \pm (\begin{smallmatrix}1 & -1\\1&0\end{smallmatrix}) = \pm TS$.
Dann
\[
    -\frac{1}{2} = \Re\left(-\frac{1}{z}\right) = \Re\left(\frac{-\conj z}{\abs z ^2}\right) = -x
    \,.
\]
Also $x=\frac{1}{2}$, wegen $x^2+y^2 = 1$ folgt $y = \frac{\sqrt{3}}{2}$, d.\,h. $z = \frac{1+i\sqrt{3}}{2} = - \conj\rho$. Also $z' = -\frac{1}{z} + 1 = 1 + \frac{1}{\conj\rho} = 1 + \rho = -\conj\rho$.


\item $a=-1$

Es folgt wie oben $z' = z = \rho$, $M = \pm(ST)^2$.
\end{enumerate}

\emph{3. Fall}: $c=\pm 1$, $d=\pm 1$

Man zeigt $z' = z = \rho$ und $M = \pm ST$ oder $z' = z = -\conj\rho$ und $M = \pm (TS)^2$.
\end{bewe-ind}
\end{bewe-list}

\begin{koro}
Die Gruppe $\SL_2(\ZZ)$ wird erzeugt von $T = (\begin{smallmatrix}1 & 1\\0 & 1\end{smallmatrix})$ und $S = (\begin{smallmatrix}0 & -1 \\ 1 & 0\end{smallmatrix})$.
\end{koro}

\begin{bewe}
Man zeigt zunächst: Sei $\Gamma(1)'$ die von $S$ und $T$ erzeugte Untergruppe.
Sei $z \in \HH$.
Dann existiert $M' \in \Gamma(1)'$ mit $M' \circ z \in \closure\F$.
\emph{Denn}: Man zeigt wie vorher: Der Orbit $\Gamma(1)' \circ z$ hat Elemente maximaler Höhe.
Sei $M' \in \Gamma'(1)$ mit Höhe $h(M' \circ z)$ maximal.
Sei $n \in \ZZ$ mit
\[
    \abs*{\Re(\underbrace{T^nM'\circ z)}_{=: z'}} \leq \frac{1}{2}
\]
\emph{Behauptung}: $z' \in \closure\F$.
Andernfalls $\abs{z'} < 1$.
Aber
\[
    h(S \circ z') = \Im\left(-\frac{1}{z'}\right) = \frac{\Im(z')}{\abs{z'}^2}
    > \Im(z') = h(z') = h(M' \circ z)
\]
ein Widerspruch zur Maximalität von $h(z')$.

Sei $M \in \Gamma(1)$.

Zu zeigen: $M \in \Gamma(1)'$.
Wähle $z_0 \in \F$ fest, z.\,B. $z_0 = 2i$.
Zu $z = M \circ z_0$ existiert ein $M' \in \Gamma(1)'$ mit $M' \circ z = M'M \circ z_0 \in \closure\F$, wie gerade gezeigt.
Nach \autoref{lemm:fundamental_gamma1_lemm4} folgt $M'M = \pm E$. Also $M' = \pm M^{-1} \in <S, T, -E>= <S, T>$ wegen $S^2 = -E$.
\end{bewe}
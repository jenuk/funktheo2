\section{Definition von Modulfunktionen und Modulformen}

\begin{defi}\label{def:modulfunktion}
Sei $k\in\ZZ$.
Eine Funktion $f\colon \HH \ra \closure\CC$ heißt \myemph{Modulfunktion} vom \emph{Gewicht} $k$, falls gilt
\begin{enumerate}
\item $f$ ist meromorph
\item $f(\frac{az+b}{cz+d}) = (cz+d)^kf(z)$ für alle $\abcd \in \Gamma(1)$
\item $f$ ist meromorph in $\infty$.
\end{enumerate}

\emph{Erklärung von (iii)}: man wende (ii) an mit $M = T = (\begin{smallmatrix}1&1\\0&1\end{smallmatrix})$, also $f(z+1) = f(z)$.
Wir können jetzt vorgehen wie in \autoref{satz:fourierentwicklung}:
Sei $\mathcal{R} = \Set{q\in \CC \mid 0 < \abs q < 1}$ ein Ringgebiet, $z \mapsto q = e^{2\pi i z}$ bildet $\HH$ auf $\mathcal{R}$ ab.
Sei $F\colon \mathcal{R} \ra \closure\CC$ gegeben durch $F(q) = f(z)$.
Dann ist $F$ auf $\mathcal{R}$ holomorph bis auf Polstellen, die sich nicht gegen $q=0$ häufen können (sonst sind keine Häufungen möglich, denn $f$ ist meromorph).
Man fordert, dass $F$ in $q=0$ eine isolierte nicht-wesentliche Singularität hat (also hebbar oder Pol).

Nach \autoref{satz:fourierentwicklung} hat dann $F$ eine Laurententwicklung
\[
    F(q) = \sum_{n\in\ZZ} a_nq^n
    \qquad \text{für } 0 < \abs q < \abs{q_0}
    \,,
\]
wobei $a_n=0$ für alle bis auf endlich viele $n<0$.
\[
    f(z) = \sum_{n\in\ZZ} a_ne^{2\pi i z}
    \qquad \text{für } 0 < \abs{y_0} < \abs{y}
    \,.
\]
\end{defi}

\begin{defi}
Eine Modulfunktion $f$ heißt \myemph{Modulform} vom Gewicht $k$, falls $f$ auf $\HH$ holomorph und holomorph in $\infty$ ist.
Letzteres bedeutet, dass $F(q)$ in $q=0$ eine hebbare Singularität besitzt, also $f(z) = \sum_{n \geq 0} a_ne^{2\pi i nz}$ für alle $z\in\HH$.

Eine Modulform heißt \myemph{Spitzenform}, falls $a_0 = 0$.
\end{defi}

\begin{defi}
Sei $f\colon \HH \ra \CC$ Funktion, $M = \abcd \in \SL_2(\RR)$, $k\in\ZZ$.
Man setzt
\[
    (f|_k M)(z) := (cz+d)^{-k} f\left(\frac{az+b}{cz+d}\right)
    \,.
\]
Der Operator $f \mapsto (f|_k M)(z)$ heißt \myemph{Peterssonscher Strichoperator}.
\end{defi}

\begin{lemm}\label{lemm:strich_operiert}
Es gilt $f|_kE = f$ und $(f|_kM_1)|_kM_2 = f|_k(M_1M_2)$.
Das heißt $\SL_2(\RR)$ operiert von rechts auf Funktionen $f\colon \HH \ra \CC$ durch $(f, M) \mapsto f|_k M$.
\end{lemm}

\begin{bewe}
$f|_kE = f$ ist klar.

Setze $j(M, z) := cz+d$ für $M = \abcd$.
Dann $j(M_1M_2, z) = j(M_1, M_2 \circ z) j(M_2 \circ z)$ (nachrechnen).
Dann
\[
    (f|_kM)(z) = j(M,z)^{-k}f(M \circ z)
\]
Daher
\begin{align*}
    ((f|_kM_1)|_kM_2)(z)
    &= j(M_2, z)^{-k}j(M_1, M_2\circ z)^{-k} f(M_1 \circ (M_2 \circ z)) \\
    &= j(M_1M_2, z)^{-k}f(M_1M_2 \circ z)
    \,.
\end{align*}
\end{bewe}

\begin{koro}
Eine Funktion $f\colon \HH \ra \CC$ erfüllt
\begin{equation}\label{eq:koro_modulbed}
    f\left(\frac{az+b}{cz+d}\right)
    = (cz+d)^kf(z)
    \qquad \text{für alle } \abcd \in \Gamma(1)
\end{equation}
genau dann, wenn sie
\[
    f(z+1) = f(z)
    \qquad \text{und} \qquad
    f\Big(-\frac{1}{z}\Big) = z^k f(z)
\]
erfüllt.
\end{koro}

\begin{bewe}
Es gilt \eqref{eq:koro_modulbed} genau dann, wenn $f|_k M = f$ für alle $M\in\Gamma(1)$.
Aus $f|_kM = f$ folgt $f|_kM^{-1} = f$ für alle $M\in\Gamma(1)$.
Ferner wird $\Gamma(1)$ von $S = (\begin{smallmatrix}0 & -1\\ 1 & 0\end{smallmatrix})$ und $T = (\begin{smallmatrix}1 & 1\\ 0 & 1\end{smallmatrix})$ erzeugt und $f|_kS = f$ bedeutet $f(-\frac{1}{z}) = z^k f(z)$ und $f|_kT = f$ bedeutet $f(z+1) = f(z)$.
Die Aussage folgt aus \autoref{lemm:strich_operiert}.
\end{bewe}

\begin{nota}
$\mathcal{M}_k =$ $\CC$-Vektorraum der Modulformen von Gewicht $k$ und $\mathcal{S}_k$ der Unterraum der Spitzenformen.
\end{nota}

\begin{beme-list}
\item $k$ ungerade, dann ist $\mathcal{M}_k = \{0\}$.
\emph{Denn}: operiere mit $M = -E$. Dann
\[
    f(z) = f((-E) \circ z) = (-1)^k f(z)
    \,,
\]
also $f \equiv 0$.
\item $f \in \mathcal{M}_k$, $g\in \mathcal{M}_l$, dann ist $fg \in \mathcal{M}_{kl}$.
\item Eine Funktion $f\colon \HH \ra \CC$ ist Modulform vom Gewicht $k$ genau dann, wenn $f$ eine Fourierentwickling $f(z) = \sum_{n\geq 0} a_ne^{2\pi inz}$ hat, die konvergent für alle $z\in \HH$ ist, und es gilt $f(-\frac{1}{z}) = z^kf(z)$.
\end{beme-list}
\section{Beispiele für Modulformen}

\subsection{Thetareihen}

\begin{nota-list}
\item $A \in M_m(\RR)$, $A$ ist symmetrisch, falls $A = A^t$. Man schreibt $A > 0$, falls $A$ positiv definit ist, d.\,h. $x^tAx > 0$ für alle $x\in\RR^m$, $x\not=0$.
\item Ist $A \in M_m(\CC)$, $A$ symmetrisch, $B \in M_{m,n}(\CC)$. Setze $A[B] := B^tAB \in M_{n,n}(\CC)$.
Dann gilt $A[B_1B_2] = (A[B_1])[B_2]$ (nachrechnen).
\end{nota-list}

\begin{defi}
Sei $A \in M_m(\RR)$ symmetrisch und positiv definit.
Dann heißt
\[
    \theta_A(z)
    := \sum_{g \in \ZZ^m} e^{\pi i A[g]z}
    \qquad \text{für } z \in \HH
\]
\myemph{Thetareihe}.
\end{defi}

\begin{lemm}
Die Reihe $\theta_A(z)$ konvergiert gleichmäßig absolut für $y \geq y_0 > 0$, insbesondere auf Kompakta $K \subset \HH$.
Daher ist $\theta_A(z)$ holomorph af $\HH$.
\end{lemm}

\begin{bewe}
Siehe \autoref{lemm:konvergenz_theta} (Im Beweis wurde nicht benutzt, dass $A$ ganzzahlig gerade ist).
\end{bewe}

\begin{satz}[Theta-Transformationsformel]\label{satz:theta_trafo}
Es gilt
\[
    \theta_{A^{-1}}\Big(-\frac{1}{z}\Big)
    = \sqrt{\det A} \cdot \Big(\frac{z}{i}\Big)^{\frac{m}{2}} \theta_A(z)
    \qquad \text{für alle } z \in \HH
    \,.
\]
Hierbei ist $(\frac{z}{i})^{\frac{m}{2}} = e^{\frac{m}{2}\Log(\frac{z}{i})}$.
\end{satz}

\begin{beme}
Ist $A > 0$, dann ist $A^{-1} > 0$ (denn $A^{-1} = A[A^{-1}]$).
\end{beme}

\begin{bewe}
Sei $z\in\HH$ fest.
Für $w\in\CC^m$ setze
\[
    f(w) := \sum_{g \in \ZZ^m} e^{\pi i A[g+w]z}
    \,.
\]
Dann gilt
\[
    \theta_A(z) = f(0)
    \,.
\]

\emph{Behauptung}: Die Reihe $f(w)$ ist auf Kompakta $K \subseteq \CC^m$ gleichmäßig, absolut konvergent.
Insbesondere ist $f(w)$ stetig und separat in jeder Variablen holomorph.

\emph{Denn}: Da $A$ positiv definit ist, existiert $\delta > 0$ mit $A[g] \geq \delta \norm{g}^2 = \delta g^tg$ für alle $g \in \ZZ^m$ (siehe Abschnitt 1).
Betrachte also
\[
    \Im(A[g+w]z)\,.
\]
Schreibe $A[g+w] = A[g] + 2w^tAg + A[w]$.
Daher
\begin{align}\label{eq:theta_transfo}
    \Im(A[g+w]z)
    &= A[g]y + 2\Im(w^tAgz) + \Im(A[w]z) \nonumber \\
    &\geq y\delta g^tg + 2\Im(w^tAgz) + \Im(A[w]z)
    \,.
\end{align}
Sei $K \subset \CC^m$ kompakt.
Da die Abbildung $g \mapsto g^tg$ quadratisch und $g \mapsto \Im(w^tAgz)$ linear ist für alle $z \in K$ und für fast alle $g \in \ZZ^m$ die rechte Seite von \eqref{eq:theta_transfo} $ \geq \frac{1}{2}y\delta g^tg$.

Es gilt daher für alle $w\in\CC^m$
\[
    \sum_{g\in\ZZ^m} \abs{e^{\pi i A[g+w]z}}
    = \sum_{g\in\ZZ^m} e^{-\Im(\pi A[g+w]z)}
    \leq C + \sum_{g\in\ZZ^m} e^{-\frac{\pi y \delta}{2} g^tg}
    < \tilde{C} < \infty
    \,.
\]

\medskip

Wegen absoluter Konvergenz der Reihe $f(w)$ gilt
\[
    f(w+h) = f(w)
    \qquad \text{für alle } h \in \ZZ^m
    \,.
\]
Daher hat $f(w)$ eine Fourierentwicklung der Form
\[
    f(w)
    = \sum_{h\in\ZZ^m} a(h) e^{2\pi i h^tw}
    \qquad \text{für } w \in \CC^m
    \,,
\]
die gleichmäßig absolut auf Kompakta konvergiert. Mit Fourierkoeffizienten
\[
    a(h) = \int_0^1 \ldots \int_0^1 f(w)e^{-2\pi i h^tw} \opd u
    \qquad w = u+iv, v\in\RR^m \text{ fest}
\]
Für $m=1$ wird dies im Detail in \autoref{satz:fourierentwicklung} gezeigt, der Fall $m \geq 1$ folgt aus dem Fall $m=1$ und Induktion.

Es folgt
\[
    a(h) = \int_{I_m} f(w)e^{-2\pi i h'w} \opd u
    \qquad \text{wobei } I_m = [0,1]^m \subset \RR^m \text{ (Fubini)}\,.
\]

Einsetzen liefert
\[
    a(h)
    = \int_{I_m} \left(\sum_{g\in\ZZ^m} e^{\pi i(A[g+w]z-2h^tw])}\right) \opd u
    \,.
\]
Vertauschen von Summation und Integration ist gerechtfertig, wegen der absoluten Konvergenz, also gilt:
\begin{align*}
    a(h)
    &= \sum_{g\in\ZZ^m} \left(\int_{I_m} e^{\pi i(A[g+w]z-2h^tw])} \opd u\right) \\
    &= \sum_{g\in\ZZ^m} \left(\int_{g+I_m} \underbrace{e^{\pi i(A[w]z-2h^tw])}}_{\text{unabhängig von $g$}} \opd u\right) \qquad (u \mapsto u-g) \\
    &= \int_{\bigcup\limits_{g\in\ZZ^m} g+I_m} e^{\pi i(A[w]z-2h^tw])} \opd u
        \qquad (\text{Poissonscher Summationstrick}) \\
    &= \int_{\RR^m}  e^{\pi i(A[w]z-2h^tw])} \opd u
    \,.
\end{align*}
Quadratische Ergänzung (ähnlich wie für $m=1$, nachrechnen), es gilt
\[
    A[w]z - 2h^tw
    = zA[w - A^{-1}hz^{-1}]-A^{-1}[h]z^{-1}
    \,.
\]
Damit erhalten wir
\begin{align*}
    a(h)
    &= e^{-\pi iA^{-1}[h]z^{-1}} \int_{\RR^m} e^{\pi i z A[w-A^{-1}hz^{-1}]} \\
    &\qquad \left( w = u+iv, \text{ wähle } v = \Im(A^{-1}hz^{-1}),\ u\mapsto u + \Re(A^{-1}hz^{-1}) \right) \\
    &= e^{-\pi iA^{-1}[h]z^{-1}} \int_{\RR^m} e^{\pi iA[u]z} \opd u
\end{align*}

Also haben wir
\[
    \theta_A(z)
    = f(0)
    = \sum_{h\in\ZZ^m} a(h)
    = \underbrace{\sum_{h\in\ZZ^m} e^{\pi i(-\frac{1}{z})A^{-1}[h]}}_{=\theta_{A^{-1}}(-\frac{1}{z})} \cdot \left( \int_{\RR^m} e^{\pi iA[u]z} \opd u \right)
    \,.
\]

Wir müssen nun nur noch das Integral berechnen.
Dafür wollen wir $A$ diagonalisieren:
Nach LA existiert $U \in \GL_n(\RR)$ mit
\[
    A[U] = \begin{pmatrix}d_1 & &\\ & \ddots & \\ & & d_m\end{pmatrix} = D^tD = E[D]
    \qquad \text{mit } D = \begin{pmatrix}\sqrt{d_1} & &\\ & \ddots & \\ & & \sqrt{d_m}\end{pmatrix}
\]
Also $A = E[DU^{-1}] = B^tB$ mit $B := DU^{-1} \in \GL_n(\RR)$.
Wir erhalten
\begin{align*}
    \int_{\RR^m} e^{\pi iA[u]z} \opd u
    &= \int_{\RR^m} e^{\pi i(Bu)^t(Bu)z} \opd u \\
    &= \abs{\det B^{-1}} \int_{\RR^m} e^{\pi iu^tuz} \opd u \qquad (u \mapsto B^{-1}u) \\
\end{align*}
Die Substitution $u \mapsto B^{-1}u$ hat die Jacobi-Determinante $\det B^{-1}$.
Aus $A = B^tB$ folgt $\det A = (\det B)^2$, also $\abs{\det B^{-1}} = (\sqrt{\det A})^{-1}$.
Mit mehrfacher Anwendung von Fubini folgt
\begin{align*}
    \int_{\RR^m} e^{\pi iA[u]z} \opd u
    &= (\det A)^{-\frac{1}{2}} \int_{\RR^m} e^{\pi i (u_1^2 + \ldots + u_m^2)z} \opd u \\
    &= (\det A)^{-\frac{1}{2}} \int_{-\infty}^\infty\ldots\int_{-\infty}^\infty e^{\pi i (u_1^2 + \ldots + u_m^2)z} \opd u
    = \left( \int_{-\infty}^\infty e^{\pi i t^2 z} \opd t \right)^m
\end{align*}

Berechne $\int_{-\infty}^\infty e^{\pi i t^2 z} \opd t$ zunächst für $z = iy$, $y > 0$.

\begin{lemm-ind}\label{lemm:theta_trafo}
\[
    \int_{-\infty}^\infty e^{\pi iyt^2} \opd t = y^{-\frac{1}{2}}
\]
\end{lemm-ind}

\begin{bewe-ind}
Setze $t = \frac{x}{\sqrt{\pi y}}$.
Dann
\[
    \int_{-\infty}^\infty e^{\pi iyt^2} \opd t
    = \frac{1}{\sqrt{\pi y}} \int_{-\infty}^\infty e^{-x^2} \opd x
    \,.
\]
Also zu zeigen
\[
    \int_{-\infty}^\infty e^{-x^2} \opd x = 2\int_0^\infty e^{-x^2} \opd x
    = \sqrt{\pi}
    \,.
\]
Eulersche Ergänzungsstaz (\autoref{satz:gamma-eigenschaften})
\[
    \Gamma(s) \Gamma(1-s) = \frac{\pi}{\sin(\pi s)}
    \,,
\]
setze $s = \frac{1}{2}$, dann $\Gamma(\frac{1}{2}) = \pi$, also $\Gamma(\frac{1}{2}) = \sqrt{\pi}$.
Euler Integral (\autoref{satz:eulerint})
\[
    \Gamma(s) = \int_0^\infty t^{s-1}e^{-t} \opd t \qquad (\Re(s) > 0)\,.
\]
Damit folgt für $s=\frac{1}{2}$
\[
    \sqrt \pi = \Gamma(\frac{1}{2}) = \int_0^\infty t^{-\frac{1}{2}}e^{-t} \opd t
    = \int_0^\infty x^{-1}e^{-x^2}\cdot 2x \opd x %t = x^2
    = 2 \int_0^\infty e^{-x^2} \opd x
\]
\end{bewe-ind}

Die Funktion $(\frac{z}{i})^{-\frac{m}{2}}$ und $(\int_{-\infty}^\infty e^{\pi i t^2 z} \opd t)^m$ sind auf $\HH$ holomorph und stimmen auf der positiven imaginären Achse überein (\autoref{lemm:theta_trafo}).
Nach dem Identitässatz stimmen sie also auf $\HH$ überein.
Damit folgt die Behauptung.
\end{bewe}

\begin{satz}\label{satz:theta_modulform}
Sei $A \in M_m(\ZZ)$, $A$ symmetrisch, positiv definit, gerade und $\det A = 1$.
Dann gilt $8 | m$ und $\theta_A \in \M_{\frac{m}{2}}$.
\end{satz}
\begin{beme}
Solche $A$ existieren und können konstruiert werden mit Hilfe der Theorie der unimodularen Gitter.
\end{beme}

\begin{bewe}
Zunächst beachte: Ist $A \in M_m(\RR)$, $A$ symmetrisch, positiv definit und $B = A[U]$ mit $U \in \GL_m(\ZZ)$, so gilt $\theta_B(Z) = \theta_A(z)$.

\emph{Denn}:
\[
    \theta_B(z)
    = \sum_{g \in \ZZ^m} e^{\pi iB[g]z}
    = \sum_{g \in \ZZ^m} e^{\pi i A[U][g]z}
    = \sum_{g \in \ZZ^m} e^{\pi i A[Ug]z}
\]
denn wegen $U\in\GL_m(\ZZ)$, durchläuft mit $g$ auch $Ug$ alle Elemente aus $\ZZ^m$.

Sei $\det A = 1$.
Dann ist $A^{-1} = A[U]$ mit $U = A^{-1}$ mit $U = A^{-1} \in \GL_m(\ZZ)$.
Aus den Voraussetzungen folgt damit
\[
    \theta_A = \theta_{A^{-1}}
    \,.
\]
Also folgt mit \autoref{satz:theta_trafo}
\[
    \theta_{A}\Big(-\frac{1}{z}\Big)
    = \sqrt{\det A} \cdot \Big(\frac{z}{i}\Big)^{\frac{m}{2}} \theta_A(z)
    \,.
\]

\emph{Angenommen}: $8 \nmid m$.

Dann kann man annehmen $m \equiv 4 \mod 8$, denn man kann $A$ immer ersetzen durch
\[
    (**)\quad \begin{pmatrix}A&\\&A\end{pmatrix} \qquad \text{oder} \qquad (*) \quad \begin{pmatrix}A\\&A\\&&A\\&&&A\end{pmatrix}
\]
und diese Matrizen haben den gleichen Typ.

\begin{tabular}{l l}
$m \equiv 1$ & nehme (**) dann $m \equiv 4 \mod 8$ \\
$m \equiv 2$ & nehme (*)\phantom{*} dann $m \equiv 4 \mod 8$ \\
$m \equiv 3$ & nehme (**) dann $m \equiv 4 \mod 8$ \\
$m \equiv 4$ & \ldots
\end{tabular}

Sei also $m \equiv 4 \mod 8$.
Dann gilt $\theta_A(-\frac{1}{z}) = z^{\frac{m}{2}} \cdot i^{-\frac{m}{2}} \theta_A(z)$.
Also $\theta_A|_{\frac{m}{2}}S = \theta_A$. Aber $A$ gerade, also $\frac{1}{2}g^tAg = \frac{1}{2}A[g] \in \ZZ$.
Also Fourierentwicklung von $\theta_A$ ist in $e^{2\pi iz}$, d\,h. $\theta|_{\frac{m}{2}}T = \theta_A$.
Also $\theta_A|TS = \theta_A|T|S = - \theta_a$.
Also $\theta_A = \theta_A|(-E) = \theta_A|(TS)^3 = \theta_A|TS|TS|TS = -\theta_A$.
Also $\theta_A = 0$ \blitz konstanter Term ist 1.

Also $8|m$, Aber dann $\theta_A|T = \theta_A$, $\theta_A|S = \theta_A$. Also $\theta_A \in M_{\frac{m}{2}}$.
\end{bewe}

\emph{Problem}: Wie erhält man Matrizen $A$ wie in \autoref{satz:theta_modulform} gefordert?

\emph{Idee}: Man verwende die Theorie der \emph{unimodularen Gitter}!

\subsubsection{Exkurs: unimodulare Gitter}

\begin{defi}
	Eine Teilmenge $L \subset \RR^m$ heißt \myemph{Gitter}, falls eine Basis $\Set{v_1, \ldots, v_m}$ des $\RR^m$ existiert, so dass $L = \ZZ v_1 \oplus \ldots \oplus \ZZ v_m$.
\end{defi}

\begin{bsp}
	$L = \ZZ^m$
\end{bsp}

\begin{beme}
	Ist $m = 2$ und $\RR^2 \cong \CC$, so wurden Gitter $L \subset \CC$ im Kapitel über elliptischen Funktionen betrachtet.
	
	Ist $B$ die Matrix, deren Spalten gerade die Komponenten von $v_1$, \ldots, $v_m$ sind, so gilt $L = B(\ZZ^m)$.
	Weiterhin gilt
	\[
	\abs{\det B} = \vol(\modulo{\RR^m}{L}) = \int_\F\opd\mu
	\,,
	\]
	wobei $\F$ ein \emph{Fundamentalbereich} ist, z.\,B. die \emph{Grundmasche}.
	
	\emph{Fakt}: Ist $L_1 \subset L$ ein $\ZZ$-Untermodul von $L$ mit $[L_1:L] < \infty$, so ist auch $L_1 \subset \RR^m$ ein Gitter.
	
	Schränkt man die quadratische Form
	\[
	\RR^m \ra \RR, \quad
	x \mapsto x^t x = \sum_{\nu=1}^m x_\nu^2
	\]
	auf $L$ ein, so erhält man eine quadratische Form auf $L$ welche für $x = Bg \in L$ beliebig mit $g\in\ZZ^m$ gegeben ist durch
	\[
	x^tx = (Bg)^t(Bg) =g^tB^tBg = g^tAg
	\,,
	\]
	wobei $A = B^tB = (v_i \cdot v_j)_{1 \leq i,j \leq m}$.
	Nach Konstruktion ist $A$ symmetrisch und positiv definit.
	
	Man konstruiere Matrizen $A$ wie in \autoref{satz:theta_modulform}, indem man geeignete Gitter $L \subset \RR^m$ und deren \emph{Gram-Matrix} $A$ anschaut.
\end{beme}

\begin{bsp}
	Sei $m \in \NN$, $8|m$.
	Sei
	\[
	E_m := \Set{x\in\ZZ^m \biggm| \sum_{\nu=1}^m x_\nu \equiv 0 \mod 2}
	\,.
	\]
	Die Abbildung $\ZZ^m \ra \modulo{\ZZ}{2\ZZ}, x \mapsto \conj{\sum_{\nu = 1}^m x_\nu}$ ist ein surjektiver Homomorphismus und sein Kern ist $E_m$, nach dem Homomorphiesatz ist daher $\modulo{\ZZ^m}{E_m} \cong \modulo{\ZZ}{2\ZZ}$.
	Also $[\ZZ^m:E_m] = 2$, insbesondere ist $E_m$ also auch ein Gitter.
	
	Sei $L_m := E_m + \ZZ e$, wobei $e^t = (\frac{1}{2}, \frac{1}{2}, \ldots, \frac{1}{2}) \in \RR^m$.
	Man zeigt, dass auch $L_m \subset \RR^m$ ein Gitter ist und $[L_m : E_m] = 2$.
	Explizit kann man $L_m$ angeben:
	\[
	L_m = \Set{x \in \RR^m \biggm| x_i \in \frac{1}{2}\ZZ,\ x_i - x_j \in \ZZ \text{ für alle } i, j \text{ und } \frac{1}{2} \sum_{\nu =1}^m x_\nu \in \ZZ}
	\]
	
	Es kann gezeigt werden, dass $x \cdot y \in \ZZ$ für alle $x, y \in L_m$ und $x \cdot x \in 2\ZZ$ für alle $x \in L_m$. Insbesondere ist $e \cdot e = \frac{1}{4} m = \frac{m}{4} \in 2\ZZ$ wegen $8|m$.
	
	Es verbleibt zu zeigen, dass $\det A = 1$.
	Zunächst gilt
	\begin{align*}
	&E_m \underset{2}{\subset} \ZZ^m \subset \RR^m \\
	&E_m \underset{2}{\subset} L_m \subset \RR^m
	\end{align*}
	Damit erhalten wir
	\begin{align*}
	&\vol(\modulo{\RR^m}{E_m}) = 2 \cdot \vol(\modulo{\RR^m}{\ZZ^m}) = 2 \\
	&\vol(\modulo{\RR^m}{E_m}) = 2 \cdot \vol(\modulo{\RR^m}{L_m})
	\end{align*}
	Also
	\[
	\vol(\modulo{\RR^m}{L_m}) = 1
	\,.
	\]
	Woraus wir $\abs{\det B} = 1$ und damit $\det A = 1$, wegen $A = B^tB$, erhalten.
	
	\emph{Speziell} heißt $L_8$ \myemph{Leech-Gitter}.
	Eine Basis hierfür ist gegeben durch
	\begin{align*}
	&\frac{1}{2}(e_1 + e_8) - \frac{1}{2}(e_2 + \ldots + e_7)\,,\\
	&e_1+e_2\,,\\
	&e_i - e_{i-1} \qquad \text{ für } 2 \leq i \leq 7\,.
	\end{align*}
	
	Damit ergibt sich
	\[
	A = 
	\begin{pmatrix}
	2 & 0 & -1 \\
	0 & 2 & 0 & -1 \\
	-1 & 0 & 2 & -1 \\
	& -1 & -1 & 2 & -1 \\
	& & & -1 & 2 & -1 \\
	& & & & -1 & 2 & -1 \\
	& & & & & -1 & 2 & -1 \\
	& & & & & & -1 & 2 \\
	\end{pmatrix}
	\]
	und $A$ ist positiv definit, symmetrisch und $\det A = 1$.
\end{bsp}

\subsection{Eisensteinreihen}

\begin{defi}
	Sei $k \in \NN$, $k$ gerade, $k \geq 4$. Dann heißt
	\[
	G_k(z) := \sumprime_{m,n} \frac{1}{(mz+n)^k}
	\qquad \text{für } z \in \HH
	\]
	\myemph{Eisensteinreihe} vom Gewicht $k$.
\end{defi}

\begin{beme-list}
	\item $\sumprime_{m,n}$ bedeutet die Summation über alle $(m, n) \in \ZZ^2$ mit $(m,n) \not= (0,0)$.
	
	\item Beachte $\sumprime_{m,n} \frac{1}{(mz+n)^k} = 0$ falls $k$ ungerade (ersetze $(m, n)$ durch $(-m, -n)$).
	
	\item Es wird $k \geq 4$ gefordert, damit die Reihe konvergiert.
\end{beme-list}

\begin{satz-list}
	\item Die Reihe $G_k(z)$ ist absolut, gleichmäßig konvergent auf 
	\[
		D_\epsilon = \Set{z = x + iy \mid y \geq \epsilon,\ \abs{x}^2 \leq \frac{1}{\epsilon}}
	\]
	für $\epsilon > 0$.
	
	\item $G_k \in \M_k$.
\end{satz-list}

\begin{bewe-list}
	\item Zunächst gilt
	\begin{equation}\label{eq:satz_konv_eisensteinreihen}
	\abs{mz + n}^2 \geq \delta (m^2+n^2) = \delta \abs{mi +n}^2
	\qquad \text{für alle } m, n\in\ZZ, z \in D_\epsilon
	\end{equation}
	für $\delta > 0$ geeignet ist.
	\emph{Denn}:
	\eqref{eq:satz_konv_eisensteinreihen} ist äquivalent zu 
	\[
	m^2(x^2+y^2) + 2mnx+n^2 \geq \delta (m^2 + n^2) \qquad \text{für alle } m, n\in\ZZ, z \in D_\epsilon
	\,.
	\]
	Dies gilt genau dann, wenn
	\[
	(x^2+y^2-\delta)m^2 + 2xmn + (1-\delta)n^2 \geq 0 \qquad \text{für alle } m, n\in\ZZ, z \in D_\epsilon
	\,.
	\]
	Es genügt also zu zeigen, dass die quadratische Form
	\[
	(X, Y) \mapsto (x^2+y^2 - \delta)X^2 + 2xXY + (1-\delta)Y^2
	\]
	positiv definit ist für alle $z \in D_\epsilon$ mit geeignetem $\delta > 0$.
	Das heißt es muss gelten:
	\begin{enumerate}
		\item $1 - \delta > 0$
		\item Diskriminante $(2x)^2 - 4(x^2+y^2-\delta)(1-\delta) < 0$
	\end{enumerate}
	
	Dies gilt tatsächlich
	\begin{enumerate}
		\item falls $\delta < 1$.
		
		\item ist äquivalent zu $y^2(1- \delta) - \delta x^2 - \delta(1-\delta) > 0$.
		Aber für $z \in D_\epsilon$ gilt $x^2 \leq \frac{1}{\epsilon}$, also gilt dort
		\[
		y^2(1 - \delta) - \delta x^2 - \delta(1 - \delta)
		> y^2 (1- \delta) - \delta\epsilon^{-1}-\delta(1-\delta)
		\,,
		\]
		also reicht es zu zeigen, dass für alle $z \in D_\epsilon$ gilt
		\[
		y^2 > \frac{\delta\epsilon^{-1} + \delta(1-\delta)}{1 - \delta}
		\,.
		\]
		Aber $y^2 \geq \epsilon^2$ ist nach unten beschränkt und $\frac{\delta\epsilon^{-1} + \delta(1-\delta)}{1 - \delta} \xto{\delta \to 0} 0$.
	\end{enumerate}
	Also gibt es $\delta > 0$ mit den gewünschten Eigenschaften.
	
	Damit folgt nun für alle $z\in D_\epsilon$
	\[
	\sumprime_{m,n} \frac{1}{\abs{mz+n}^k} \leq \delta^{-\frac{k}{2}} \sumprime_{m,n} \frac{1}{\abs{mz+n}^k}
	\]
	die Konvergenz der rechten Seite folgt aus \autoref{lemm:weierstrass-konv} und damit folgt die gleichmäßige, absolute Konvergenz auf $D_\epsilon$ von $G_k$.
	
	\item zu zeigen
	\begin{enumerate}
		\item $G_k(z + 1) = G_k(z)$, $G_k(-\frac{1}{z}) = z^kG_k(z)$
		\item $G_k$ ist holomorph in $\infty$.
	\end{enumerate}
	
	(a) Aufgrund der absoluten Konvergenz folgt
	\begin{align*}
	G_k(z+1)
	&= \sumprime_{m,n} \frac{1}{(m(z+1)+n)^k}
	= \sumprime_{m,n} \frac{1}{(mz+(m+n))^k} \\
	&= \sumprime_{m,n} \frac{1}{(mz+n)^k}
	= G_k(z)
	\,,	
	\end{align*}
	denn wenn $(m,n)$ alle Elemente von $\ZZ^2\setminus\{0\}$ durchläuft, so auch $(m,n+m)$.
	
	Außerdem gilt
	\[
	G_k\Bigl(-\frac{1}{z}\Bigr)
	= \sumprime_{m,n} \frac{1}{(m(-\frac{1}{z})+n)^k}
	= z^k \sumprime_{m,n} \frac{1}{(nz+m)^k}
	= z^k G_k(z)
	\,.
	\]
	
	(b) Es ist zu zeigen, dass die Funktion $H(q) := G_k(z)$ mit $q = e^{2\pi iz}$, $0 < \abs q < 1$ in $q = 0$ eine hebbare Singularität hat.
	Nach dem Riemann'schen Hebbarkeitssatz bedeutet dies, dass $H(q)$ in einer kleinen punktierten Umgebnung von $q=0$ beschränkt ist.
	Dies ist sicherlich der Fall, wenn $\lim_{q \to 0} H(q)$ existiert, d.\,h. $\lim_{y \to \infty} G_k(z)$ existiert.
	
	Wähle Folge $(z_\nu)_\nu$ in $\HH$ mit $y_\nu \to \infty$. Wegen $G_k(z+1) = G_k(z)$ kann man annehmen, dass $\abs{x_\nu} \leq \frac{1}{2}$.
	Wegen der gleichmäßigen Konvergenz in $D_\epsilon$ folgt
	\[
	\lim_{\nu \to \infty} G_k(z_\nu)
	= \lim_{\nu \to \infty} \sumprime_{m,n} \frac{1}{(mz_\nu+n)^k}
	= \sumprime_{m,n} \lim_{\nu \to \infty} \frac{1}{\underbrace{(mz_\nu+n)^k}_{\to 0 \text{ für } m\not=0}}
	= \sum_{n \not= 0} \frac{1}{n^k}
	< \infty
	\]
\end{bewe-list}

\begin{satz}
	Sei $k \in \NN$. Es gilt für $k$ gerade
	\[
	G_k(z) = 2\zeta(k) + \frac{2(2\pi i)^k}{(k-1)!} \sum_{n\geq1} \sigma_{k-1}(n) e^{2\pi inz}
	\qquad \text{für } z\in\HH
	\]
	mit der \myemph{Riemann'schen $\zeta$-Funktion}
	\[
	\zeta(k) = \sum_{n=1}^\infty \frac{1}{n^k}
	\]
	und
	\[
	\sigma_{k-1}(n) = \sum_{\substack{d|n \\ \scriptscriptstyle d >0}} d^{k-1}
	\]
\end{satz}

\begin{bewe}
	Der Teil der Reihe von $G_k$ mit $m=0$ ist gerade $\sum_{n\not=0} \frac{1}{n^k} = 2\zeta(k)$ (dann sind alle $n\in\ZZ$, $n\not=0$).
	Damit folgt wegen der absoluten Konvergenz:
	\[
	G_k(z)
	= 2 \zeta(k) + \sum_{m\not=0} \left(\sum_{n\in\ZZ} \frac{1}{(mz+n)^k}\right)
	= 2 \zeta(k) + 2 \sum_{m \geq 1} \left( \sum_{n\in\ZZ} (mz+n)^{-k}\right)
	\,.
	\]
	In \autoref{bsp:eisensteinreihentrafo} wurde bewiesen
	\[
	\sum_{n\in\ZZ} \frac{1}{(z+n)^k}
	= \frac{(-2\pi i)^k}{(k-1)!} \sum_{n\geq1} n^{k-1}e^{2\pi inz}
	\,.
	\]
	Benutze dies mit $z \mapsto mz$ mit $m \geq 1$ und $N := mn$, so folgt
	\begin{align*}
	G_k(z)
	&= 2\zeta(k) + \frac{2(2\pi i)^k}{(k-1)!} \sum_{m\geq1, n\geq1} e^{2\pi inmz} \\
	&= 2\zeta(k) + \frac{2(2\pi i)^k}{(k-1)!} \sum_{N\geq1} \left(\sum_{n|N} n^{k-1}\right) e^{2\pi iNz}
	\,.
	\end{align*}
\end{bewe}

\begin{satz}\label{satz:bernoullie}
	Man definiere die \myemph{Bernoulliezahlen} $B_n$ für alle $n\in\NN$ durch die Taylorentwicklung
	\[
	\frac{t}{e^t-1} = \sum_{n=0}^\infty \frac{B_n}{n!} t^n
	\qquad \abs t < 2\pi
	\,.
	\]
	Dann gilt
	\begin{enumerate}
		\item $B_n \in \QQ$ für alle $n\in\NN$. Speziell ist $B_0 = 1$, $B_1 = -\frac{1}{2}$ und $B_n = 0$ für alle $n\in\NN$ mit $n$ ungerade.
		
		\item Es gilt die Rekursionsformel $\sum_{\nu=0}^n \binom{n}{\nu} B_\nu = (-1)^nB_n$.
		
		\item (Euler) Für $k\in\NN$ gerade, $k \geq 2$ gilt
		\[
		\zeta(k) = \frac{(-1)^{\frac{k}{2}-1} 2^{k-1}B_k}{k!} \pi^k
		\]
	\end{enumerate}
\end{satz}

\begin{beme}
	Die Funktion $\frac{w}{e^w-1}$ hat in $w=0$ eine hebbare Singularität\footnote{wegen $e^w = 1 + w + \frac{w^2}{2!} + \ldots$ und der konstanten Term in der Entwicklung von $w=0$ ist $B_0 = 1$} und hat in $w = 2\pi in$ für $n\in\ZZ$, $n\not=0$ einfache Polstellen.
	Da die Taylorreihe im größtmöglichen Kreis konvergiert (Funktionentheorie I), ist die Entwicklung von $\frac{w}{e^w-1}$ in $w=0$ konvergent für $\abs w < 2\pi$.
\end{beme}

\begin{bewe-list}
	\item Man anti-symmetrisiere!
	\[
	-t = \frac{t}{e^t-1} - \frac{-t}{e^{-t}-1}
	= \sum_{n \geq 0} \frac{B_n-(-1)^nB_n}{n!} t^n
	= 2 \sum_{\substack{n\geq 1\\ \scriptscriptstyle n \text{ ungerade}}} \frac{B_n}{n!} t^n
	\]
	Also $2B_1 = -1$ beziehungsweise $B_1 = -\frac{1}{2}$ und $B_n = 0$ für alle $n\in\NN$ ungerade.
	
	\item Es gilt
	\begin{align*}
	\sum_{n\geq1} \left(\sum_{\nu = 0}^n \binom{n}{\nu} B_\nu \right) \frac{t^n}{n!}
	&= \sum_{\substack{n\geq 0 \\ \scriptscriptstyle 0 \leq \nu \leq n}} \frac{1}{\nu!(n-\nu)!} B_\nu t^n % Setze k := n-\nu
	= \sum_{\nu \geq 0, k \geq 0} \frac{1}{\nu!k!} B_\nu t^{\nu+k} \\
	&= \left(\sum_{\nu\geq0} \frac{B_\nu}{\nu!}t^\nu\right) \left(\sum_{k\geq0} \frac{t^k}{k!}\right)
	= \frac{t}{e^t-1} \cdot e^t \cdot \frac{e^{-t}}{e^{-t}} \\
	&= \frac{t}{1-e^{-t}}
	= \frac{-t}{e^{-t}-1}
	= \sum_{n\geq1} (-1)^n B_n\frac{t^n}{n!}
	\,.
	\end{align*}
	Also gilt die Rekursionsformel und man schließt, dass alle $B_n \in \QQ$.\footnote{z.\,B. $n=3$: $B_0 + 3B_1 + 3B_2 + B_3 = -B_3$, also $B_2 = \frac{1}{6}$}
	
	\item Setze $t = 2\pi iz$ mit $\abs z < 1$, $z\not=0$.
	Dann
	\[
	\frac{2\pi iz}{e^{2\pi iz - 1}}
	= \sum_{n \geq 0} \frac{B_n}{n!} (2\pi iz)^n
	= 1 - \pi iz + \sum_{\substack{n\geq 2\\ \scriptscriptstyle n \text{ gerade}}} \frac{B_n}{n!} (-1)^\frac{n}{2} 2^n\pi^n z^n
	\]
	denn
	\begin{align*}
	\cot \pi z - i
	&= \frac{\cos \pi z}{\sin\pi z}
	= \frac{\frac{e^{\pi iz} + e^{-\pi iz}}{2}}{\frac{e^{\pi iz} - e^{-\pi iz}}{2i}} - i
	= i\frac{e^{2\pi iz}+1}{e ^{2\pi iz} -1} - i \\
	&= i\left( \frac{e^{2\pi iz} + 1 - e^{2\pi iz} + 1}{e^{2\pi iz} - 1}\right)
	= \frac{2i}{e^{2\pi iz}}
	\end{align*}
	
	Es folgt
	\[
	\pi z \cot \pi z = 1 - \pi iz + \sum_{\substack{n\geq 2\\ \scriptscriptstyle n \text{ gerade}}} \frac{B_n}{n!} (-1)^{\frac{n}{2}} 2^n \pi^n + \pi iz
	\]
	Partialbruchzerlegung des Kotangens (siehe \autoref{bsp:partialbruch_cot}):
	\[
	\pi \cot \pi z = \frac{1}{z} + \sum_{n\geq1} \left(\frac{1}{z-n} + \frac{1}{z+n}\right)
	\]
	also
	\begin{align*}
	\pi z \cot \pi z
	&= 1 + \sum_{n\geq1} \frac{2z^2}{z^2-n^2}
	= 1 - 2z^2\sum_{n\geq1} \frac{1}{(1-(\frac{z}{n})^2)n^2} \\ % geometrische Reihe
	&= 1 - 2z^2 \sum_{n\geq1} \frac{1}{n^2} \sum_{k \geq 0} \Big(\frac{z}{n}\Big)^{2k}
	= 1 - 2 \sum_{k \geq 0} \sum_{n \geq 1} \frac{1}{n^{2k+2}} z^{2k+2} \\
	&= 1 - 2\sum_{\substack{k\geq2\\ \scriptscriptstyle k \text{ gerade}}} \zeta(k)z^k
	\,.
	\end{align*}
	Mit einem Koeffizientenvergleich ergibt sich die Behauptung.
\end{bewe-list}

\begin{defi}
	Man definiere die \myemph{normalisierte Eisensteinreihe} $E_k(z)$ vom Gewicht $k$ durch
	\[
	E_k(z) := \frac{1}{2\zeta(k)} G_k(z)
	\,.
	\]
	Aus \autoref{satz:bernoullie} (iii) folgt dann
	\[
	E_k(z) = 1 - \frac{2k}{B_k} \sum_{n\geq1} \sigma_{k-1}(n) q^n
	\]
	hierbei sind alle Fourierkoeffizienten von $E_k$ rational.
\end{defi}

Insbesondere findet man
\[
E_4 = 1 + 240 \sum_{n\geq 1} \sigma_3(n)q^n, \qquad
E_6 = 1 - 504 \sum_{n \geq 1} \sigma_5(n)q^n
\,.
\]
Man kann zeigen: $E_4$ und $E_6$ erzeugen die Algebra der Moulformen für $\Gamma(1)$.
Das heißt, jedes $f \in \M_k$ hat eine eindeutige Darstellung
\[
f = \sum_{\substack{\alpha, \beta \geq 0\\ \scriptscriptstyle 4\alpha + 6\beta = k}} \lambda_{\alpha, \beta} E_4^\alpha E_6^\beta
\,,
\]
wobei $\lambda_{\alpha, \beta} \in \CC$.
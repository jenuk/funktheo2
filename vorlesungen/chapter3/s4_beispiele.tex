\section{Beispiele für Modulformen}

\subsection{Thetareihen}

\begin{nota-list}
\item $A \in M_m(\RR)$, $A$ ist symmetrisch, falls $A = A^t$. Man schreibt $A > 0$, falls $A$ positiv definit ist, d.\,h. $x^tAx > 0$ für alle $x\in\RR^m$, $x\not=0$.
\item Ist $A \in M_m(\CC)$, $A$ symmetrisch, $B \in M_{m,n}(\CC)$. Setze $A[B] := B'AB \in M_{n,n}(\CC)$.
Dann gilt $A[B_1B_2] = (A[B_1])[B_2]$ (nachrechnen).
\end{nota-list}

\begin{defi}
Sei $A \in M_m(\RR)$ symmetrisch und positiv definit.
Dann heißt
\[
    \theta_A(z)
    := \sum_{g \in \ZZ^m} e^{\pi i A[g]z}
    \qquad \text{für } z \in \HH
\]
\myemph{Thetareihe}.
\end{defi}

\begin{lemm}
Die Reihe $\theta_A(z)$ konvergiert gleichmäßig absolut für $y \geq y_0 > 0$, insbesondere auf Kompakta $K \subset \HH$.
Daher ist $\theta_A(z)$ holomorph af $\HH$.
\end{lemm}

\begin{bewe}
Siehe \autoref{lemm:konvergenz_theta} (Im Beweis wurde nicht benutzt, dass $A$ ganzzahlig gerade ist).
\end{bewe}

\begin{satz}[Theta-Transformationsformel]
Es gilt
\[
    \theta_{A^{-1}}\Big(-\frac{1}{z}\Big)
    = \sqrt{\det A} \cdot \Big(\frac{z}{i}\Big)^{\frac{m}{2}} \theta_A(z)
    \qquad \text{für alle } z \in \HH
    \,.
\]
Hierbei ist $(\frac{z}{i})^{\frac{m}{2}} = e^{\frac{m}{2}\Log(\frac{z}{i})}$.
\end{satz}

\begin{beme}
Ist $A > 0$, dann ist $A^{-1} > 0$ (denn $A^{-1} = A[A^{-1}]$).
\end{beme}
\section{Beispiele für Modulformen}

\subsection{Thetareihen}

\begin{nota-list}
\item $A \in M_m(\RR)$, $A$ ist symmetrisch, falls $A = A^t$. Man schreibt $A > 0$, falls $A$ positiv definit ist, d.\,h. $x^tAx > 0$ für alle $x\in\RR^m$, $x\not=0$.
\item Ist $A \in M_m(\CC)$, $A$ symmetrisch, $B \in M_{m,n}(\CC)$. Setze $A[B] := B'AB \in M_{n,n}(\CC)$.
Dann gilt $A[B_1B_2] = (A[B_1])[B_2]$ (nachrechnen).
\end{nota-list}

\begin{defi}
Sei $A \in M_m(\RR)$ symmetrisch und positiv definit.
Dann heißt
\[
    \theta_A(z)
    := \sum_{g \in \ZZ^m} e^{\pi i A[g]z}
    \qquad \text{für } z \in \HH
\]
\myemph{Thetareihe}.
\end{defi}

\begin{lemm}
Die Reihe $\theta_A(z)$ konvergiert gleichmäßig absolut für $y \geq y_0 > 0$, insbesondere auf Kompakta $K \subset \HH$.
Daher ist $\theta_A(z)$ holomorph af $\HH$.
\end{lemm}

\begin{bewe}
Siehe \autoref{lemm:konvergenz_theta} (Im Beweis wurde nicht benutzt, dass $A$ ganzzahlig gerade ist).
\end{bewe}

\begin{satz}[Theta-Transformationsformel]\label{satz:theta_trafo}
Es gilt
\[
    \theta_{A^{-1}}\Big(-\frac{1}{z}\Big)
    = \sqrt{\det A} \cdot \Big(\frac{z}{i}\Big)^{\frac{m}{2}} \theta_A(z)
    \qquad \text{für alle } z \in \HH
    \,.
\]
Hierbei ist $(\frac{z}{i})^{\frac{m}{2}} = e^{\frac{m}{2}\Log(\frac{z}{i})}$.
\end{satz}

\begin{beme}
Ist $A > 0$, dann ist $A^{-1} > 0$ (denn $A^{-1} = A[A^{-1}]$).
\end{beme}

\begin{bewe}
Sei $z\in\HH$ fest.
Für $w\in\CC^m$ setze
\[
    f(w) := \sum_{g \in \ZZ^m} e^{\pi i A[g+w]z}
    \,.
\]
Dann gilt
\[
    \theta_A(z) = f(0)
    \,.
\]

\emph{Behauptung}: Die Reihe $f(w)$ ist auf Kompakta $K \subseteq \CC^m$ gleichmäßig, absolut konvergent.
Insbesondere ist $f(w)$ stetig und separat in jeder Variablen holomorph.

\emph{Denn}: Da $A$ positiv definit ist, existiert $\delta > 0$ mit $A[g] \geq \delta \norm{g}^2 = \delta g^tg$ für alle $g \in \ZZ^m$ (siehe Abschnitt 1).
Betrachte also
\[
    \Im(A[g+w]z)\,.
\]
Schreibe $A[g+w] = A[g] + 2w^tAg + A[w]$.
Daher
\begin{align}\label{eq:theta_transfo}
    \Im(A[g+w]z)
    &= A[g]y + 2\Im(w^tAgz) + \Im(A[w]z) \nonumber \\
    &\geq y\delta g^tg + 2\Im(w^tAgz) + \Im(A[w]z)
    \,.
\end{align}
Sei $K \subset \CC^m$ kompakt.
Da die Abbildung $g \mapsto g^tg$ quadratisch und $g \mapsto \Im(w^tAgz)$ linear ist für alle $z \in K$ und für fast alle $g \in \ZZ^m$ die rechte Seite von \eqref{eq:theta_transfo} $ \geq \frac{1}{2}y\delta g^tg$.

Es gilt daher für alle $w\in\CC^m$
\[
    \sum_{g\in\ZZ^m} \abs{e^{\pi i A[g+w]z}}
    = \sum_{g\in\ZZ^m} e^{-\Im(\pi A[g+w]z)}
    \leq C + \sum_{g\in\ZZ^m} e^{-\frac{\pi y \delta}{2} g^tg}
    < \tilde{C} < \infty
    \,.
\]

\medskip

Wegen absoluter Konvergenz der Reihe $f(w)$ gilt
\[
    f(w+h) = f(w)
    \qquad \text{für alle } h \in \ZZ^m
    \,.
\]
Daher hat $f(w)$ eine Fourierentwicklung der Form
\[
    f(w)
    = \sum_{h\in\ZZ^m} a(h) e^{2\pi i h^tw}
    \qquad \text{für } w \in \CC^m
    \,,
\]
die gleichmäßig absolut auf Kompakta konvergiert. Mit Fourierkoeffizienten
\[
    a(h) = \int_0^1 \ldots \int_0^1 f(w)e^{-2\pi i h^tw} \opd u
    \qquad w = u+iv, v\in\RR^m \text{ fest}
\]
Für $m=1$ wird dies im Detail in \autoref{satz:fourierentwicklung} gezeigt, der Fall $m \geq 1$ folgt aus dem Fall $m=1$ und Induktion.

Es folgt
\[
    a(h) = \int_{I_m} f(w)e^{-2\pi i h'w} \opd u
    \qquad \text{wobei } I_m = [0,1]^m \subset \RR^m \text{ (Fubini)}\,.
\]

Einsetzen liefert
\[
    a(h)
    = \int_{I_m} \left(\sum_{g\in\ZZ^m} e^{\pi i(A[g+w]z-2h^tw])}\right) \opd u
    \,.
\]
Vertauschen von Summation und Integration ist gerechtfertig, wegen der absoluten Konvergenz, also gilt:
\begin{align*}
    a(h)
    &= \sum_{g\in\ZZ^m} \left(\int_{I_m} e^{\pi i(A[g+w]z-2h^tw])} \opd u\right) \\
    &= \sum_{g\in\ZZ^m} \left(\int_{g+I_m} \underbrace{e^{\pi i(A[w]z-2h^tw])}}_{\text{unabhängig von $g$}} \opd u\right) \qquad (u \mapsto u-g) \\
    &= \int_{\bigcup\limits_{g\in\ZZ^m} g+I_m} e^{\pi i(A[w]z-2h^tw])} \opd u
        \qquad (\text{Poissonscher Summationstrick}) \\
    &= \int_{\RR^m}  e^{\pi i(A[w]z-2h^tw])} \opd u
    \,.
\end{align*}
Quadratische Ergänzung (ähnlich wie für $m=1$, nachrechnen), es gilt
\[
    A[w]z - 2h^tw
    = zA[w - A^{-1}hz^{-1}]-A^{-1}[h]z^{-1}
    \,.
\]
Damit erhalten wir
\begin{align*}
    a(h)
    &= e^{-\pi iA^{-1}[h]z^{-1}} \int_{\RR^m} e^{\pi i z A[w-A^{-1}hz^{-1}]} \\
    &\qquad \left( w = u+iv, \text{ wähle } v = \Im(A^{-1}hz^{-1}),\ u\mapsto u + \Re(A^{-1}hz^{-1}) \right) \\
    &= e^{-\pi iA^{-1}[h]z^{-1}} \int_{\RR^m} e^{\pi iA[u]z} \opd u
\end{align*}

Also haben wir
\[
    \theta_A(z)
    = f(0)
    = \sum_{h\in\ZZ^m} a(h)
    = \underbrace{\sum_{h\in\ZZ^m} e^{\pi i(-\frac{1}{z})A^{-1}[h]}}_{=\theta_{A^{-1}}(-\frac{1}{z})} \cdot \left( \int_{\RR^m} e^{\pi iA[u]z} \opd u \right)
    \,.
\]

Wir müssen nun nur noch das Integral berechnen.
Dafür wollen wir $A$ diagonalisieren:
Nach LA existiert $U \in \GL_n(\RR)$ mit
\[
    A[U] = \begin{pmatrix}d_1 & &\\ & \ddots & \\ & & d_m\end{pmatrix} = D^tD = E[D]
    \qquad \text{mit } D = \begin{pmatrix}\sqrt{d_1} & &\\ & \ddots & \\ & & \sqrt{d_m}\end{pmatrix}
\]
Also $A = E[DU^{-1}] = B^tB$ mit $B := DU^{-1} \in \GL_n(\RR)$.
Wir erhalten
\begin{align*}
    \int_{\RR^m} e^{\pi iA[u]z} \opd u
    &= \int_{\RR^m} e^{\pi i(Bu)^t(Bu)z} \opd u \\
    &= \abs{\det B^{-1}} \int_{\RR^m} e^{\pi iu^tuz} \opd u \qquad (u \mapsto B^{-1}u) \\
\end{align*}
Die Substitution $u \mapsto B^{-1}u$ hat die Jacobi-Determinante $\det B^{-1}$.
Aus $A = B^tB$ folgt $\det A = (\det B)^2$, also $\abs{\det B^{-1}} = (\sqrt{\det A})^{-1}$.
Mit mehrfacher Anwendung von Fubini folgt
\begin{align*}
    \int_{\RR^m} e^{\pi iA[u]z} \opd u
    &= (\det A)^{-\frac{1}{2}} \int_{\RR^m} e^{\pi i (u_1^2 + \ldots + u_m^2)z} \opd u \\
    &= (\det A)^{-\frac{1}{2}} \int_{-\infty}^\infty\ldots\int_{-\infty}^\infty e^{\pi i (u_1^2 + \ldots + u_m^2)z} \opd u
    = \left( \int_{-\infty}^\infty e^{\pi i t^2 z} \opd t \right)^m
\end{align*}

Berechne $\int_{-\infty}^\infty e^{\pi i t^2 z} \opd t$ zunächst für $z = iy$, $y > 0$.

\begin{lemm-ind}\label{lemm:theta_trafo}
\[
    \int_{-\infty}^\infty e^{\pi iyt^2} \opd t = y^{-\frac{1}{2}}
\]
\end{lemm-ind}

\begin{bewe-ind}
Setze $t = \frac{x}{\sqrt{\pi y}}$.
Dann
\[
    \int_{-\infty}^\infty e^{\pi iyt^2} \opd t
    = \frac{1}{\sqrt{\pi y}} \int_{-\infty}^\infty e^{-x^2} \opd x
    \,.
\]
Also zu zeigen
\[
    \int_{-\infty}^\infty e^{-x^2} \opd x = 2\int_0^\infty e^{-x^2} \opd x
    = \sqrt{\pi}
    \,.
\]
Eulersche Ergänzungsstaz (\autoref{satz:gamma-eigenschaften})
\[
    \Gamma(s) \Gamma(1-s) = \frac{\pi}{\sin(\pi s)}
    \,,
\]
setze $s = \frac{1}{2}$, dann $\Gamma(\frac{1}{2}) = \pi$, also $\Gamma(\frac{1}{2}) = \sqrt{\pi}$.
Euler Integral (\autoref{satz:eulerint})
\[
    \Gamma(s) = \int_0^\infty t^{s-1}e^{-t} \opd t \qquad (\Re(s) > 0)\,.
\]
Damit folgt für $s=\frac{1}{2}$
\[
    \sqrt \pi = \Gamma(\frac{1}{2}) = \int_0^\infty t^{-\frac{1}{2}}e^{-t} \opd t
    = \int_0^\infty x^{-1}e^{-x^2}\cdot 2x \opd x %t = x^2
    = 2 \int_0^\infty e^{-x^2} \opd x
\]
\end{bewe-ind}

Die Funktion $(\frac{z}{i})^{-\frac{m}{2}}$ und $(\int_{-\infty}^\infty e^{\pi i t^2 z} \opd t)^m$ sind auf $\HH$ holomorph und stimmen auf der positiven imaginären Achse überein (\autoref{lemm:theta_trafo}).
Nach dem Identitässatz stimmen sie also auf $\HH$ überein.
Damit folgt die Behauptung.
\end{bewe}

\begin{satz}
Sei $A \in M_m(\ZZ)$, $A$ symmetrisch, positiv definit, gerade und $\det A = 1$.
Dann gilt $8 | m$ und $\theta_A \in \M_{\frac{m}{2}}$.
\end{satz}
\begin{beme}
Solche $A$ existieren und können konstruiert werden mit Hilfe der Theorie der unimodularen Gitter.
\end{beme}

\begin{bewe}
Zunächst beachte: Ist $A \in M_m(\RR)$, $A$ symmetrisch, positiv definit und $B = A[U]$ mit $U \in \GL_m(\ZZ)$, so gilt $\theta_B(Z) = \theta_A(z)$.

\emph{Denn}:
\[
    \theta_B(z)
    = \sum_{g \in \ZZ^m} e^{\pi iB[g]z}
    = \sum_{g \in \ZZ^m} e^{\pi i A[U][g]z}
    = \sum_{g \in \ZZ^m} e^{\pi i A[Ug]z}
\]
denn wegen $U\in\GL_m(\ZZ)$, durchläuft mit $g$ auch $Ug$ alle Elemente aus $\ZZ^m$.

Sei $\det A = 1$.
Dann ist $A^{-1} = A[U]$ mit $U = A^{-1}$ mit $U = A^{-1} \in \GL_m(\ZZ)$.
Aus den Voraussetzungen folgt damit
\[
    \theta_A = \theta_{A^{-1}}
    \,.
\]
Also folgt mit \autoref{satz:theta_trafo}
\[
    \theta_{A}\Big(-\frac{1}{z}\Big)
    = \sqrt{\det A} \cdot \Big(\frac{z}{i}\Big)^{\frac{m}{2}} \theta_A(z)
    \,.
\]

\emph{Angenommen}: $8 \nmid m$.

Dann kann man annehmen $m \equiv 4 \mod 8$, denn man kann $A$ immer ersetzen durch
\[
    (**)\quad \begin{pmatrix}A&\\&A\end{pmatrix} \qquad \text{oder} \qquad (*) \quad \begin{pmatrix}A\\&A\\&&A\\&&&A\end{pmatrix}
\]
und diese Matrizen haben den gleichen Typ.

\begin{tabular}{l l}
$m \equiv 1$ & nehme (**) dann $m \equiv 4 \mod 8$ \\
$m \equiv 2$ & nehme (*)\phantom{*} dann $m \equiv 4 \mod 8$ \\
$m \equiv 3$ & nehme (**) dann $m \equiv 4 \mod 8$ \\
$m \equiv 4$ & \ldots
\end{tabular}

Sei also $m \equiv 4 \mod 8$.
Dann gilt $\theta_A(-\frac{1}{z}) = z^{\frac{m}{2}} \cdot i^{-\frac{m}{2}} \theta_A(z)$.
Also $\theta_A|_{\frac{m}{2}}S = \theta_A$. Aber $A$ gerade, also $\frac{1}{2}g^tAg = \frac{1}{2}A[g] \in \ZZ$.
Also Fourierentwicklung von $\theta_A$ ist in $e^{2\pi iz}$, d\,h. $\theta|_{\frac{m}{2}}T = \theta_A$.
Also $\theta_A|TS = \theta_A|T|S = - \theta_a$.
Also $\theta_A = \theta_A|(-E) = \theta_A|(TS)^3 = \theta_A|TS|TS|TS = -\theta_A$.
Also $\theta_A = 0$ \blitz konstanter Term ist 1.

Also $8|m$, Aber dann $\theta_A|T = \theta_A$, $\theta_A|S = \theta_A$. Also $\theta_A \in M_{\frac{m}{2}}$.
\end{bewe}
\section{Valenzformel und Anwedungen}

\begin{erin}
$f\colon \dot{U}_r(a) \ra \CC$ holomorph mit $f\not\equiv 0$ und $a$ keine wesentliche Singularität.
Dann besitzt $f$ eine Laurent-Entwicklung um $a$
\[
	f(z) = \sum_{n=-\infty}^\infty a_n(z-a)^n
	\qquad \text{für } z \in \dot{U}_r(a)
\]
und fast alle $a_n$ mit $n < 0$ sind gleich 0. Dann
\[
	\ord_a f := \min\Set{n \in \ZZ \mid a_n \not= 0}
	\,.
\]
\end{erin}

\begin{defi}
Sei $f$ eine Modulfunktion (siehe \autoref{def:modulfunktion}) vom Gewicht $k$.
Man setzt
\[
	\ord_\infty f := \ord_0 F
\]
mit $F(q) = f(z)$ für $q = e^{2\pi iz}$.
\end{defi}

\begin{satz}[Valenzformel]
Sei $f$ eine Modulfunktion vom Gewicht $k$, $f\not\equiv 0$.
Dann gilt
\[
	\ord_\infty f + \frac{1}{2} \ord_i f + \frac{1}{3} \ord_\rho f + \sum_{\substack{z \in \modulo{\HH}{\Gamma(1)} \\ \scriptscriptstyle z \not\sim i, z\not\sim \rho}} \ord_z f = \frac{k}{12}
\]

mit $\rho = e^{\frac{2\pi i}{3}}$.
\emph{Beachte} Die Vorfaktoren oben sind gerade die Ordnung der Gruppe $\modulo{\Gamma(1)_z}{\pm E}$.
\end{satz}
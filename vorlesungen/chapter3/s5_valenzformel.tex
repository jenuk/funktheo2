\section{Valenzformel und Anwedungen}

\begin{erin}
$f\colon \dot{U}_r(a) \ra \CC$ holomorph mit $f\not\equiv 0$ und $a$ keine wesentliche Singularität.
Dann besitzt $f$ eine Laurent-Entwicklung um $a$
\[
	f(z) = \sum_{n=-\infty}^\infty a_n(z-a)^n
	\qquad \text{für } z \in \dot{U}_r(a)
\]
und fast alle $a_n$ mit $n < 0$ sind gleich 0. Dann
\[
	\ord_a f := \min\Set{n \in \ZZ \mid a_n \not= 0}
	\,.
\]
\end{erin}

\begin{defi}
Sei $f$ eine Modulfunktion (siehe \autoref{def:modulfunktion}) vom Gewicht $k$.
Man setzt
\[
	\ord_\infty f := \ord_0 F
\]
mit $F(q) = f(z)$ für $q = e^{2\pi iz}$.
\end{defi}

\begin{satz}[Valenzformel]\label{satz:valenzformel}
	Sei $f$ eine Modulfunktion vom Gewicht $k$, $f\not\equiv 0$.
	Dann gilt
	\[
	\ord_\infty f + \frac{1}{2} \ord_i f + \frac{1}{3} \ord_\rho f + \sum_{\substack{[z] \in \linksmodulo{\Gamma(1)}{\HH} \\ \scriptscriptstyle z \not\sim i, z\not\sim \rho}} \ord_z f = \frac{k}{12}
	\]
	
	mit $\rho = e^{\frac{2\pi i}{3}}$.
	\emph{Beachte} Die Vorfaktoren oben sind gerade die Ordnung der Gruppe $\modulo{\Gamma(1)_z}{\pm E}$.
\end{satz}

\begin{beme-list}\label{beme:valenzformel}
	\item $\ord_z f$ hängt nur von der Klasse $[z]$ in $\linksmodulo{\Gamma(1)}{\HH}$ ab, denn
	\[
	f\bigg(\frac{az+b}{cz+d}\bigg) = \underbrace{(cz+d)^k}_{\not=0} f(z)
	\qquad \text{für } \abcd \in \Gamma(1)
	\,.
	\]
	
	\item Die linke Seite hat nur endlich viele Summanden, d.\,h. $f$ hat nur endlich viele Null- und Polstellen modulo $\Gamma(1)$, \emph{denn}
	sei $\F = \Set{z=x+iy \mid \abs z > 1,\ \abs x < \frac{1}{2}}$ der \emph{Standard Fundamentalbereich} für $\SL_2(\ZZ)$.
	Dann ist jedes $z\in\HH$ äquivalent zu einem Punkt in $\closure \F$ und es genügt zu zeigen, dass $f$ nur endlich viele Null- und Polstellen in $\closure\F$ hat.
	Nach Voraussetzung ist $f$ auf $\HH$ und in $\infty$ meromorph und $f \not\equiv 0$. Daher ist $F$ in $\abs q < 1$ meromorph, $F \not\equiv 0$, also sind die Pole und Nullstellen von $F$ isolierte Punkte.
	Daher hat $F$ in $0 < \abs q < e^{-2\pi \delta}$ für ein geeignetes $\delta > 0$ keine Null oder Polstellen, d.\,h. $f$ hat in $y > \delta$ keine Null- oder Polstellen.
	Aber $\closure\F = \Set{z \in \closure\F \mid y \leq \delta} \cup \Set{z \in\closure\F \mid y > \delta}$ und die erste Menge ist kompakt, also hat $f$ dort nur endlich viele Null- und Polstellen.
\end{beme-list}

\begin{koro}\label{koro:dimMk}
	Es ist $\dim_\CC \M_k < \infty$. Genauer ist $\M_k = \Set{0}$ für $k < 0$ und $\dim M_k \leq \floor{\frac{k}{12}} + 1$ für $k \geq 0$.
\end{koro}

\begin{bewe-list}
	\item Sei $k < 0$. \emph{Angenommen} es gibt $f \in \M_k$ mit $f\not\equiv 0$. Alle Terme links in der Formel sind $\geq 0$, denn $f$ ist holomorph auf $\HH$ und in $\infty$. Aber rechts steht eine Zahl $< 0$. Dies ist ein Widerspruch
	
	\item Sei $k \geq 0$. Sei $N := \floor{\frac{k}{12}} \geq 0$. Sei
	\[
	\phi\colon \M_k \ra \CC^{N+1},
	\quad f \mapsto \underbrace{(a_0(f), a_1(f), \ldots, a_N(f))}_{\text{Fourierkoeffizienten von $f$}}
	\,.
	\]
	Dann ist $\phi$ linear.
	
	\emph{Beh.} $\phi$ ist injektiv.
	Sei $\phi(f) = 0$, also $a_0 = a_1 = \ldots = a_N = 0$.
	\emph{Angenommen} $f \not\equiv 0$.
	Dann gilt $\ord_\infty f \geq N + 1$.
	Nach der Valenzformel folgt
	\[
	\floor{\frac{k}{12}} + 1
	= N + 1
	\leq \ord_\infty f
	\leq \ord_\infty f + (\ldots)
	= \floor{\frac{k}{12}}
	\,.
	\]
	Dies ist ein Widerspruch, also ist $\phi$ injektiv.
	Es folgt, dass $\dim\M_k = \dim\phi(M_k) \leq \dim \CC^{N+1} = N + 1$.
\end{bewe-list}
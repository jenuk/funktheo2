\newdate{26. Oktober 2017}

\section{Unendliche Produkte}

Gegeben sei eine Folge $(p_n)_{n\in\NN}$ komplexer Zahlen.
Wir wollen nun in sinnvoller Weise das unendliche Produkt
\[
	\prod _ {n=1} ^\infty p_n
\]
definieren.
Ein naheliegender Vorschlag dafür ist:
$\prod _ {n=1} ^\infty p_n$ heißt konvergent, falls die Folge $(P_n)_{n\in\NN}$ der Partialprodukte $P_N = \prod _{n=1}^N p_n$ konvergent ist.
In diesem Fall setzen wir
\[
	\prod_{n=1} ^\infty p_n := \lim_{N\to\infty} P_n =: P\,.
\]

Das Problem was sich mit dieser Definition stellt ist, dass falls einer der Faktoren Null ist, so ist der Wert des Produktes gleich $0$.
Dieses hängt also gar nicht von der Gesamtheit der Faktoren ab.
Ferner möchte man oft $\prod _{n=1}^N p_n$ bzw. $P$ mit der Summe $\sum _{n=1}^N \log p_n$ bzw. mit $\log P$ vergleichen.
Und das geht nur falls $p_n\not=0$ für alle $n\in\NN$ und entsprechend $P\not=0$.
Später wollen wir allerdings holomorphe Funktionen als Produkte darstellen, dies sollte auch möglich sein, wenn diese Nullstellen haben.

\begin{defi}[Unendliches Produkt]
Sei $(p_n)_{n\in\NN}$ eine Folge in $\CC$ derart, dass nur endlich viele der $p_n$ Null sind.
Sei $m\in \NN$ der größte Index mit $p_m = 0$ (und $m:=0$, falls $p_n \not= 0$ für alle $n\in\NN$).
Dann heißt das \myemph{unendliche Produkt}
\[
	\prod_{n=1}^\infty p_n 
\]
konvergent, falls der Limes
\[
	\lim_{\substack{N\to\infty\\ \scriptscriptstyle N \geq m+1}} P_n \text{ mit } P_N = \prod _ {n=m+1} ^N p_n
\]
existiert und ungleich Null ist.
Man setzt dann
\[
	\prod _ {n=1} ^\infty p_n :=
	\begin{cases}
		\lim\limits_{N\to\infty} P_n &\text{falls } m=0 \\
		0 &\text{sonst}
	\end{cases}
	\,.
\]
Dabei ist zu beachten, dass nach Definition ein konvergentes unendliches Produkt den Wert 0 genau dann hat, wenn ein Faktor gleich Null ist.
\end{defi}

\begin{bsp}
\begin{enumerate}
\item Das unendliche Produkt $\prod_{n\geq2}(1-\frac{1}{n^2})$ ist konvergent und hat den Wert $\frac{1}{2}$.
\begin{bewe} Zunächst sind alle Faktoren ungleich Null und 
\begin{align*}
	P_n
	&= \prod_{n=2}^N \left(1-\frac{1}{n^2}\right)
	= \prod_{n=2}^N \frac{(n-1)(n+1)}{n^2} \\
	&= \frac{(2\cdot3\cdot\,\ldots\,\cdot(N-1))\,\cdot\,(3\cdot4\cdot\,\ldots\,\cdot(N+1)}{(2\cdot3\cdot\,\ldots\,\cdot N)\,\cdot\,(2\cdot3\cdot\,\ldots\,\cdot N)} \\
	&= \frac{1}{N}\frac{N+1}{2} \\
	&= \frac{1}{2}\left(1+\frac{1}{N}\right) \\
	&\xto{N \to \infty} \frac{1}{2}
\end{align*}
\end{bewe}

\item $\prod_{n\geq1}(1-\frac{1}{n^2}) = 0\cdot \prod_{n\geq2}(1-\frac{1}{n^2})$ ist konvergent und hat Wert 0

\item $\prod_{n=1}^\infty \frac{1}{n}$ ist nicht konvergent in unserem Sinn.
Denn
\[
	P_N
	= \prod_{n=1}^N \frac{1}{n} = \frac{1}{N!}\xto{N\to\infty}0
	\,.
\]
\end{enumerate}
\end{bsp}

\begin{satz}
\begin{enumerate}
\item Ist $\prod_{n=1} ^\infty p_n$ konvergent, so gilt notwendigerweise $\lim\limits_{n\to\infty}p_n=1$.
\item Sei $p_n \not= 0$ für alle $n\in \NN$.
Dann ist $\prod_{n=1} ^\infty p_n$ konvergent genau dann, wenn
\[
	\sum _{n=1}^\infty \Log p_n
\]
konvergiert. (Erinnerung $\Log z = \log \abs z + i\Arg z$ der Hauptwert des Logarithmus und $-\pi < \Arg z \leq \pi$ das Argument von $z$.)
Insbesondere ist $\prod_{n=1} ^\infty p_n = P$, so existiert $h\in\ZZ$ so dass 
\[
	\sum _{n=1}^\infty \Log p_n = \Log P + 2\pi ih
\] 
gilt.
Ist umgekehrt $S = \sum _{n=1}^\infty \Log p_n$, so gilt
\[
	e^S = \prod_{n=1} ^\infty p_n
\]
\end{enumerate}
\end{satz}

\begin{bewe}
\begin{enumerate}
\item Es ist
\[
	p_{N+1} = \frac{P_{N+1}}{P_N} \xto{N\to\infty} \frac{P}{P} = 1
\]
für $N\geq m+1$,hierbei benutzt man $p_n \not= 0$ für $n\geq m+1$ und $P \not= 0$.
\item Es gelte
\[
	S = \sum _{n=1}^\infty \Log p_n
\]
Also $S = \lim \limits _{N\to\infty} S_N$ mit $S_n = \sum_{n=1}^N \Log p_n$.
Da $\exp$ stetig ist, folgt
\begin{align*}
	0
	\not= e^S
	&= \lim _ {N\to \infty} e^{S_n} = \lim_{N\to\infty} e^{\log p_1 + \ldots + \log p_n} \\
	&= \lim _ {N\to \infty} e^{\log p_1} \cdot \ldots \cdot e^{\log p_n}
	= \lim _ {N\to \infty} (p_1 \cdot \ldots \cdot p_N) \\
	&= \lim _ {N\to \infty} \prod _{n=1}^N p_n
	= P
\end{align*}
\end{enumerate}
\end{bewe}

\begin{satz}
\begin{enumerate}
\item Sei $1+a_n \not= 0$ für $n \geq 1$. Dann ist 
\[
	\sum _{n=1}^\infty \Log (1+a_n)
\]
genau dann absolut konvergent, wenn $\sum_{n=1}^\infty a_n$ absolut konvergiert.

\item Sei $\sum_{n=1}^\infty a_n$ absolut konvergiert. Dann ist $\prod _{n=1}^\infty (a+a_n)$ konvergent.
Außerdem ist das Produkt unbedingt konvergent, d.\,h. jede Umordnung konvergiert und hat den gleichen Limes.
\end{enumerate}
\end{satz}